\subsubsection{Knowledge about tracking and privacy}
\label{sec:methods-tracking}

\underline{Privacy} Interviewees scored a median of 2 on General Caution scale
and a median of 3 on Privacy Concern scale, which show that they were about
below to average in terms of privacy attitudes. Nevertheless, interviewees had
a median score of 4 on Technical Protection Scale. Such rating could indicate
that our interviewees were less concerned  about the their privacy because
they actively acted against its violation. 

On the SeBIS scale, interviewees scored a median of 5 on Device Securement,
3,5 on Password Generation, 3 on Proactive Awareness and a median of 3 on
Updating. On the whole SeBIS scale, interviewees had a median score of 3, so
an average score of Security Behavior Intentions with an above average rating
of Device Securement.  \underline{Tracking} Before the questions about VPNs,
we asked our interviewees about their general knowledge about tracking. 

\textbf{What is Tracking}

Twelve interviewees defined \textit{Online tracking} as creating data about an
individual, and gathering data about individual's location (5). More than half
(17) of interviewees said that it happens through user's search history and
general online activity. Another 2 that it can be done through web camera but
also through apps (1) or looking at key logger passwords (1). Also, for 12
interviewees  \textit{tracking} meant that someone could see what they are
doing online and 6 believed that it happens each time you go online, while 3
stated that you are not able to hide.

\begin{quote}To me that would mean collecting data about a user in any service
or any capacity while they're online or doing anything that involves being
online. That could be something like keeping a lot of the messages you sent,
even if you delete them or that could be something as simple as just tracking
what times you sign on to different things. I think that also applies to
things like, with our [Proxy] cards, every time we buzz into a building that's
documented somewhere and that's documented online, so that kind of goes into
online tracking as well. Anytime we're using an online system, information we
recorded. (P32)\end{quote}



\textbf{Comfort with Tracking} For 4 interviewees tracking was frightening,
and tracking made 2 interviewees' overall online experience uncomfortable.
Moreover, 8 interviewees believed that tracking was something bad that should
not have place. For example, P13 believed that tracking is more visible online
than censorship.
Moerover, P25 did not feel like there was any good law in place that would
make users' data safe, which led them to believe that companies have low
standards of data processing and make users more vulnerable. 

On the other hand 7 interviewees interpreted it as something good, for example
using it as a tool for crime investigation (2) or suggesting ads with clothing
(2). 

\begin{quote} it [online tracking] could mean positive tracking. I guess
Google tracks many things that I do and I'm okay with that and I'm aware so I
don't do too personal searches or anything. It's just for personalized
advertising or search results that could be beneficial. (P31) \end{quote}


\textbf{Who Tracks Data} We asked our interviewees who they think track
online. 18 interviewees believed that big companies, such as Facebook (4) and
Google (3) track people online. On the other hand 15 interviewees pointed
government as an institution that tracks, naming NSA (3), FBI (2), CIA (1) and
IRS (1). Furthermore, 8 interviewees stated that every website tracks its
users, 6 that it is done by advertising agencies or data companies. Four
interviewees said that ISP can track its users, another 4 that it is done by
hackers and 2 that school/university tracks your online activity. Five
interviewees said that tracking is not done by anyone in particular.

Twenty-seven interviewees admitted that they believed that they are being
tracked. Thirteen of which believed they are tracked by big companies, such as
Google (3) and another 13 that even though they are tracked, they are not
targeted. 


\begin{quote}I think the way that I, totally uninformed way you think of it is
that there are probably machines, some AI, some program that looks for certain
keywords and a certain combination or indicators, histories. I'm sure there's
some algorithm that people have developed. I don't think someone is
necessarily actively tracking me but I'm sure there is some software that I,
along with many other people are being tracked by and if something I were to
say was flagged on that, then I think there would be an individual, or at
least more attention to me.(P32)\end{quote}


Five interviewees believed that they are not tracked, as they do not do
anything suspicious.


\textbf{Reasons for Tracking}    

We also asked participants about reasons that these entities may have to
track. Fifteen responded that tracking is needed to gather data about users'
preferences, also 15 said that to gather data about our online behavior and 14
that it is needed for advertising. Nineteen believed that tracking is for
financial interest. For example, P25 mentioned that tracking is an essential
part of A/B testing. 

P23 associated online tracking with location services, that track users'
geographical position. They also added about their feeling being watched:

\begin{quote} Essentially, I feel sometimes like, I don't know if you ever
heard of the novel 1984, but it's just like George Orwell, like beautifully
describes just being watched and how were under a lot of supervision or if you
read this philosopher Foucault, who talks about this Panopticon, that's were
constantly being just supervised by a higher power and this power being like
the people who controlling these, this data, cause essentially it is data,
like where you are, it's stored where your cache. And also like the things you
like and the sites you visit it's all start in your cashe, in your history and
a site has the permission to use that cash in order to market things a certain
way for you, for example if you. This is like super intense, but like, I don't
know, I googled a pair of shoes right and then like two seconds, maybe like
five minutes later I was just scrolling through my Instagram feed and I see
those same shoes being on an ad in Instagram, that's like annoying. I mean I
know you want me to buy your shit because it's capitalism and all, but I don't
need to know that you're, I don't need to be like primed into like I want to
buy that thing. That's essentially what I mean by tracking.\end{quote}

On the other hand, P07 described tracking as repercussion of Internet speed
debate. The internet providers would track users to evaluate the number of
people that visit each website. The bigger company with more viewers, would
get faster Internet to users, so Internet providers would control speed of the
Internet for different companies. 







OLD CONCLUSION
Data privacy is an increasingly important concern for Internet users. However,
most Internet users today do not fully understand the Internet [18, 19], and
the US education system does not provide sufficient background on online
technologies [29]. Virtual Private Networks (VPNs) are important tools that
can help safeguard against data-intrusive practices. Although VPNs are growing
in usage and prevalence [3], we lack an understanding of how users perceive
VPNs and why they use them. Our study focuses on college students and asks the
following research questions:

TODO

We employed a mixed-methods approach involving interviews, a survey, and a lab
study.

We found that students have weak understandings of VPNs, both in terms of what
they are and how they work. They use VPN infrequently and primarily for
content access, regarding privacy and security as secondary concerns. Students
consider cost, ease of use, and speed as the most important elements of a VPN.
Most college students use school-provided VPNs, though a substantial number
use commercial VPNs. Students are generally dissatisfied with the speed,
stability, and interfaces of their VPNs.

Our findings have implications for educators, policymakers, and VPN providers
and designers. Educators and VPN providers should become better resources for
users to learn about online privacy and security. VPN providers should also
educate its users more effectively on VPN technology, and also run pricing
analyses. In tandem with other policies in the online privacy space,
policymakers should explore transparency regulations for VPN providers to
disclose their own privacy violations. VPN designers should improve the user
experience of the installation, login, and reconnecting processes of VPNs;
they should also develop designs that better communicate what VPNs are
actively doing.

Our study is a first foray into the perception and usage of VPNs, and
establishes a broad overview of the space through the lens of college
students. Our work provides insights that can inform practical improvements in
policy and design. However, we also highlighted many areas for further
exploration. Future researchers can take deeper dives into the areas that we
have introduced, or corroborate our design recommendations through
comprehensive usability studies. In the long term, Internet users stand to
benefit from a stronger, better-communicated, and better-received VPN
ecosystem.


Almost all respondents (99\%, 348/350) believed that some data was collected
about them when they used the Internet. The vast majority of these respondents
(N=348) believed that companies (93\%, 323/348), websites (93\%, 322/348).
their government (83\%, 290/348), and Internet Service Provider (81\%,
282/348) were collecting their data; in contrast, only 2\% (8/348) of them
believed that friends and family were collecting data on them. Almost all of
these respondents believed that their data was collected for advertising and
other financial motives (99\%, 343/348); a smaller majority believed that
their data was collected for political motives, such as influencing political
leanings (72\%, 252/348). Almost all of these respondents believed that their
online activities (97\%, 339/348), interests and preferences (96\%, 335/348),
and location (96\%, 333/348) were collected. Smaller majorities believed that
demographic information (85\%, 293/348) and device type (81\%, 283/348) were
collected. Significant minorities of respondents believed that more sensitive
data including private messages (41\%, 141/348), keystrokes (31\%, 109/348),
and recordings (31\%, 109/348) were captured. On a five-point Likert scale,
participants were on average "somewhat concerned" about this data collection
(mean 2.94, median 3).

Survey respondents used a wide variety of tools and tactics outside of VPN.
Among the most popular were ad blockers (80\%, 280/350), using two-factor
authentication (75\%, 263/350), avoiding spam email (70\%, 246/350), and using
private browsing mode (63\%, 222/350). Fewer respondents utilized high-effort
tactics such as changing passwords frequently (15\%, 52/350), using password
managers (17\%, 60/350), or avoiding social media accounts (27\%, 95/350).
More obscure online tools were also less popular among respondents, with only
12\% (43/350) using tracker blockers such as Ghostery and 9\% (33/350) using
Tor. Most respondents (57\%, 198/350) used at least some of these tools most
of the time when they go online, on both laptops (98\%, 343/350) and phones
(77\%, 271/350). However, the overall effort that respondents put into
protecting themselves online was low. Out of a maximum score of 35, the median
"protection effort score" of respondents was only 12 (see section 3.5 for how
this statistic was derived). Participants largely heard about these tools and
tactics online (73\%, 256/350) and from friends and family (71\%, 248/350),
and most (64\%, 225/350) started using these tools and tactics 3 or more years
age



Also, P13 did not think the VPN is secure, even the institutional one:
\begin{quote}Probably nothing. I think it's pretty open so whatever security
wherever you're [inaudible] into. Because I'm going to the university network,
I know there security is there. But depending on where I've been, those
securities have been better or worse. So at Rutgers ... I did my Ph.D. at
Rutgers, and they had a huge hack shut down the whole system. I mean, that was
just a student that did that. I don't think their securities are very good at
all. Here, I haven't heard about any problems, but I don't know. Seems like
the security is a little better here. But it's very easy for me to VPN and
grab my IP off any of the computers in the lab and just get it onto a
computer. I don't think it's that secure. There's no extra steps I have to
take to do that.\end{quote}



\textbf{Perfect VPN} \textbf{Performance} We asked our interviewees what, in
their opinion, "perfect" VPN would provide. 9 interviewees listed that they
would like their VPN to operate at the same speed and latency as  “normal”
Internet. For at least 8 interviewees reliability was a key component of a
perfect VPN and for 7 it was security. 

\textbf{Security} Also 6 interviewees mentioned that perfect VPN would not
access any personal information, 6 that it would not hand over or sell data to
other companies. 

\textbf{Data practices} At least 5 interviewees believed that VPN should be
transparent and also 5 that it should serve only users' interest. But as P25
observed, this could not make usage better, since users do not get information
about their data being stored anyway.


\textbf{Access} Moreover, 8 interviewees believed that number of locations
would be important, 7 added that perfect VPN would keep offering location
spoofing and 4 that ideal VPN would provide an access to absolutely
everything.

P18 described their perfect VPN: \begin{quote}I would look to some sort of a
    VPN that operates in another country, free from the jurisdiction of
    whatever country I'm residing in, because I know that in some countries
    it's harder for the United States to work with a VPN or whatever companies
    if the VPN is hosted in a country that's not really friendly towards the
    United States, that would be interested in having people from the
    government of the US working in their country. I think that would be a big
    plus.


I think another big plus would be that it's not relatively well known,
ironically. That goes against sort of what I said before, that it has good
reviews. At the same time, I don't want many other people knowing about the
existence of that VPN because I don't want it coming up on the radar of say
the US government scouring these VPN's looking to get information on people.
It's also not profitable. It's not profitable just to go for people who are
downloading some movies or something like that, to go halfway across the world
just to get a few IP's that are going to lead nowhere, relatively speaking.
It's not profitable.\end{quote}



As well as P18: \begin{quote}The usage of this VPN sort of hinges on two
things, the desire to obtain copyrighted material for free and also the
knowledge of the existence of VPN's. Those are two I think pretty big
bottlenecks that sort of limit this sort of information to sort of tech nerds,
if that makes sense.\end{quote}

For example, P20 was not sure how the IP address was changed: \begin{quote} As for how they're making it look like you're using that IP
address rather than your actual IP address, I actually don't know.\end{quote}

Notably, having a higher level of self-reported VPN knowledge was correlated
with a number of different surveyed variables. Students who were Computer
Science or Engineering majors tended to have more knowledge about VPNs.
Students with more VPN knowledge tended to use other online privacy and
security tools more often, and tended to use paid VPN options and feel safer
when doing so. These students were also more likely to hear about VPNs online,
and were more likely to do their own research to learn how they work (R=.35).
In terms of VPN usage, students with more VPN knowledge tended to first use
VPNs 3 or more years ago, use VPNs more often, use VPNs for privacy and
security reasons, and still use VPNs at the time of the survey. In addition,
students with higher VPN knowledge tended to put more effort into verifying
that their VPN was working, and were more likely to not think that their VPN
provider was collecting their data.

We also tested if students who were Computer Science or Engineering majors had
particular knowledge patterns for VPNs, as they may have learned about VPNs
through their studies or have other affinities for technologies. We found that
Computer Science and Engineering majors tended to have higher VPN knowledge
and put more effort into verifying that their VPN was working. In addition,
Computer Science and Engineering majors were more likely to first hear about
VPN from online research. Correlations were generally small, however (R<.2).

Our survey participants had weak knowledge of VPNs overall, with very few
respondents indicating that they had high or expert-level knowledge of VPNs.
However, the short-answer responses for what a VPN is showed a greater
disconnect between students’ perceived and actual knowledge of VPNs. Though a
majority of respondents indicated that they had "some knowledge" of VPNs, a
vast majority of the responses about what a VPN was only referred to VPN
features and purposes. We can observe that the respondents equate what a VPN
does and what it is commonly used for to what a VPN is. Though only a small
minority of respondents indicated that they do not know how VPNs work at all,
it appears that a majority of respondents do not really know what a VPN is or
how it works.

\textbf{Usage in different countries}
%Think about home country/other countries 
P15, one of our interviewees from Romania, believed that people from US are
more afraid of using VPN than in their home country: \begin{quote}I feel like
people are more likely to buy their stuff rather than try to get it for free,
while at home no one cares, so they're just gonna go for the free thing
because it's easier and you're never gonna see someone getting accused for, of
a, and getting a criminal record for downloading a game. It's only if you did
some sort of, I don't know, WikiLeaks type of thing. But, yeah.\end{quote}


\textbf{Why others use VPN}We also asked what other people's reasons to use
VPN are. 5 believed that others use VPN for privacy. 11 interviewees replied
that others use it for watching something online and 3 that for piracy.

As P18 described what VPN is: \begin{quote} Just a way for people to get
around government agencies looking at their IP address so that they can
download whatever they want from wherever they want within reason because
these VPN's are also subject to their own vulnerabilities. These IP addresses,
some VPN's have policies where they will provide information from IP addresses
if they're requisitioned from a government entity. That's pretty much it with
regards to VPN's, just used to download information. Tore isn't really used
for that because it's slower, because of the technical architecture of Tore.
VPN's are just used mostly for that.\end{quote}

and why people would use it: \begin{quote}Oh, so VPN's with regards to
downloading things illegally. People use VPN's to maybe get copyrighted
material but don't want their IP address to be tracked by law enforcement
agencies or recorded by their internet service provider. Just a plain old
dummy would probably go somewhere, to a file hosting site like the pirate bay
or mega upload or something like that, that have lists of torrents. (\dots) I
also know that VPN's are used in engaging illicit commerce on websites that
\dots A famous example is the silk road which was taken down back in 2013.
\end{quote}



\mc{unclear what the points mean} \textit{Security} scored 77 and was the most important for
our participants. Second was \textit{Fast} scoring 54 and the least important
was \textit{Cheap}, with 41 points.


\subsubsection{Privacy policy, have you read?} We also wanted to know whether
our interviewees read their VPN providers’ privacy policies. 24 interviewees
replied that they did not read it and 11 of them added that they never read
any privacy policies. Also, for 11 interviewees it was too long and not
readable. Moreover, 11 of the interviewees who did not read it, believed that
it was not important to read it, for various reasons; 5 believed that it did
not matter whether they would read it or not, because either way you have to
trust your VPN provider (3) or because they would collect their data no matter
what they would write in their privacy policy (2). 2 interviewees believed
that they did not have any valuable information to be worried about and one
believed that reading it would only make them more confused. Among all
interviewees, 6 believed that providing and reading privacy policy was
important. On the other hand 5 interviewees read their VPN provider privacy
policy but 3 of them only looked through it, one did not find anything
specific and one found it standard. 

For example, P11 read it after some of his friends had bad experience with a
different VPN: \begin{quote}I had a good look, because I'd heard from a couple of friends
who were using I think it was Onavo, and that they were kind of worried
that it was owned by Facebook, and it was kind of influencing what they were
getting adverts for, I think.  So, I didn't want to have that lingering
over me when I was using a VPN.  \end{quote}


We also asked our interview participants whether they know if and where their
real IP is visible. Half (16) of the interviewees stated that they did not
know, while 14 believed that their VPN provider would be able to see it. Also,
6 respondents had a sense that such information are anonymous and 2
respondents said that their Internet Service Provider (ISP) is still able to
see it. Moreover, 2 respondents did not find it important.

51\% (179/350) of respondents indicated that they knew how to see their
VPN’s server location; 54\% (190/350) reported that they knew how to view
their IP address under the VPN.

At least 25 interviewees admitted
that they had not prepared themselves and their devices before installing
their VPN.

 We asked our interviewees to rate three
components: secure, fast and cheap, from the most important to least important
when choosing VPN. Most rated security as the most important factor followed by fast and then cheap. This was also reflected in the survey.
We asked our interviewees whether the number of servers
locations that VPN provides and to which they can connect to, is important.
For 12 interviewees it was important, for example 3 reported that there are
different data regulations in different countries and 2 stated that
decentralization was a good thing. Another 2 interviewees simply believed that
more locations give them confidence that they would stay connected: if one
connection would not work, there would be another one. Two other interviewees
believed that bigger number of servers’ locations is useful when travelling.
On the other hand, 11 interviewees did not think that number of servers’
location is important, as long as there would be couple of them (5). 

P26 also used it for their relative to be able to watch a tv show:
\begin{quote}With the university, there
    are certain free textbook websites that they don't like students going on,
    but then I use my VPN to be able to access those sites. For example,
    Venezuela has blocked everything coming from their YouTube channels, and I
    have my mom reroute the US IP address to a Mexican IP address with a VPN,
    so then she could watch her Venezuelan TV shows.\end{quote}
    
    We asked respondents to report, in short-answer form, what they liked and
disliked about their VPNs. We categorized these text responses by topic based
on their contents; some responses contained multiple topics. The ability to
access restricted content was by far the most commonly “liked” quality of
students’ VPNs (63\%, 221/350). Other qualities, including security, privacy,
interface, and transparency received far fewer mentions. Respondents indicated
that they prefer VPNs based on cost, speed, ease of use, privacy, and
security, yet they do not perceive the VPNs that they use as having those
positive qualities.


P13 encountered problems while trying to set up VPN with their friend:
\begin{quote}There were instructions online on how to download it. But they
don't have a very good system. \end{quote}

Moreover, P13 explained how he would like to change security and access
permission around institutional VPN: \begin{quote}Okay. I just would have more
    security around it. It just doesn't feel secure to me. It doesn't ... it's
    too easy to access, too easy for me to get into pretty much anything I
    wanted here, which is good for my work for bad if I wasn't doing work.
    Let's say I had a vindictive student who then could go in and delete all
    our files or put some kind of virus in the computer in the lab. There's no
    protections against that. They have full access just as much as we do. \end{quote}