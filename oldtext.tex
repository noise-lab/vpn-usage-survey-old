\subsubsection{Knowledge about tracking and privacy}
\label{sec:methods-tracking}

\underline{Privacy} Interviewees scored a median of 2 on General Caution scale
and a median of 3 on Privacy Concern scale, which show that they were about
below to average in terms of privacy attitudes. Nevertheless, interviewees had
a median score of 4 on Technical Protection Scale. Such rating could indicate
that our interviewees were less concerned  about the their privacy because
they actively acted against its violation. 

On the SeBIS scale, interviewees scored a median of 5 on Device Securement,
3,5 on Password Generation, 3 on Proactive Awareness and a median of 3 on
Updating. On the whole SeBIS scale, interviewees had a median score of 3, so
an average score of Security Behavior Intentions with an above average rating
of Device Securement.  \underline{Tracking} Before the questions about VPNs,
we asked our interviewees about their general knowledge about tracking. 

\textbf{What is Tracking}

Twelve interviewees defined \textit{Online tracking} as creating data about an
individual, and gathering data about individual's location (5). More than half
(17) of interviewees said that it happens through user's search history and
general online activity. Another 2 that it can be done through web camera but
also through apps (1) or looking at key logger passwords (1). Also, for 12
interviewees  \textit{tracking} meant that someone could see what they are
doing online and 6 believed that it happens each time you go online, while 3
stated that you are not able to hide.

\begin{quote}To me that would mean collecting data about a user in any service
or any capacity while they're online or doing anything that involves being
online. That could be something like keeping a lot of the messages you sent,
even if you delete them or that could be something as simple as just tracking
what times you sign on to different things. I think that also applies to
things like, with our [Proxy] cards, every time we buzz into a building that's
documented somewhere and that's documented online, so that kind of goes into
online tracking as well. Anytime we're using an online system, information we
recorded. (P32)\end{quote}



\textbf{Comfort with Tracking} For 4 interviewees tracking was frightening,
and tracking made 2 interviewees' overall online experience uncomfortable.
Moreover, 8 interviewees believed that tracking was something bad that should
not have place. For example, P13 believed that tracking is more visible online
than censorship.
Moerover, P25 did not feel like there was any good law in place that would
make users' data safe, which led them to believe that companies have low
standards of data processing and make users more vulnerable. 

On the other hand 7 interviewees interpreted it as something good, for example
using it as a tool for crime investigation (2) or suggesting ads with clothing
(2). 

\begin{quote} it [online tracking] could mean positive tracking. I guess
Google tracks many things that I do and I'm okay with that and I'm aware so I
don't do too personal searches or anything. It's just for personalized
advertising or search results that could be beneficial. (P31) \end{quote}


\textbf{Who Tracks Data} We asked our interviewees who they think track
online. 18 interviewees believed that big companies, such as Facebook (4) and
Google (3) track people online. On the other hand 15 interviewees pointed
government as an institution that tracks, naming NSA (3), FBI (2), CIA (1) and
IRS (1). Furthermore, 8 interviewees stated that every website tracks its
users, 6 that it is done by advertising agencies or data companies. Four
interviewees said that ISP can track its users, another 4 that it is done by
hackers and 2 that school/university tracks your online activity. Five
interviewees said that tracking is not done by anyone in particular.

Twenty-seven interviewees admitted that they believed that they are being
tracked. Thirteen of which believed they are tracked by big companies, such as
Google (3) and another 13 that even though they are tracked, they are not
targeted. 


\begin{quote}I think the way that I, totally uninformed way you think of it is
that there are probably machines, some AI, some program that looks for certain
keywords and a certain combination or indicators, histories. I'm sure there's
some algorithm that people have developed. I don't think someone is
necessarily actively tracking me but I'm sure there is some software that I,
along with many other people are being tracked by and if something I were to
say was flagged on that, then I think there would be an individual, or at
least more attention to me.(P32)\end{quote}


Five interviewees believed that they are not tracked, as they do not do
anything suspicious.


\textbf{Reasons for Tracking}    

We also asked participants about reasons that these entities may have to
track. Fifteen responded that tracking is needed to gather data about users'
preferences, also 15 said that to gather data about our online behavior and 14
that it is needed for advertising. Nineteen believed that tracking is for
financial interest. For example, P25 mentioned that tracking is an essential
part of A/B testing. 

P23 associated online tracking with location services, that track users'
geographical position. They also added about their feeling being watched:

\begin{quote} Essentially, I feel sometimes like, I don't know if you ever
heard of the novel 1984, but it's just like George Orwell, like beautifully
describes just being watched and how were under a lot of supervision or if you
read this philosopher Foucault, who talks about this Panopticon, that's were
constantly being just supervised by a higher power and this power being like
the people who controlling these, this data, cause essentially it is data,
like where you are, it's stored where your cache. And also like the things you
like and the sites you visit it's all start in your cashe, in your history and
a site has the permission to use that cash in order to market things a certain
way for you, for example if you. This is like super intense, but like, I don't
know, I googled a pair of shoes right and then like two seconds, maybe like
five minutes later I was just scrolling through my Instagram feed and I see
those same shoes being on an ad in Instagram, that's like annoying. I mean I
know you want me to buy your shit because it's capitalism and all, but I don't
need to know that you're, I don't need to be like primed into like I want to
buy that thing. That's essentially what I mean by tracking.\end{quote}

On the other hand, P07 described tracking as repercussion of Internet speed
debate. The internet providers would track users to evaluate the number of
people that visit each website. The bigger company with more viewers, would
get faster Internet to users, so Internet providers would control speed of the
Internet for different companies. 







OLD CONCLUSION
Data privacy is an increasingly important concern for Internet users. However,
most Internet users today do not fully understand the Internet [18, 19], and
the US education system does not provide sufficient background on online
technologies [29]. Virtual Private Networks (VPNs) are important tools that
can help safeguard against data-intrusive practices. Although VPNs are growing
in usage and prevalence [3], we lack an understanding of how users perceive
VPNs and why they use them. Our study focuses on college students and asks the
following research questions:

TODO

We employed a mixed-methods approach involving interviews, a survey, and a lab
study.

We found that students have weak understandings of VPNs, both in terms of what
they are and how they work. They use VPN infrequently and primarily for
content access, regarding privacy and security as secondary concerns. Students
consider cost, ease of use, and speed as the most important elements of a VPN.
Most college students use school-provided VPNs, though a substantial number
use commercial VPNs. Students are generally dissatisfied with the speed,
stability, and interfaces of their VPNs.

Our findings have implications for educators, policymakers, and VPN providers
and designers. Educators and VPN providers should become better resources for
users to learn about online privacy and security. VPN providers should also
educate its users more effectively on VPN technology, and also run pricing
analyses. In tandem with other policies in the online privacy space,
policymakers should explore transparency regulations for VPN providers to
disclose their own privacy violations. VPN designers should improve the user
experience of the installation, login, and reconnecting processes of VPNs;
they should also develop designs that better communicate what VPNs are
actively doing.

Our study is a first foray into the perception and usage of VPNs, and
establishes a broad overview of the space through the lens of college
students. Our work provides insights that can inform practical improvements in
policy and design. However, we also highlighted many areas for further
exploration. Future researchers can take deeper dives into the areas that we
have introduced, or corroborate our design recommendations through
comprehensive usability studies. In the long term, Internet users stand to
benefit from a stronger, better-communicated, and better-received VPN
ecosystem.


Almost all respondents (99\%, 348/350) believed that some data was collected
about them when they used the Internet. The vast majority of these respondents
(N=348) believed that companies (93\%, 323/348), websites (93\%, 322/348).
their government (83\%, 290/348), and Internet Service Provider (81\%,
282/348) were collecting their data; in contrast, only 2\% (8/348) of them
believed that friends and family were collecting data on them. Almost all of
these respondents believed that their data was collected for advertising and
other financial motives (99\%, 343/348); a smaller majority believed that
their data was collected for political motives, such as influencing political
leanings (72\%, 252/348). Almost all of these respondents believed that their
online activities (97\%, 339/348), interests and preferences (96\%, 335/348),
and location (96\%, 333/348) were collected. Smaller majorities believed that
demographic information (85\%, 293/348) and device type (81\%, 283/348) were
collected. Significant minorities of respondents believed that more sensitive
data including private messages (41\%, 141/348), keystrokes (31\%, 109/348),
and recordings (31\%, 109/348) were captured. On a five-point Likert scale,
participants were on average "somewhat concerned" about this data collection
(mean 2.94, median 3).

Survey respondents used a wide variety of tools and tactics outside of VPN.
Among the most popular were ad blockers (80\%, 280/350), using two-factor
authentication (75\%, 263/350), avoiding spam email (70\%, 246/350), and using
private browsing mode (63\%, 222/350). Fewer respondents utilized high-effort
tactics such as changing passwords frequently (15\%, 52/350), using password
managers (17\%, 60/350), or avoiding social media accounts (27\%, 95/350).
More obscure online tools were also less popular among respondents, with only
12\% (43/350) using tracker blockers such as Ghostery and 9\% (33/350) using
Tor. Most respondents (57\%, 198/350) used at least some of these tools most
of the time when they go online, on both laptops (98\%, 343/350) and phones
(77\%, 271/350). However, the overall effort that respondents put into
protecting themselves online was low. Out of a maximum score of 35, the median
"protection effort score" of respondents was only 12 (see section 3.5 for how
this statistic was derived). Participants largely heard about these tools and
tactics online (73\%, 256/350) and from friends and family (71\%, 248/350),
and most (64\%, 225/350) started using these tools and tactics 3 or more years
age



Also, P13 did not think the VPN is secure, even the institutional one:
\begin{quote}Probably nothing. I think it's pretty open so whatever security
wherever you're [inaudible] into. Because I'm going to the university network,
I know there security is there. But depending on where I've been, those
securities have been better or worse. So at Rutgers ... I did my Ph.D. at
Rutgers, and they had a huge hack shut down the whole system. I mean, that was
just a student that did that. I don't think their securities are very good at
all. Here, I haven't heard about any problems, but I don't know. Seems like
the security is a little better here. But it's very easy for me to VPN and
grab my IP off any of the computers in the lab and just get it onto a
computer. I don't think it's that secure. There's no extra steps I have to
take to do that.\end{quote}