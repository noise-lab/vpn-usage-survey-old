% remember to put in tool design
% interview Q's etc - mirror sections from study part 1
In order to better understand how students think about privacy when choosing between VPN providers, we designed a small-scale lab experiment. \AH{I don't think "experiment" is the right word, but I'm using it for now.}
The results of the interviews and the survey described in sections~\ref{} suggest that privacy is not of significant concern to students when choosing a VPN provider to use.
Nonetheless, we were interested in seeing if students would value privacy more if they were given more information about the privacy practices of VPN providers.

\subsection{Recruitment} 
We once again recruited participants through listservs maintained by our institution's survey center.
We filtered for students who had used a VPN before and were currently undergraduate or graduate students at our institution.
All interviews were audio-taped, and participants were compensated with a \$20 Amazon gift card.

In total, 64 people responded to our recruitment e-mails.
Of these respondents, 5 had participated in either the survey or the interview portion of our study, so they were not eligible to participate in our experiment.
One of the respondents was also not a student, so they were not eligible to participate in our experiment.

We were able to conduct our experiment with 14 of the 65 eligible participants. 
Unfortunately, one of the experiments was conducted in a noisy setting, which may have influenced the participant's responses.
We also later discovered that one of the participants we conducted the experiment with had also participated in the survey portion of our study.
This left us with valid data from 12 participants to analyze.

\subsection{Participants} 
Breakdown of participant demographics

\subsection{Pre-experiment Survey} 
Before participating in the experiment, participants were asked to fill out a consent form and a short survey where we collected demographic information.
In the survey, participants confirmed that they have used a VPN before, and they listed the VPNs they have used in the past.
Participants also gave consent to audio recording for the experiment.

We also asked the participants several multiple choice questions to elicit their views on privacy with respect to VPNs.
For example, we asked participants who they think has access to their data while using a VPN.
We also asked participants what kinds of data they think are being shared.
These questions gave us a broad understanding of the views participants had about VPNs before participating in the experiment.
 
\subsection{Focus Group Methods}


\subsubsection{Design}
Tool design goes here

\subsubsection{Content}
Interview questions go here

\subsection{Analysis} 
How we analyzed the data

\subsection{Limitations}
Limitations of tool, study design, demographics, etc
