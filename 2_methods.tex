% remember to put in tool design
% interview Q's etc - mirror sections from study part 1
In order to better understand how students think about privacy when choosing between VPN providers, we designed a small-scale lab experiment. \AH{I don't think "experiment" is the right word, but I'm using it for now.}
The results of the interviews and the survey described in sections~\ref{} suggest that privacy is not of significant concern to students when choosing a VPN provider to use.
Nonetheless, we were interested in seeing if students would value privacy more if they were given more information about the privacy practices of VPN providers.

\subsection{Recruitment} 
We once again recruited participants through listservs maintained by our institution's survey center.
We filtered for students who had used a VPN before and were currently undergraduate or graduate students at our institution.
All interviews were audio-taped, and participants were compensated with a \$20 Amazon gift card.

In total, 64 people responded to our recruitment e-mails.
Of these respondents, 5 had participated in either the survey or the interview portion of our study, so they were not eligible to participate in our experiment.
One of the respondents was also not a student, so they were not eligible to participate in our experiment.

We were able to conduct our experiment with 14 of the 65 eligible participants. 
Unfortunately, one of the experiments was conducted in a noisy setting, which may have influenced the participant's responses.
We also later discovered that one of the participants we conducted the experiment with had also participated in the survey portion of our study.
This left us with valid data from 12 participants to analyze.

\subsection{Pre-experiment Survey} 
Before participating in the experiment, respondents were asked to fill out a consent form and a short survey where we collected demographic information.
In the survey, participants confirmed that they have used a VPN before, and they listed the VPNs they have used in the past.
Participants also gave consent to audio recording for the experiment.

We also asked the participants several multiple choice questions to elicit their views on privacy with respect to VPNs.
For example, we asked participants who they think has access to their data while using a VPN.
We also asked participants what kinds of data they think are being shared.
These questions gave us a broad understanding of the views participants had about VPNs before participating in the experiment.

\subsection{Participants} 
According to the results of our pre-study survey, 9 of our 12 participants were between the ages of 18 and 24 (75\%) and 3 participants were between the ages of 25 and 34 (25\%).
Of these participants, 7 were females (58.3\%), and 5 were males (41.7\%).
Furthermore, 7 of the participants were from the United States (58.3\%), and 5 participants were of other nationalities (41.7\%). 
Lastly, 4 of the participants were Computer Science students (33.3\%), 3 of the participants were Political Science students (25\%), and 5 of the participants were studying other majors (41.7\%).

With respect to VPN usage, 8 participants indicated that they have used institutional VPNs (66.7\%), and 3 participants indicated that they have used VPNs that their employers offered (25\%).
Furthermore, 6 participants indicated that they have used a paid, commercial VPN (41.7\%), and 5 participants indicated that they have used a free, commercial VPN (50\%).
Interestingly, 1 participant indicated that they have used a VPN that they set up themselves (8.3\%).

Lastly, with respect to privacy, 10 participants indicated that they have not looked through a VPN provider's privacy policy (83.3\%), with only 2 participants indicating otherwise (16.7\%).
Every participant believed their VPN provider has access to their location (100\%).
Furthermore, 8 of the participants believed their VPN provider collects data about which websites they visit (66.7\%).
Interestingly, 7 of the participants believe VPNs guarantee access to the Internet (58.3\%), 6 of the participants believe VPNs mask their IP address (50\%), and 4 participants believe VPNs guarantee safety from tracking (33.3\%).
However, only 1 participant believes VPNs guarantee anonymity (8.3\%), and only 1 participant believes VPNs guarantee "privacy" (8.3\%).
 
\subsection{Experiment Design}
\subsubsection{VPN Audit Extension}
\begin{figure}[t]
    \includegraphics[width=0.85\linewidth]{sections/figures/vpn-audit.png}
    \caption{A screenshot of our Chrome extension for showing the browsing history sent to trackers outside of Hotspot Shield.}
    \label{fig:vpn-audit}
\end{figure}

We became interested in studying how university students think about privacy with respect to VPNs  when we learned that Hotspot Shield--a free VPN provider--sends data to select trackers through its Chrome extension~\cite{windscribe-hotspot-shield}.
According to the PAC file the extension, if an object on a webpage that points to the domains "analytics.google.com" (owned by Google), "pixel.quantserve.com" (owned by Quantcast), or "shelljacket.us", then any connections to these domains are created outside of the VPN.
\AH{Should we show the code for the PAC file here as a figure?}
This means Google and Quantcast can see which websites users of Hotspot Shield are visiting, even while the users are behind the VPN.

After we learned of this practice, we decided to develop our Chrome extension to show the practice and design an experiment to see how university students would react.
Figure~\ref{fig:vpn-audit} shows a screenshot of our extension.
The extension works by examining webpages in real time that users visit while using Hotspot Shield and looking for the Google Analytics, Quantserve, and Shelljacket trackers.
It then counts the number of unique webpage that these trackers were present on.
When a user clicks on the name of a tracker, they see a list of each unique webpage under the "Browsing history sent to tracker" heading.

\subsubsection{Interview Questions}
Interview questions go here

\subsection{Analysis}
We first transcribed the recordings from each experiment and developed codings to apply to the transcripts.
Our codings were based on general trends we discovered from reading the transcriptions.
These trends relate to how about how participants choose VPN providers and how participants think about privacy with respect to VPNs.
In total, we created 13 parent codes and 68 child codes.

Once we finished coding the interviews, we reviewed text from the transcripts that corresponded to the most common codings.
We then selected quotes from the text that we believe were most illuminating for the most common codings.
We note that these quotes do not necessarily summarize the views of each participant.

\subsection{Limitations}
There are several limitations of our participant demographics that must be considered.
For example, our sample size was limited to 12 participants, which means that our findings cannot generalize to all university students.
Our participants are also mostly Computer Science students that may be more technically sophisticated than students of other disciplines.
Most of our participants are from the United States, which limits our ability to understand how international students think about privacy with respect to VPNs.
Lastly, all of our participants were students at a particular university.

We are also limited by our experiment design in several ways.
Our participants were only exposed to the privacy practices of Hotspot Shield, so we are unable to generalize how they would react to the privacy practices of other VPN providers.
We are also unable to gauge how participants would react to the privacy practices of VPN providers on their own computers and within their own personal space; at best, we can ask them to imagine.