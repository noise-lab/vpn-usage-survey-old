\section{Discussion}\label{sec:discussion} 

Our study highlights potential areas of improvement for VPN messaging, design,
and policy. We organize our presentation of these areas by the same topics as
the Results section. We further denote the implications our findings have for
stakeholders in the VPN usage process, including educators, VPN providers, VPN
designers, and policymakers. We also present potential future studies and
address limitations.

\subsection{Mental Models, Perceptions, and Expectations of VPNs}
\subsubsection{Educators}

Educators should introduce privacy and security to students earlier as an
important concern. As previously discussed, Internet usage is exploding, and
online technologies are becoming more and more integrated with daily life.
While companies have rapidly developed tools to accommodate and exploit this
phenomenon, the American education system has largely not caught up with
online technologies \cite{levin_29}. A higher rate of Internet literacy in today’s
digital age would be beneficial for a multitude of reasons, including popular
concern for online privacy and security risks. We observed that there was a
general feeling of indifference towards data collection, both in terms of
perception and tool usage. Improved curriculum on Internet technology,
including not only how user data is collected and used online through case
studies but also what people can do to protect against it, could spur the next
generation of Internet users to be more privacy and security conscious.

We observed that students who use other privacy and security tools earlier are
more likely to use VPN earlier. However, we also found that students who were
introduced to VPN at a younger age were more likely to use potentially
dangerous free VPNs, likely because they view high cost as a larger barrier
than privacy and security concerns \cite{taylor_7}. Students who are more concerned about
data collection are also more likely to habitually use VPN for privacy and
security reasons. Improved education on the risks of online data collection at
a younger age could make privacy and security higher priorities in students’
browsing habits, including their usage of protective online tools.

\subsection{VPN Usage} \subsubsection{VPN Providers}

VPN providers should educate their users more effectively on what VPNs are and
what they do. We observed that students who learned about how their VPNs
worked through online research or directly from their providers tend to be
more diligent about VPN usage and more conscious about VPN technology than
those who learned from other sources. As students cannot be forced to do their
own research, this knowledge gap should be addressed by VPN providers who can
do more to inform users of VPN functionality and purposes. Anecdotally,
searching for definitions of VPNs online yields many different results that
could prove to be confusing even for tech-savvy users; some explanations are
very technical, and others are simplified to the point of being misleading.
Having authoritative and clear answers from their VPN providers could help
users gain better knowledge and understanding of VPNs. Students largely see
VPNs as censorship circumventors and location spoofers, and less so as a tool
for privacy and security. However, privacy and security are important factors
that students consider when choosing a VPN; furthermore, students who use VPNs
for privacy and security reasons are more likely to keep using VPN. Thus,
commercial VPN providers are incentivized to educate users on these factors
and highlight them as features, as this could improve user acquisition and
retention. This would have the added benefit of increasing users’ perception
of VPN transparency.

\subsubsection{Policymakers}

As the threat of large-scale data privacy violations is now omnipresent,
technology companies’ privacy policies have come under scrutiny by governing
bodies. With existing US laws on individual privacy rendered largely outdated
by the information age, policymakers are incentivized to create updated
regulations that will protect consumer privacy starting with consumer data
\cite{runnegar_31}. In 2018, the EU Parliament passed the “General Data Protection
Regulation,” which bolstered individual privacy rights by introducing new
rules and regulations on companies requiring them to request consent,
anonymize collected data, and notify consumers about data breaches \cite{de_groot_30}. The
US government has yet to come up with similar legislation on the federal level
\cite{runnegar_31}.

While policy to help combat the misuse of data by corporations would be
welcomed, policymakers could additionally make efforts to encourage individual
usage of privacy- and security-enhancing online tools. One way to achieve this
is to build Internet literacy into secondary school curriculum, as discussed
in section 5.1.1. However, we have highlighted that students who adopt VPNs
earlier tend to use free VPNs, which can be dangerous, invasive of privacy,
and malware-infested \cite{taylor_7}. Requiring VPNs and other online tool providers to
disclose their data collection and sharing practices would deter educated VPN
seekers from choosing dangerous options. In tandem with improving educational
standards, policymakers have an opportunity to increase the prevalence of
safer VPN options for individual consumers.

\subsection{VPN Issues and Improvements} \subsubsection{VPN Providers}

We observed that cost is a key reason why many students choose to stop using
paid VPNs. Paid VPN providers should fix the messaging of their products, as
discussed in section 5.2.1, to properly inform users of the benefits of their
products, then do pricing analyses to optimize profit.

\subsubsection{VPN Designers}

Many students indicated that accessibility, ease of setup, and ease of use
were important considerations when choosing VPNs. However, we also observed
that many students had issues with the interfaces of their VPNs, showing that
the current offerings are not satisfying user expectations in this area. While
other popular issues such as speed, stability, and cost can be difficult to
resolve quickly at scale, designers can improve user satisfaction and
retention by improving VPN user interfaces. Key complaints that were repeated
throughout respondents’ complaints involved installation and setup processes,
login processes, and behavior when auto-reconnecting. Furthermore, a number of
respondents do not understand what their VPNs are doing. In addition, students
who verify that their VPN is working generally do so through passive methods
such as viewing alerts from the VPN or viewing the VPN’s tray or taskbar icon.
These situations could be improved with design elements that better
communicate the actions of the VPN, which could additionally improve user
perception of VPN transparency.

