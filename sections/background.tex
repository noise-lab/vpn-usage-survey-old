\section{Background}\label{sec:background}
VPNs, or Virtual Private Networks, are one of the many tools that Internet
users can utilize to protect against online privacy and security risks. VPNs
work by creating a secure, private connection ("tunnel") through the
provider's server through which the user can safely access a destination
server [25]. VPN providers can encrypt and authenticate this connection using
a number of methods with varying degrees of effectiveness, including OpenVPN,
Layer 2 Tunneling Protocol, Internet Protocol Security, and several others
[25]. To outside snoopers such as Internet Service Providers, the VPN user's
traffic appears to be coming from the VPN server as opposed to from the user's
computer. As such, the user's IP address is masked, and the VPN server's IP
address is used instead. As IP addresses are geolocated, the user's traffic
also appears to be from the VPN server's location.

Depending on the provider and the user's goals, VPNs can be used to access
destinations on the Internet or on private networks. Users in areas with
censored Internet access can utilize VPNs to access blocked content, i.e.
using Twitter in China. More generally, users can use VPN to access
location-restricted content, such as watching Hulu, a US-only service, outside
the US. Commercial VPN providers often offer multiple servers located in areas
with open Internet access, such as the US or Hong Kong. Other VPN users may
use VPN to access content on a private network. For example, Princeton
University's VPN allows off-campus users to access the university's library
system. Of course, assuming a strong encryption protocol and a trustworthy
provider, VPNs also provide better privacy and security in normal, day-to-day
Internet usage. Users may want to protect against Internet Service Provider
snooping when browsing at home, or against hackers when using unsecured public
WiFi hotspots. Although solely relying VPN is not enough to protect an
Internet user from the risks of online data collection, it can be very
valuable when utilized in tandem with other privacy-conscious tools and
tactics such as tracker blockers.

VPNs were first introduced in 1996 with the release of the now-deprecated Peer
to Peer Tunneling Protocol [26]. Originally created for enterprises to
communicate securely, VPNs rapidly gained broad commercial appeal as personal
Internet usage soared [26]. VPNs today are effective tools in protecting
privacy and security, as well as circumventing censorship. In 2016, the
worldwide market for VPNs was 15.64 billion USD, and is forecasted to reach
double that by 2021 [3]. Yet, VPN usage is far from mainstream -- only 16\% of
Internet users in North America reported using a VPN at some point [27].
