\subsection{Findings}\label{Findings}

We organize the presentation of our findings by general topic, including TODO.

All survey questions were required for respondents, but individual questions were shown to respondents only when applicable. As such, questions that have fewer than 350 data points contain responses from every applicable respondent; a lack of response does not indicate a respondent's choice to abstain. In presenting our results, we discuss percentages in terms of the number of people who were shown the question. However, the figures presented in this section will display percentages in terms of the total number of respondents (350), and include "N/A" values for those who were not shown the question.

Because our analysis involved the presence of both unobserved variables and the unpredictability of human behavior, we expected our correlation analysis to produce low R values. We will only be including correlations that are notable and statistically significant (p < .05). Unless otherwise noted, R values range from .1 to .3.


The main themes that emerged from are as follows: 

\subsubsection{Knowledge about VPNs / Mental model}

\textbf{What is VPN and how it works}

When asked what is VPN in their opinion, 14 interviewees defined VPN as an Internet activity routed through third party machines and 14 as changing IP address. 
\begin{quote}It's sort of a middle man. So instead of you actually downloading the file from someplace where somebody might be looking at you downloading it, they download it for you and then they send it to your computer. So it figures that they downloaded it and not you.( P18)\end{quote}


But for example, P20 was not sure how IP address was changed:
\begin{quote}I think that depending on where you geographically use the internet, you have an IP address, and a VPN will be able to know the IP address of a different place and make it look like you're using the internet from that place instead.(\dots) As for how their making it look like you're using that IP address rather than your actual IP address, I actually don't know.\end{quote}


According to 7 interviewees you are getting into another network while using VPN and 6 interviewees describe this network as secure and 4 as private. 
%(Only) 
Three described VPN as an extra level of safety. Furthermore, 13 interviewees defined VPN usage as accessing blocked things and 5 that using VPN is bypassing rules and laws. 10 interviewees believed that VPN is masking their identity and another 10 believed that it is reducing others ability to track you.  6 of our interviewees described VPN as a high-tech thing and 4 that it is not really known. 12 admitted that they did not have an extensive knowledge about VPNs.

At least 4 participants defined VPN as a blackbox and described people who are interested in VPN as technology-savvy. As P27 shared their mental model of people who use VPN: 
\begin{quote}When I think of VPN I feel like you, I mean I do computer coding but I feel like you have to be a super nerd to kind of (\dots) understand what all that is and to be able to manipulate it in any way\end{quote}


As well as P18:
\begin{quote}The usage of this VPN sort of hinges on two things, the desire to obtain copyrighted material for free and also the knowledge of the existence of VPN's. Those are two I think pretty big bottlenecks that sort of limit this sort of information to sort of tech nerds, if that makes sense.\end{quote}


And P25 descibed VPN as a blackbox but also as a statement:
\begin{quote}Everything in computer science is a black box. I don't know if the VPN is working, I don't know is somewhere along the line \dots (\dots)
I think that it is a button that is important for very serious reasons. It's part of citizenship in a sense, trying to do something that is right. Its usefulness is pragmatism, it's like, "I need to see this YouTube video, but they don't let me see it in Brazil so I'm just going to do it in Belgium." I think that that's what VPNs are to me. What I was saying about black box is I don't actually know if VPNs work, at the end of the day there could be some step in their server chain, their routers that happens to be in a jurisdiction, happens to be something not kosher in a sense and then the whole thing collapses. 'Cause it's just sort of an endgame of all the links in the chain. I don't know if there is some other link in the chain that is already going wrong and compromising the whole thing, and using VPNs is useless.


I think that there is a very fair chance that I'm achieving nothing as far as security goes with VPNs, but it's also a statement in a sense and I guess that that is important too.\end{quote}

We also asked our interview participants whether they know if and where their real IP is visible. Half (16) of the interviewees stated that they did not know, while 14 believed that their VPN provider would be able to see it. Also, 6 respondents had a sense that such information are anonymous and 2 respondents said that their Internet Service Provider (ISP) is still able to see it. Moreover, 2 respondents did not find it important.

We asked our survey participants to report their knowledge of Virtual Private Networks on a 4-point scale. 76\% (266/350) reported having some knowledge of VPNs, and 19\% (68/350) reported having no knowledge. In contrast, only 4\% (16/350) considered themselves to have high knowledge or to be experts on VPNs. 51\% (179/350) of respondents indicated that they knew how to see their VPN’s server location; 54\% (190/350) reported that they knew how to view their IP address under the VPN.

Notably, having a higher level of self-reported VPN knowledge was correlated with a number of different surveyed variables. Students who were Computer Science or Engineering majors tended to have more knowledge about VPNs. Students with more VPN knowledge tended to use other online privacy and security tools more often, and tended to use paid VPN options and feel safer when doing so. These students were also more likely to hear about VPNs online, and were more likely to do their own research to learn how they work (R=.35). In terms of VPN usage, students with more VPN knowledge tended to first use VPNs 3 or more years ago, use VPNs more often, use VPNs for privacy and security reasons, and still use VPNs at the time of the survey. In addition, students with higher VPN knowledge tended to put more effort into verifying that their VPN was working, and were more likely to not think that their VPN provider was collecting their data.

We also tested if students who were Computer Science or Engineering majors had particular knowledge patterns for VPNs, as they may have learned about VPNs through their studies or have other affinities for technologies. We found that Computer Science and Engineering majors tended to have higher VPN knowledge and put more effort into verifying that their VPN was working. In addition, Computer Science and Engineering majors were more likely to first hear about VPN from online research. Correlations were generally small, however (R<.2).

Survey participants were asked to report, in short-answer form, what they thought a VPN was. We categorized these text responses by topic based on their contents; some responses contained multiple topics. 42\% (147/350) of responses mentioned that VPNs are tools that mask the user’s location, while 4\% (14/350) of respondents indicated that they did not know what VPNs were.

Most respondents indicated that they thought their VPNs guaranteed content access (75\%, 263/350) and masking of their IP addresses (53\%, 186/350). Significant minorities believed that their VPNs guaranteed privacy (36\%, 125/350), anonymity (30\%, 104/250), and safety from tracking (28\%, 99/350).

Our survey participants had weak knowledge of VPNs overall, with very few respondents indicating that they had high or expert-level knowledge of VPNs. However, the short-answer responses for what a VPN is showed a greater disconnect between students’ perceived and actual knowledge of VPNs. Though a majority of respondents indicated that they had "some knowledge" of VPNs, a vast majority of the responses about what a VPN was only referred to VPN features and purposes. We can observe that the respondents equate what a VPN does and what it is commonly used for to what a VPN is. Though only a small minority of respondents indicated that they do not know how VPNs work at all, it appears that a majority of respondents do not really know what a VPN is or how it works.


\subsubsection{VPN Expectations of Privacy}
At least 20 interviewees did not believe that VPN guaranteed them anonymity, although 8 interviewees admitted that VPN offers something, that in their opinion, is the closest to anonymity. On the other hand, 7 interviewees thought that VPN guaranteed them privacy and 6 access to sites they wanted to visit. Six interview respondents believed that nothing is guaranteed.

As P04 explained:
\begin{quote}I think it never really guarantees me it. Because even though \dots if I use even a private VPN that I paid for, even though the majority of the world does not see my IP address and everything, I feel like the owners of the VPN provider will be able to still see it. And then it's just a matter of having that security breached. So I don't think there's ever really a sense of true privacy. Unless I make my own VPN.\end{quote}


Also, P13 did not think the VPN is secure, even the institutional one:
\begin{quote}Probably nothing. I think it's pretty open so whatever security wherever you're [inaudible] into. Because I'm going to the university network, I know there security is there. But depending on where I've been, those securities have been better or worse. So at Rutgers ... I did my Ph.D. at Rutgers, and they had a huge hack shut down the whole system. I mean, that was just a student that did that. I don't think their securities are very good at all. Here, I haven't heard about any problems, but I don't know. Seems like the security is a little better here. But it's very easy for me to VPN and grab my IP off any of the computers in the lab and just get it onto a computer. I don't think it's that secure. There's no extra steps I have to take to do that.\end{quote}


At least 24 interviewees responded that in their opinion you can be tracked while using VPN as well, as they believed that there is always a way to do so (8) but also because you can be tracked by VPN provider itself (9). 
%Check in dedoose if P01 is in this 9:
As P01 explained:

\begin{quote}So if it is SSL encryption, the VPN provider would still know that you are communicating with a certain web service but the VPN provider would not or probably not know the contents of the communication if it's SSL encrypted. They would only know who you want to communicate with. And if it's not encrypted, then they can see, they can be do packet sniffing or even more malicious things like deep packet injection and deep packet inspection to actually look at the contents of that communication and do potential malicious things with that. \end{quote}

Also, 4 interviewees admitted that while using VPN, one could be tracked by the government. For example P30 used VPN only in different countries to access blocked content. They stopped in USA as they did not need VPN's access properties anymore and did not see any privacy protecting reasons:

\begin{quote}I don't think there is any marginal benefit to using a VPN to evade tracking. Seems like VPN's are all like, at least partially, controlled or transparent to the government.\end{quote}

 Moreover, two interviewees responded that one can still be tracked by advertising agencies. Although, 3 interviewees admitted that VPN makes tracking at least harder than normal. 

P21 explained that using VPN is not enough. In order to make tracking harder for companies, they changed locations that they connected within VPN:
\begin{quote}Yes [I can be tracked while using VPN], especially if I'm using the same IP address. That creates a problem because my internet footprint ... or like, Chrome, for example, my web browser could definitely still track me and connect that, see where I've been connecting from. Or Gmail could see that. Gmail always tells you, "Oh, you've connected from this weird device, or from this location that we don't recognize." So I think you can definitely still be tracked, so it's important not to be complacent and stick with the same IP address and obscure your web footprint in other ways in order to make it ... I think there's some stuff that's out there. There's some amount that I've already been tracked, and I can't take that back. But I can change my footprint, and try to make my devices that I'm using and the way that I connect to the internet as vanilla or ... I don't even know the right way to say it, but basically in a way that doesn't jump out and can't be profiled.\end{quote}


\subsubsection{Configure VPN - Expectations}
At least 25 interviewees admitted that they had not prepared themselves and their devices before installing their VPN.

\subsubsection{VPN practices: Keeping logs, Sharing information, Transparency - Expectations} 
\textbf{How VPNs keep data}
We asked our interviewees whether they believe that VPN providers keep their logs and other information stored. 23 interviewees answered “yes”, giving examples such as keeping information for user statistics (5) or to sell data (5). Some also referred to university’s VPN (7): 4 interviewees thought that university wants to have an access to all information and 3 believed that VPN helps university to monitor if someone’s cheating during exams etc. There were 9 interviewees who did not think that VPN providers could keep their logs, from which at least 4 were not sure but hoped they did not and 3 stated that this is not what their service is for.

\textbf{Who VPNs share data with}
We also asked our interview participants about their opinion on other VPN practices. When asked whether they thought their VPN providers could be sharing their information, 17 responded “no” and 11 “yes” but 12 were uncertain about their response or did not know the answer.  Moreover, from these who said that VPN providers do not share information with other entities, 8 confessed that they hope their information are not being shared, 5 admitted that while their VPN providers do not share any information, others do. Moreover, 2 of these interviewees believed that even though their VPN providers do not share data with others on regular basis, they would with legal authorities. 

\textbf{What VPNs tell users}
Furthermore we asked our interviewees whether in their opinion their VPN providers were transparent. 14 of our interviewees replied that they did not feel like their VPN providers were transparent, while 12 believed that they were. 

At least 15 interview participants answered that they did receive description on how to use VPN from their providers and 14 admitted that they did not receive any description.  

Moreover, 13 believed that it is important to get such description for multiple reasons; interviewees believed that such description is necessary in order to know how to make the best use of VPN (4), as well as to know what VPN provides (4). They believed that it would make usage easier (2), they would like to know what should and should not be done while using VPN (2) and how VPN works with their devices (2) and how to connect to it properly (2). On the other hand, at least 9 interviewees believed that using VPN is very easy and at least 8 did not consider important getting such description. 

We found that 74\% (260/350) of survey respondents believed that their VPN collects data. The remainder of this subsection will discuss the responses of these respondents (N=260). The majority of them believed that VPNs collect location (87\%, 226/260) and online activity (74\%, 192/260) data. Fewer respondents believed that VPNs collected more sensitive, “nefarious” data, such as private messages (17\%, 44/260), recordings (15\%, 39/260) or keystrokes (15\%, 38/260). The majority of these respondents believed that VPNs collected this data for commercial motives (69\%, 178/260), or simply because data collection is a “default consequence of using the Internet” (69\%, 178/260). Around half of these respondents (49\%, 127/260) selected both of those options.

There was little consensus on who had access to the data collected by VPNs. The largest proportion of these respondents believed that companies (43\%, 113/260) and the government (34\%, 88/260) had access to the data. A significant number believed that only the VPN had access (18\%, 47/260). A substantial number of these respondents indicated that they did not know where their data went (19\%, 50/260). Of the 213 respondents who believed that their VPN shared their data, they generally thought that their online activities (70\%, 150/213), location (74\%, 157/213), interests (58\%, 123/213) and demographic information (57\%, 122/213) were shared. Figures 9 and 10 contain a full breakdown of response data.

Our results suggest that students possess feelings of indifference and/or defeatism towards data collection that extends to their perception of VPNs. We observed a connection between “inevitable” data collection and “commercially motivated” data collection in respondents’ mental models. However, respondents generally did not believe that their VPNs were collecting especially sensitive data such as keystrokes or recordings.



\subsubsection{How people choose VPN} 
\label{sec:findings-choosing}
\textbf{Guidelines when choosing VPN}
We were also interested in factors that our participants take into account when choosing their VPN provider.

The most important considerations for survey respondents when choosing VPNs were cost (76\%, 266/350), ease of use (66\%, 266/350), and speed (60\%, 210/350); many respondents also regarded security (58\%, 202/350) and privacy (48\%, 169/350) as important. Branding (20\%, 69/350) was somewhat less important.

Students who regarded security as an important factor tended to avoid using free VPNs. These respondents also tended to care about privacy when choosing VPNs, and less about VPN brand or accessibility (R=.42). Respondents who care about privacy when choosing VPNs tended to care less about accessibility factors, and also used VPN more often and in more places.

College students who regard security or privacy as important factors when choosing a VPN tend to care about the other as well, and are more diligent about using their VPN. In addition, these students tend to care about tangible, usage-related features of their VPN over “softer” factors like branding and accessibility.


\textbf{Reputation} For the majority of our interviewees (19) the most important was good reputation of the VPN provider and 3 interviewees added that the fact that their friends had used it before had an impact on their decision. 

\textbf{Security less important} Moreover, 10 interviewees had different security and privacy requirements: 5 made sure that VPN had secure network, 3 that VPN provider did not store any of users‘ records, 2 that VPN provider did not sell users‘ information and also 2 that VPN protect users‘ data. For one interviewee it was important that VPN did not require any personal information when setting up the account and another one that there was an option of secure payment. 

\textbf{Ease of use} Another factors considered by our interviewees were also ease of use (8), speed (7) and ease of set up (5). Moreover, 5 participants would look at the price before purchasing subscription and for 5 it was important that VPN was for free.   

For example for P11, the main factors were word of mouth, experts’ opinion, cost as well as customer service available:
\begin{quote}I tend to look at reviews. I like to think that I'd know whether I could decide a VPN was good enough or not on my own, but I tend to trust other people, so I look on, like, Tech Radar and PC Monitor, those kinds of websites, to ascertain whether experts thinks it's the best. I get some reviews from friends, as well, see if their having a good experience with VPNs, and then I'll go on website if I'll, like \dots I think I narrowed it down to a couple of options. So, when I came to China I was deciding between Express and Astro, and I just looked on their websites, went through, like, server locations, cost, and their privacy policies, as well, and then into deciding on Express.(\dots)
and then, I also had Express available customer service, which was very important as well.\end{quote}

\textbf{Cheap, Fast, Secure?} We asked our participants to rate three components: secure, fast and cheap, from the most important to least important when choosing VPN. \textit{Security} scored 77 and was the most important for our participants. Second was \textit{Fast} scoring 54 and the least important was \textit{Cheap}, with 41 points.


\textbf{Locations} We asked our interviewees whether the number of servers locations that VPN provides and to which they can connect to, is important. For 12 interviewees it was important, for example 3 reported that there are different data regulations in different countries and 2 stated that decentralization was a good thing. Another 2 interviewees simply believed that more locations give them confidence that they would stay connected: if one connection would not work, there would be another one. Two other interviewees believed that bigger number of servers’ locations is useful when travelling.  On the other hand, 11 interviewees did not think that number of servers’ location is important, as long as there would be couple of them (5).  

\textbf{Source of knowledge}
Half of our interviewees (16) found out about Virtual Private Network and their VPN providers through word of mouth. At least 9 interviewees learned about VPNs through their school or university and 7 through research. 4 interviewees admitted that they simply searched VPN providers through Google search engine and 2 of them used first thing in search results. Moreover, 3 interviewees were told about VPNs in stores while purchasing phone, 2 interview participants got to know about VPNs because they were travelling to China and another 2 learned about them at their jobs.


P26 learned about different VPNs through advertisements on websites such as The Pirate Bay:
\begin{quote}So typically when I'm accessing those websites, like open-source libraries and especially The Pirate Bay, they always have different ads for different VPNs. So that's where I typically get a lot of them. I'll download them, try them out. Typically, I'll stick to the free ones. So I'll test it out and see like ... the one I use now, Windscribe, it has a 10GB limit, and so that's more than enough for what I need. I've had others that an ad comes up every five minutes that you're on there. That bothered me. There was one, it would slow it down substantially, so I didn't use that either. I think the one I found right now, Windscribe, is a pretty good one.
(\dots)I'm not going to see those ads on Google or YouTube. It's on those open-source libraries that you can, that they'll advertise them because you should probably use one of those before you're downloading anything.\end{quote}

The majority of respondents indicated first hearing about VPNs through friends and family (61\%, 21/350) and/or online (53\%, 185/350). Smaller but significant minorities reported hearing about VPNs through their institutions, such as their school (38\%, 132/350) or employer (15\%, 52/350). The majority of respondents started using VPN between 1 and 5 years ago (62\%, 218/350) and in college or high school (80\%, 280/350).

Students who first heard about VPNs online or through personal connections were more likely to use commercial VPNs; in contrast, students who heard about VPNs through their institutions were more likely to use institutional VPNs (R=.48). Students who heard about VPNs online generally did not feel safe using institutional VPNs, and also tended to use VPNs for privacy and security reasons; in contrast, students who heard about VPNs from their institutions felt less safe when using free VPNs. Students who heard about VPNs online or through personal connections tended to learn how VPNs worked from the same sources (R=.38, R=.48). Students who heard about VPNs through personal connections were more likely to consider VPN brand as an important factor in choosing a VPN, and were generally less concerned about protecting security.

It appears that students who first hear about VPNs through online research tended to be more concerned about privacy and security, and less trustful of institutional VPNs. 

We also asked survey participants to report where they learned how VPNs work. The majority of respondents (55\%, 193/350) indicated that they learned how VPNs work from online research, and 36\% (125/350) of respondents indicated that they learned from friends and family. A smaller percentage of respondents reported learning this information from their VPN providers (21\%, 72/350), and very few reported learning from an expert’s testimony (5\%, 16/350). A substantial number of participants (21\%, 72/350) did not know how VPNs work at all.
Students who learned how VPNs work from their provider tended to be more concerned about data collection in general, and tended to use their VPNs more. Students who learned how VPNs work through online research tended to protect themselves with more other online tools (R=.33). Students who indicated that they VPNs to protect their privacy were more likely to have learned how VPNs work through their providers or through online research. Similarly, students who used VPN to protect their security were more likely to have learned how VPNs work through their providers or through online research. Students who learned how VPNs work online or through their providers also tended to put more effort into verifying that their VPN was working.

Interestingly, students who learned how VPNs work through online research or through their VPN provider tended to be more diligent and conscious about their VPN usage. Students were also more likely to learn how VPNs work through the same channels where they first heard about VPNs.

\textbf{Trust in VPN}
When asked interviewees how they determined whether their VPN provider was trustworthy, 13 of out interviewees said that it had good reviews online. Another 10 would verify that through word of mouth and 7 knew it was trustworthy because of the founders of their VPN, including university (4). 

\textbf{Price}
22 interviewees reported that they would use free VPNs but with many comments on it and restrictions,such as making sure that it was safe, while 9 more strictly said that they would not use it. Among these from the former group, 6 admitted that they did not need VPN, nor used it often, which is why they did not mind using free VPN. 

As P24 shared, they were not a regular user so there was no need to pay. Nevertheless, paid VPN could mean more secure VPN:
\begin{quote} I think the one that you have to pay for is more trustworthy. But, it could easily be the other way around. Just because you have to pay for something doesn't mean that it is more reliable, or even more efficient. But, I do think that the paid ones generally people might think that they are more safe to use. And that their information may be more secured, just because of that added price tag on it.\end{quote}

P03 was also suspicious of free VPNs:
\begin{quote}I would definitely try a free VPN, but at the same time, if others cost money, and this one is free, I'm like, so why aren't you? Why are you free? That'd definitely make me a little, if most of the VPNs cost money and that's one free, then I'll be suspicious. But then if it's like half of them cost money and half of them are free. I'll like, oh, maybe the ones that cost money are for people who have really, really, really sensitive information. And the ones that are free are maybe for people who are just casual internet users who want that extra level of security, but don't have a justification to need to pay money. But I'd definitely be wiling to purchase a free one if I just saw what other VPN, like I compare the reasons as to why some cost money and some didn't cost money. (...) I guess maybe the ones that cost money offer a lot more protections. I don't know if some VPNs are more secure. Maybe the ones that cost money are very, very secure, whereas the ones that are free are maybe just basic level of security, just a little bit better than what you get on the internet. Just things like that.\end{quote}

As well as P14, who was worried that installing free VPN could be unsecure:
\begin{quote}I was worried that I possibly also downloaded some virus along with the VPN, so, that's also what motivated me to just purchase ExpresVPN and just deal with the cost because I wasn't really sure if what I had and so it was from a trustworthy place whether that's a reasonable or not.\end{quote}

5 interviewees said that they would use it only if they were sure that particular free VPN is secure. For example P12, would check reviews first:
\begin{quote}Why I want free VPN? 'Cause it's not very important to me, and I don't have a lot of money, and I don't spend money on things that I don't really need. Why I think I can trust it? I mean I would read up on people's reviews of the privacy status of different VPNs to choose the one that I install. I would only install it if I get a pretty good impression from the reviews that I read.\end{quote}

\textbf{Different VPN providers} When asked whether it was important who their VPN provider was, 14 interviewees reported that they would not pay attention to this information and they simply did not care about this, while 11 interviewees reported that it was important, especially for these who used university’s VPN (7), which was reassuring for them that this VPN was safe. On the other hand, for 2 interviewees it was not important that university was their provider, as one explained, university is only a client of a different VPN provider, not a VPN provider itself. 


\subsubsection{Goals}



\textbf{Perfect VPN}
\textbf{Performance} We asked our interviewees what, in their opinion, "perfect" VPN would provide. 9 interviewees listed that they would like their VPN to operate at the same speed and latency as  “normal” Internet. For at least 8 interviewees reliability was a key component of a perfect VPN and for 7 it was security. 

\textbf{Security} Also 6 interviewees mentioned that perfect VPN would not access any personal information, 6 that it would not hand over or sell data to other companies. 

\textbf{Data practices} At least 5 interviewees believed that VPN should be transparent and also 5 that it should serve only users' interest. But as P25 observed, this could not make usage better, since users do not get information about their data being stored anyway.


\textbf{Access} Moreover, 8 interviewees believed that number of locations would be important, 7 added that perfect VPN would keep offering location spoofing and 4 that ideal VPN would provide an access to absolutely everything.

P18 described their perfect VPN:
\begin{quote}I would look to some sort of a VPN that operates in another country, free from the jurisdiction of whatever country I'm residing in, because I know that in some countries it's harder for the United States to work with a VPN or whatever companies if the VPN is hosted in a country that's not really friendly towards the United States, that would be interested in having people from the government of the US working in their country. I think that would be a big plus.


I think another big plus would be that it's not relatively well known, ironically. That goes against sort of what I said before, that it has good reviews. At the same time, I don't want many other people knowing about the existence of that VPN because I don't want it coming up on the radar of say the US government scouring these VPN's looking to get information on people. It's also not profitable. It's not profitable just to go for people who are downloading some movies or something like that, to go halfway across the world just to get a few IP's that are going to lead nowhere, relatively speaking. It's not profitable.\end{quote}


\textbf{Possible improvements}
When asked what kind of improvements our interviewees would like to see, half of them reported more transparency from the VPN provider side (16) and 14 proposed more education effort for users to know how VPN works (11), reasons why they should use VPN (3) and about online security overall (2), since there are many people without sufficient technological background to know it (7). 

\begin{quote}So it would be nice if the VPN providers themselves had more of an analysis of what kinds of tracking is generally being done to you if you are say in the US or if you're say in Europe.It would be nice if the VPNs themselves made a better case for using them. Like if you just go to Private Internet Access's website, it's kind of preaching to the choir, you know? It says, "Browse anonymously. Keep your IP address cloaked. Defend yourself from data monitoring," whatever. But it'd be nice to have like very concrete examples of, you know, "In the US, you could get in trouble for doing this, that, or the other on the Internet, and it becomes impossible to get in trouble for that if you use this."

I guess they can't exactly endorse illegal activity like that, but they could use other examples, like say Amazon is tracking your buying history. Well, I guess you still have to be logged into Amazon, but even if you just search on Amazon, they can track your searches or whatever. So you know, things like that. (P02)\end{quote}


Also, 5 interviewees wanted VPN providers to communicate what they do with their logs. Furthermore, 8 interviewees would like VPN to be more accessible and 5 more user-friendly. 

P13 encountered problems while trying to set up VPN with their friend:
\begin{quote}There were instructions online on how to download it. But they don't have a very good system. I was actually trying to help my friend over the phone the other day with her Apple computer, and we couldn't get it to work. (\dots)They were missing some key steps. (\dots)That were really important, like where to find certain things if you don't use your computer very often and because I'm not an Apple user, I had trouble being like well, I know in Windows, you go here and here. I don't know how to find that in your computer. So we had to Google the in between steps. (\dots)I think a lot of tech documents assume that most people are expert users of their computer. (\dots) It should have text. It should have pictures so people know what it looks like when they get there and less computer jargon the better.\end{quote}

Moreover, P13 explained how he would like to change security and access permission around institutional VPN:
\begin{quote}Okay. I just would have more security around it. It just doesn't feel secure to me. It doesn't ... it's too easy to access, too easy for me to get into pretty much anything I wanted here, which is good for my work for bad if I wasn't doing work. Let's say I had a vindictive student who then could go in and delete all our files or put some kind of virus in the computer in the lab. There's no protections against that. They have full access just as much as we do. (…) It's backed up theoretically, but there's still limitation to that backup. The backup is every 12 hours it backs up. Let's imagine you just recorded 12 hours of data and then you piss off a student and then he goes in and deletes it all. There's nothing stopping them from doing that. 
(\dots)You can put a no delete on your files for certain people, permissions for certain people. But then every time somebody joins the lab and has access to that system, you would have to do that and it just requires a lot more work. Some people have their files completely blocked that you can't access them, but that makes it hard to collaborate and with collaboration, there has to be some sense of trust.\end{quote}



\subsubsection{Usage}
\textbf{First experience}
Our interviewees were introduced to VPN in many different ways. At least 8 started to use VPN at institutions like university, 4 were looking for ways to bypass geographic firewalls, so they could watch something, 3 found out about this through their parents and 2 were told to use it when they were going to China. Three interviewees reported that they could not remember their first encounter with VPNs.  
At least 9 interviewees described their first experience with VPN as not good, for various reasons. Some could not connect (2), for some it was too slow (2) and some found it confusing (2).  On the other hand, at least 2 interviewees found it simple to use for the first time.

%How do you use it
%7 interviewees reported that they used it only when they needed. Should we talk about it? It’s out of 14. And vs 4 always on.

\subsubsection{Using VPN Types}

We found that most respondents (65\%, 228/350) used VPNs through their school, which was not surprising due to the accessibility of Princeton’s free VPN service for students. A significant portion of respondents also used commercial VPNs, including both free (49\%, 172/350) and paid (28\%, 97/350) options. Few respondents set up their own personal VPNs (7\%, 24/350).

Paid commercial VPNs that a substantial number of respondents reported using include ExpressVPN (13\%, 44/350), NordVPN (6\%, 20/350), and Private Internet Access (6\%, 22/350). Free commercial VPNs that a substantial number of respondents reported using include Hotspot Shield (13\%, 45/350), TunnelBear (11\%, 37/350), Hola (9\%, 31/350), and Betternet (8\%, 29/350). For both options, several respondents indicated that they used SonicWall and/or Connect Tunnel which is the software used by the Princeton VPN, indicating that some students are confused about the distinction between institutional and commercial VPNs, as well as VPN providers and VPN software.

We asked respondents to indicate how safe they felt when using free, paid, and institutional VPNs on a 5-point Likert scale. We mapped these choices to integer values 0 through 4, and found that respondents felt safer when using institutional (2.56/4.00) or paid (2.45/4.00) VPNs than when using free commercial (1.84/4.00) VPNs. Students who started using VPN in high school or earlier are more likely to use free VPNs (R=.46). Students who use paid VPNs tend to have higher VPN knowledge and concern about data collection, and put more effort into protecting themselves online. Paid VPN users tend to use their VPNs more often (R=.36) and more carefully, and they also tend to use VPNs for privacy and security reasons. Institutional and free commercial VPN users are more likely to use their VPNs for content access reasons. Free VPN users are more likely to think that their VPN providers collect their data, while paid VPN users generally do not think so. Institutional VPN users tend to be indifferent to their providers collecting data. Free VPN users were more likely to have issues with ease of use, features, and content access.

Paid VPN users are generally more privacy-conscious, caring more about online protection and putting in more effort to secure themselves. They use their VPNs more frequently and carefully, and feel safe using their paid VPN over other alternatives. In contrast, free commercial and institutional VPN users seem to use their VPNs primarily for content access.


\subsubsection{Using institutional VPN} 
Out of these who used institutional VPN, 5 reported that they would use it only for work, because they simply did not feel private (3), they felt that university
could track them (2) and institutional network was vulnerable  (2). P13 gave an example about the student who shut down exams by hacking into university's VPN through another VPN. %ADD HERE THE QUOTE ABOUT STUDENTS HAVING ACCESS TO POSTDOC’S WORK.

On the other hand, 5 interviewees used institutional VPN for private activities, like browsing, as well. For example, one of them believed that it added security, and one would simply not switch it off. 

\subsubsection{Privacy policy, have you read?}
We also wanted to know whether our interviewees read their VPN providers’ privacy policies. 24 interviewees replied that they did not read it and 11 of them added that they never read any privacy policies. Also, for 11 interviewees it was too long and not readable. Moreover, 11 of the interviewees who did not read it, believed that it was not important to read it, for various reasons; 5 believed that it did not matter whether they would read it or not, because either way you have to trust your VPN provider (3) or because they would collect their data no matter what they would write in their privacy policy (2). 2 interviewees believed that they did not have any valuable information to be worried about and one believed that reading it would only make them more confused. Among all interviewees, 6 believed that providing and reading privacy policy was important. On the other hand 5 interviewees read their VPN provider privacy policy but 3 of them only looked through it, one did not find anything specific and one found it standard. 

For example, P11 read it after some of his friends had bad experience with a different VPN:
\begin{quote}when I was renewing my subscription, and I had, like, a look on the internet to see what the privacy policy is, and the terms and conditions and things like that, just to make sure that I was definitely getting the best value \dots It was mainly the best value for money, and just to make sure that everything was as above-board as it could be, in terms of security. (\dots)I had a good look, because I'd heard from a couple of friends who were using \dots I think it was Onavo, and that they were kind of worried that it was owned by Facebook, and it was kind of influencing what they were getting adverts for, I think. (\dots) So, I didn't want to have that lingering over me when I was using a VPN. (\dots) It [ExpressVPN] seemed like it was the best kind of \dots It was the best kind of policies, and I did some, like, online research, people who know a bit more about this thing than me.\end{quote}

\textbf{Reasons for VPN usage}
The majority of our interviewees (21) reported that they use VPN to bypass geographic firewalls, 15 of which to watch movies or TV shows online.  

\textbf{High school} For example P20 used it to get access to sites that were blocked by his high school: 
%(also interesting word of mouth example and not allowing apps to see %location thinking that they can do it anyway): 
\begin{quote}So I don't use many VPN's anymore. I used them when I was in India, and I used them \dots Okay, yeah. I've used them for a few reasons, but privacy was never really one of them. It was just when my content was restricted (\dots) when I was in boarding school, I went to boarding school for high school. Our WiFi was very tightly patrolled. So any number of things were blocked, like from adult content, to a lot of sports websites for instance were blocked, because they "encouraged gambling," and I like to watch a lot of sports online illegally, because that was the only way I could watch them. So to get around that especially, but also just to not have the school looking at everything you were doing. (\dots) 

And then I came in my sophomore year and heard about the VPN. I mean, that was great because it allowed us to use the internet after 12:00 to message our girlfriends or whatever. Just stuff we couldn't do before. So that was big. But then I think someone must have tipped off the IT department again with the incident I discussed with a student illegally downloading movies. I think that they sort of realized how many people were circumventing the internet, and because my school had liability for any number of things, they didn't want someone's parents complaining, and then they have to deal with that.


So then after my sophomore year, that summer we came back, and all the Chrome add on VPN's that we had been using, basically just weren't effective anymore. I really have no idea how they did that.\end{quote}


\textbf{Geographic blocking} On the other hand P24 used VPN not only to watch TV show but also for accessing their app account that s/he subscribed to while being in America and could not use it at home country:
\begin{quote}I was in New York this past summer, and so, I had a subscription for a couple months. But then I came back to Vancouver, and wanted to end the subscription. But, I couldn't, I found out that I couldn't access the app or login to my account through the phone app, because I wasn't in the United States, basically. So, I just thought like, hey maybe if I use my phone to change my IP address, so it looks like I was accessing the app from the United States. And I downloaded, I think it's called Express VPN. And that was able to help me work around the location, geographical issue, and access the account so I could cancel the subscription. So, that was a pretty useful tool for me.\end{quote}


P11 used VPN for news websites that were not blocked but had different or limited content depending on IP address of the Internet user. As they were from UK, they wanted to get into UK BBC website, while being located in the US.

Also P25 shared that s/he used VPN in order to obtain different goods and for piracy and for geo blcoking reasons:
\begin{quote}I don't use it for privacy much anymore, I use it as a way to separate my own identities mostly, I use it when I need to be in another country theoretically. I think that that's the most common way to use it for me, if I go and I try to see a video and it's not available in my country I just scroll down from Brazil to Belgium and usually it works. I think that that's what I mostly use VPNs for. Or if I'm doing something that I feel I wouldn't want TNSA to see, TNSA is an abstract thing in my head. If I didn't want the Panopticon to see it, I would turn on the VPN but I don't think that ever happens.\end{quote}

Moreover, 7 interviewees would use it to access materials that normally they would have to pay for, 5 admitted they used VPN for piracy and 3 for downloading content.

P26 also used it for their relative to be able to watch a tv show:
\begin{quote}It masks my IP address from my internet service provider, and through that I can access content that maybe they wouldn't want me to access that's blocked from their networks, or maybe access things that they wouldn't be happy with me accessing. Like with the university, there are certain free textbook websites that they don't like students going on, but then I use my VPN to be able to access those sites. For example, Venezuela has blocked everything coming from their YouTube channels, and I have my mom reroute the US IP address to a Mexican IP address with a VPN, so then she could watch her Venezuelan TV shows. (\dots) It's things like that that I mainly use it for.
(\dots)Or like watching Netflix from another country. Things like that. (\dots)Sometimes there are shows that are only accessible from different countries. Like, for example, if you were to watch the \dots if you were to access Netflix from, let's say, Brazil, you'd have a lot more Portuguese language shows than you would on the American version of Netflix. So for things like that it's very helpful. And also sometimes the movies that are released on American Netflix aren't the same movies that are released on, let's say, Spanish Netflix, and so you can watch those. Even American-made shows are sometimes licensed differently across different countries. That's when it's really good to use an international VPN like that.\end{quote}

\textbf{VPN is not used for Privacy} 13 interviewees said that privacy was not their reason to use VPN. Fewer (7) participants used it to protect their personal information, as for example P21:
\begin{quote}I guess I don't like the idea of Princeton or an ISP being able to see all of my traffic. I don't think that I trust anyone with all of my traffic or consumer habits. I don't want that stuff to be sold off. I don't want them to be able to build a profile of me because it can be quite revealing, especially because you're device\end{quote}


Also, 5 interviewees would use it while on public Wi-Fi and 4 while travelling. 5 reported that they used it for security and 3 for anonymity.  Moreover 3 interviewees used it because they liked the idea that there was a "free" space on the Internet and P25 believed that using VPN is a statement that security is important:

\begin{quote}it's why Private Network Access I think? (...) It's why it got so popular. They tried to subpoena the guys to release information about some of the people who used the VPN, and then they actually didn't have it on their servers. So people knew that they didn't keep records, so everybody started using that one. Then using VPN is something that gives me information about how the game is played, like whether the people who run VPNs are vulnerable to these legal mechanisms that might be used by grouped institutions or legitimate institutions. It just helps make it clear in my head how the game works.\end{quote}


19 interviewees would use it for work, out of which 16 used it in their university or school. P21 used it, so their boss could not be able to see if they were online at work.


\textbf{Why others use VPN}We also asked what other people's reasons to use VPN are. 5 believed that others use VPN for privacy. 11 interviewees replied that others use it for watching something online and 3 that for piracy.

As P18 described what VPN is:
\begin{quote}
Just a way for people to get around government agencies looking at their IP address so that they can download whatever they want from wherever they want within reason because these VPN's are also subject to their own vulnerabilities. These IP addresses, some VPN's have policies where they will provide information from IP addresses if they're requisitioned from a government entity. That's pretty much it with regards to VPN's, just used to download information. Tore isn't really used for that because it's slower, because of the technical architecture of Tore. VPN's are just used mostly for that.\end{quote}

and why people would use it:
\begin{quote}Oh, so VPN's with regards to downloading things illegally. People use VPN's to maybe get copyrighted material but don't want their IP address to be tracked by law enforcement agencies or recorded by their internet service provider. Just a plain old dummy would probably go somewhere, to a file hosting site like the pirate bay or mega upload or something like that, that have lists of torrents. (\dots) I also know that VPN's are used in engaging illicit commerce on websites that \dots A famous example is the silk road which was taken down back in 2013. \end{quote}

\textbf{Usage in different countries}
%Think about home country/other countries 
P15, one of our interviewees from Romania, believed that people from US are more afraid of using VPN than in their home country:
\begin{quote}I feel like people are more likely to buy their stuff rather than try to get it for free, while at home no one cares, so they're just gonna go for the free thing because it's easier and you're never gonna see someone getting accused for, of a, and getting a criminal record for downloading a game. It's only if you did some sort of, I don't know, WikiLeaks type of thing. But, yeah.\end{quote}

\textbf{VPN vs. No VPN}
10 interviewees reported that VPN did not change their Internet usage and online experience, 8 did not recognize many differences between using and not using VPN and 7 did not see any difference while browsing with or without VPN. On the other hand, 10 interviewees saw one important difference: VPN allowed them to access content they could not access before. Moreover, 8 believed that VPN makes their online experience more secure but 2 believed that VPN is actually more vulnerable and 3 admitted that they left their guard down when using VPN. Also, 2 interviewees found that Internet connection is slower while using VPN.

Survey respondents reported that content-related reasons were most important in driving them to use VPN. 66\% (230/350) of respondents reported access to institutional materials, such as those from the university library, as an important factor; 48\% (168/350) reported circumvention of Internet censorship as a reason. Privacy (36\%, 127/350) and security (30\%, 105/350) were also significant factors, with 27\% (93/350) of respondents selecting both and 40\% (139/350) selecting at least one of the two. Of these 139 respondents, the majority were concerned about protecting themselves from companies (73\%), websites (70\%), the government (66\%), and hackers (65\%).

Students majoring in Engineering were more likely to use VPN to protect privacy and security. Students who regard privacy and security as important motivations for using VPNs tended to be more concerned about online data collection, particularly by institutions and commercial entities. These students tended to think that their actions, including their online activities and messages, were being collected, and took additional efforts to secure their online privacy and security with other tools (R=.34, R=.33). Privacy and security motivations for using VPNs were likely to be linked (R=.71). Students using institutional or free VPNs were more likely to use VPNs for content access, whereas users of paid VPNs tended to use them for privacy and security reasons. Students using VPNs for content access were less likely to list privacy and security as motivations for using VPNs. Students who used VPNs for privacy and security reasons also tended to use them more frequently,and were more likely to have learned how VPNs work from their provider or online research. 

We observe that students who use VPN for privacy or security reasons tend to use it for the other as well. They are generally more conscious of their online activities and concerned about data collection, particularly by institutions such as the government. These students generally have higher VPN knowledge, trust paid VPNs more, and were more diligent about VPN usage; they also tended to learn how their VPN works from their providers or through online research.

\subsubsection{Consistency of VPN Usage}

The vast majority of respondents used VPN only rarely (46\%, 161/350) or sometimes (41\%, 142/350). Few participants reported using VPN most of the time or always (13\%, 47/350). Figure 11 displays a further breakdown of locations where students reported using VPNs. Students who were more concerned about data collection, more knowledgeable about VPNs, and put more effort into protecting themselves online tended to use their VPN more frequently. Students who were paid VPN users also tended to use their VPN more often (R=.36).

We also surveyed respondents on what methods they used to verify that their VPN was working. “Passive” methods of verification were most frequently chosen, including viewing alerts from the VPN (41\%, 144/350), checking that normally restricted content was accessible (48\%, 127/350), and viewing the VPN’s tray or taskbar icon (45\%, 159/350).

Of the surveyed respondents, only 42\% (148/350) reported that they were still actively using VPN. These users tend to be more concerned about data collection and put more effort into protecting themselves online. They are more likely to have higher VPN knowledge, and to have started using VPN 3 or more years ago. They tend to use their VPN to protect privacy and security, and are less likely to use free VPNs.

The respondents who stopped using VPNs (58\%, 202/350) reported that they did so because they were no longer location restricted (46\%, 92/202), did not have anything to hide (37\%, 74/202), or simply did not use it enough (34\%, 68/202). Very few respondents reported a lack of security (1\%, 3/202) to be a contributing factor in their decision to stop using VPNs.

Respondents were generally not diligent or consistent about VPN usage and verification; most prior users had even abandoned VPNs entirely. We can speculate that students who have maintained their usage of VPNs still use them for privacy and security reasons, as opposed to purely a desire to access content. Students who use paid VPNs also tend to use them more frequently and more diligently. For students who stopped using VPNs, indifference to privacy and security was a theme as they primarily used it as a tool to circumvent censorship and other location-based restrictions. This trumped other factors, such as cost or speed, which we initially speculated to be top-of-mind for students.

\subsubsection{Problems, Liked Qualities, and Disliked Qualities}
We asked our interview participants what kind of problems they encountered while using VPN. 
\textbf{Accessing content} 7 interviewees said that VPN would disconnect, 6 that it was slow, another 6 that they could not access the content they wanted because some websites knew they were using VPN. 

%Check if P08 in dedoose in this 6, ale Usage by others

For example P08 was annoyed by the fact that they need to switch off the VPN while using Netflix, otherwise platform would not allow to access the content

\begin{quote}So, the only use cases I've seen mentioned in Germany was to, for example, when Netflix wasn't available in Germany, people would try to use-
(\dots)It is [available] now, but in the beginning, things like Netflix were only available in the US, so people would try to use it and then they needed a DNS unblocker or a VPN to use Netflix. I don't think that works anymore. One of the reasons that I'm not using VPNs as much anymore is that a lot of these services just block the IP ranges of these data centers, so whenever I want to watch something on Netflix, I have to turn off the VPN, which is annoying. (\dots) I don't like it. I understand why they are doing, basically being pressured by \dots I think this whole idea that they buy content for specific regions and then content that's region-blocked is stupid, but that's the world we live in and this how their contracts and everything are set up, so they have to do something to prevent people from regions to access that content. I mean, for me, it's just annoying, because then I have to switch off the VPN. \end{quote}


4 interviewees mentioned that installing VPN was problematic, because it was a complicated process (1). 16 reported that they had no problems at all. 

\textbf{Checking if VPN is working}
We also wanted to know how our interviewees check whether their VPN is actually working while they are using it. 10 interviewees replied that they get information on their screen that VPN is connected, 9 that since they use VPN for websites that otherwise are blocked, having an access means that they are connected. 6 interviewees actually checked their IP address to make sure VPN is working, 3 looked at VPN icon and 3 said that they never check. 

Survey participants were asked to report what issues they experienced with their VPNs. 41\% (144/350) reported issues with stability, and 35\% (121/350) reported issues with content access. Figure 14 shows a detailed breakdown of the issues students reported facing. Notably, 8 of the 10 “Other” responses mentioned speed as an issue, a figure that would likely be much higher if speed were presented as an option.

The types of issues students faced was not correlated with their level of knowledge with VPN, or with their effort in verifying that their VPN was working. Notably, students who experienced issues with ease of use including stability, complicated installation processes, and difficulty in understanding were more likely to have stopped using VPN.

We asked respondents to report, in short-answer form, what they liked and disliked about their VPNs. We categorized these text responses by topic based on their contents; some responses contained multiple topics. The ability to access restricted content was by far the most commonly “liked” quality of students’ VPNs (63\%, 221/350). Other qualities, including security, privacy, interface, and transparency received far fewer mentions. Respondents indicated that they prefer VPNs based on cost, speed, ease of use, privacy, and security, yet they do not perceive the VPNs that they use as having those positive qualities.

There was little consensus on what respondents did not like about their VPNs. After speed and stability, respondents most frequently disliked the interfaces (20\%, 69/350) of their VPNs. Although respondents indicated that privacy and security are important factors for VPN users, they are not mentioned substantially as either liked or disliked qualities of respondents’ VPNs. This possibly suggests that students do not care as much about privacy and security in their VPNs as they indicated earlier. Only 9\% of respondents reported having nothing they disliked about their VPNs.





