\section{Introduction}

As Internet technology becomes integral to our daily lives, data privacy risks will continue to be highlighted. Discourse about these risks has increased in recent years; data breaches occur frequently, and media coverage of data privacy scandals has become mainstream [4, 21, 22, 23, 24]. VPNs, or Virtual Private Networks, are one of the many tools that Internet users can utilize to protect against online privacy and security risks. VPNs work by creating a secure, private connection ("tunnel") through the provider's server through which the user can safely access a destination server [25]. Though VPNs are increasing in popularity, their usage is far from mainstream; only 16\% of Internet users in North America reported using VPN at some point [27]. In addition, most Internet users lack a thorough understanding of online data collection and the risks that it poses to privacy and security [9]. In the US, education curriculum has not evolved to accommodate rapidly changing technology [29]. Given this, we expect a high degree of disconnect between user perceptions of VPNs and the services actually provided by VPNs. It is unknown how users actually perceive VPNs, and what their explicit purposes for using them are. In this paper, we focus on college students and study the following research questions:

TODO

To answer our research questions, we employed a mixed-methods approach. TODO

We found that students generally had weak VPN knowledge, and did not fully understand how VPNs worked. They sporadically used VPN primarily for content access and regarded privacy and security reasons as secondary, and they considered cost, ease of use, and speed as the most important elements of a VPN. Most college students use the VPN provided by their school, though a substantial number use commercial VPNs. Students are dissatisfied with the speed, stability, and interfaces of their VPNs.

Our results have implications for educators, policymakers, and VPN providers and designers. Both educators and VPN providers should emphasize the importance of online privacy and security. VPN providers should further educate its users on VPN technology, and also run pricing analyses. Policymakers should require VPN providers to be transparent about their own privacy violations. VPN designers should improve the installation, login, and reconnecting processes of VPNs as well as create designs that better communicate what VPNs are actively doing.

\subsection{Background on VPNs}
VPNs, or Virtual Private Networks, are one of the many tools that Internet users can utilize to protect against online privacy and security risks. VPNs work by creating a secure, private connection ("tunnel") through the provider's server through which the user can safely access a destination server [25]. VPN providers can encrypt and authenticate this connection using a number of methods with varying degrees of effectiveness, including OpenVPN, Layer 2 Tunneling Protocol, Internet Protocol Security, and several others [25]. To outside snoopers such as Internet Service Providers, the VPN user's traffic appears to be coming from the VPN server as opposed to from the user's computer. As such, the user's IP address is masked, and the VPN server's IP address is used instead. As IP addresses are geolocated, the user's traffic also appears to be from the VPN server's location.

Depending on the provider and the user's goals, VPNs can be used to access destinations on the Internet or on private networks. Users in areas with censored Internet access can utilize VPNs to access blocked content, i.e. using Twitter in China. More generally, users can use VPN to access location-restricted content, such as watching Hulu, a US-only service, outside the US. Commercial VPN providers often offer multiple servers located in areas with open Internet access, such as the US or Hong Kong. Other VPN users may use VPN to access content on a private network. For example, Princeton University's VPN allows off-campus users to access the university's library system. Of course, assuming a strong encryption protocol and a trustworthy provider, VPNs also provide better privacy and security in normal, day-to-day Internet usage. Users may want to protect against Internet Service Provider snooping when browsing at home, or against hackers when using unsecured public WiFi hotspots. Although solely relying VPN is not enough to protect an Internet user from the risks of online data collection, it can be very valuable when utilized in tandem with other privacy-conscious tools and tactics such as tracker blockers.

VPNs were first introduced in 1996 with the release of the now-deprecated Peer to Peer Tunneling Protocol [26]. Originally created for enterprises to communicate securely, VPNs rapidly gained broad commercial appeal as personal Internet usage soared [26]. VPNs today are effective tools in protecting privacy and security, as well as circumventing censorship. In 2016, the worldwide market for VPNs was 15.64 billion USD, and is forecasted to reach double that by 2021 [3]. Yet, VPN usage is far from mainstream -- only 16\% of Internet users in North America reported using a VPN at some point [27].
