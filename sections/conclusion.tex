\section{Conclusion}\label{Conclusion}

Data privacy is an increasingly important concern for Internet users. However, most Internet users today do not fully understand the Internet [18, 19], and the US education system does not provide sufficient background on online technologies [29]. Virtual Private Networks (VPNs) are important tools that can help safeguard against data-intrusive practices. Although VPNs are growing in usage and prevalence [3], we lack an understanding of how users perceive VPNs and why they use them. Our study focuses on college students and asks the following research questions:

TODO

We employed a mixed-methods approach involving interviews, a survey, and a lab study.

We found that students have weak understandings of VPNs, both in terms of what they are and how they work. They use VPN infrequently and primarily for content access, regarding privacy and security as secondary concerns. Students consider cost, ease of use, and speed as the most important elements of a VPN. Most college students use school-provided VPNs, though a substantial number use commercial VPNs. Students are generally dissatisfied with the speed, stability, and interfaces of their VPNs.

Our findings have implications for educators, policymakers, and VPN providers and designers. Educators and VPN providers should become better resources for users to learn about online privacy and security. VPN providers should also educate its users more effectively on VPN technology, and also run pricing analyses. In tandem with other policies in the online privacy space, policymakers should explore transparency regulations for VPN providers to disclose their own privacy violations. VPN designers should improve the user experience of the installation, login, and reconnecting processes of VPNs; they should also develop designs that better communicate what VPNs are actively doing.

Our study is a first foray into the perception and usage of VPNs, and establishes a broad overview of the space through the lens of college students. Our work provides insights that can inform practical improvements in policy and design. However, we also highlighted many areas for further exploration. Future researchers can take deeper dives into the areas that we have introduced, or corroborate our design recommendations through comprehensive usability studies. In the long term, Internet users stand to benefit from a stronger, better-communicated, and better-received VPN ecosystem.
