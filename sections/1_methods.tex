\subsection{Interview Methods}\label{Method}
In order to understand better how users interact with Virtual Private Networks and what are the factors that have an impact on which VPN provider they choose, we conducted qualitative interviews. Both, survey and lab study were based on findings from the interviews.
\subsubsection{Recruitment}
We recruited participants through emailing listservs of Princeton Survey Research Center, Twitter and we also posted an advertisement on Princeton Human-Computer Interaction Lab website. The recruitment message requested participants who had used Virtual Private Network before and were students enrolled to US University program. Our aim was to recruit 20 International students and 10 students from United States. We concluded that such diverse group would expand our knowledge and understanding on how and why participants use VPN. We offered 20 dollars amazon card as an incentive for participation in the interview.
\subsubsection{Participants}
In total, we conducted 32 interviews: 20 International and 12 from US. Four interviewees did not give consent to recording so detailed notes where taken during these interviews. Interviewees could choose between meeting on Princeton University campus and remotely, through Skype. We conducted 23 interviews via Skype and 9 were conducted on-site. 

\subsubsection{Demographics} Put the table

\subsubsection{Knowledge about tracking and privacy}
\label{sec:methods-tracking}

\underline{Privacy}
Interviewees scored a median of 2 on General Caution scale and a median of 3 on Privacy Concern scale, which show that they were about below to average in terms of privacy attitudes. Nevertheless, interviewees had a median score of 4 on Technical Protection Scale. Such rating could indicate that our interviewees were less concerned  about the their privacy because they actively acted against its violation. 

On the SeBIS scale, interviewees scored a median of 5 on Device Securement, 3,5 on Password Generation, 3 on Proactive Awareness and a median of 3 on Updating. On the whole SeBIS scale, interviewees had a median score of 3, so an average score of Security Behavior Intentions with an above average rating of Device Securement.
\underline{Tracking}
Before the questions about VPNs, we asked our interviewees about their general knowledge about tracking. 
%\subsubsection{Censorship}
%At least 18 interviewees defined censorship on the Internet as removal of content on online platforms by higher authorities, such as big companies (6 interviewees) and government (3 interviewees). P10 made distinction between two different types of censorship, governmental one and in private sector:
%\begin{quote}You have censorship as a form of official government policy, and you see that in countries like China with The Great China Firewall and then I think you have, oddly enough, a private censorship. When you have sites that have become defacto public spaces, as Reddit or Facebook, although they are ostensibly private, they're owned by a private company, hence they do have the ability to censor however they'd like, it's sort of turned into a public space in the sense that the greater public meet the [inaudible] political opinion. However, those private spaces are often censored. (…) I think the official government censorship is terrible. (…)Well, I believe very strongly in the rights that, at least as Americans, we're afforded under the First Amendment, being able to freely say and think what we'd like, and freely associate with who we'd like to. Obviously, unless that speech crosses into violence, or rather encouraging violence, or I guess even stronger than encouraging. I can't think of the right word right now, but really compelling violence. I think unless it crosses over into that, you should be able to say and associate with whomever you'd like. Then as far as the private censorship, I mean, I think there's a case to be made for these sorts of defacto community sites historically treated a bit more how traditional media is treated whereby they are compelled to give airtime to both sides of an issue.\end{quote}


 %At least 17 defined it as limiting access to what people see or search and 8 noted that censorship depends on your geolocation. Moreover, 17 interviewees believed that censorship is worrisome, 4 of which thought that censorship is bad for democracy. At least 18 interviewees considered “free speech” as important. Three interviewees declared that in their opinion there should be no restrictions on the Internet but another 23 argued that there should be some regulations applied, which would filter harmful or dangerous content. 
 %\begin{quote}Sometimes I think it's necessary, sometimes there are things that are on the internet that shouldn't be on the internet for certain audiences, for example; children. I honestly think there should be a lot of censorship for children because there are some topics that just simply children are not ready for or shouldn't even be introduced to when they're young it's a matter of, in a way it's keeping the innocence because somethings on the internet are just very violent and yeah, again maybe censoring things for specific ages. If they're under the age of 18 because they don't have consent or anything, is necessary. I also think censorship is necessary in there's like eight, like for example I read an article recently where Facebook deleted 600 or X amount of accounts because of these Facebook groups were actually banding together and just like really, really hate crimes. They were planned hate crimes and they were advocating for more hate in other countries. (\dots) I'm all for free of speech, but when it comes to like imposing life threatening things then censorship might be necessary to kinda bridge the gap between like peoples safety and like freedom of speech. (P23)\end{quote}
%Moreover, seven of our participants believed that censorship should be left to the individual itself. 


%P10 defined censorship as an unequal treatment of different entities:

 %\begin{quote}As I understand it with American news media, if they give a certain amount of time to, say, the Republican candidate, then they need to give an equal amount of time to the Democratic candidate. I think there's something to be said for that, especially if you're trying to position yourself as the front page to the Internet as, say, Reddit does, for example. Yeah, or if you're trying to position yourself, I think, as a fair, impartial kind of platform like Twitter, you can't really go about censoring one side of an argument just because it's uncomfortable or just because you don't agree with it.\end{quote}


%P15 raised a subject of money related to censorship:  
%\begin{quote}I get it. I understand why the rules are strict and I understand that usually paying for the services that you're using, but I also think it's really hard to make people not choose the easy way and take, get stuff for free, instead of paying for it, especially because some services are pretty expensive. Netflix is \$13 a month, which is not something cheap if you think about how much your paying for the whole year and compared to how much your watching it. Same with Spotify and other services that everyone uses, but \dots yeah. It's definitely better to pay for music or films than to get a fine of \$2,000 for doing that and even get a criminal record and the actual sense for criminal record for something like that.As I said, I understand where this is coming from, but I think it's really hard to \dots keep track of how many people are actually paying, how many people are finding ways to get away with it.\end{quote}


%At least 18 interviewees believed that censorship has political background while 11 that its reasons are more societal, for example 4 interviewees from the latter group pointed purpose of censorship to be forming’s someone’s opinion.
%P27 talked about the bias that censorship produced between different groups of people:

% \begin{quote}And then the other instances of censorship that I've seen are like all of the black economic pages that I follow. A lot of times their posts get removed and I think one of my favorite social media personalities, she had a magazine shoot and she was naked on the cover. But she's a black woman and she's also doesn't fit body standards of beauty and so she's on the bigger side. She's heavy set. And they took down her photo but Kim Kardashian, her photos never get taken down but she's always half naked on the internet and stuff like the Tomi Lahren fan page.Tomi Lahren is a conservative spokesperson and her posts never get censored when they're hateful and things like that and are very political. So I think a lot of the censorship is biased.(\dots)I don't think it's fair. I think it's biased. That's exactly what I think. I think if I [inaudible] posts remove and other people not having their posts removed, it gets unfair and I also think it's representative. It's representative of the values of the United States.Even when you type in, if you type in black teen on google, I'm not sure if it's like this anymore, but the fact that you can change it kind of worries me because when you do a Google image search, you expect it to be random. Type in black teen, there's pictures of delinquents and mug shots and drugs and money and stuff like that. When you type in white teen, you get all these happy white teens in school and things.So I think that censorship kind of confirms but also perpetuates some of the racial stereotypes we have about people.\end{quote}

\textbf{What is Tracking}

Twelve interviewees defined \textit{Online tracking} as creating data about an individual, and gathering data about individual's location (5). More than half (17) of interviewees said that it happens through user's search history and general online activity. Another 2 that it can be done through web camera but also through apps (1) or looking at key logger passwords (1). Also, for 12 interviewees  \textit{tracking} meant that someone could see what they are doing online and 6 believed that it happens each time you go online, while 3 stated that you are not able to hide.

\begin{quote}To me that would mean collecting data about a user in any service or any capacity while they're online or doing anything that involves being online. That could be something like keeping a lot of the messages you sent, even if you delete them or that could be something as simple as just tracking what times you sign on to different things. I think that also applies to things like, with our [Proxy] cards, every time we buzz into a building that's documented somewhere and that's documented online, so that kind of goes into online tracking as well. Anytime we're using an online system, information we recorded. (\dots) I think the way that I, totally uninformed way you think of it is that there are probably machines, some AI, some program that looks for certain keywords and a certain combination or indicators, histories. I'm sure there's some algorithm that people have developed. I don't think someone is necessarily actively tracking me but I'm sure there is some software that I, along with many other people are being tracked by and if something I were to say was flagged on that, then I think there would be an individual, or at least more attention to me. (P32)\end{quote}





\textbf{Comfort with Tracking}
For 4 interviewees tracking was frightening, and tracking made 2 interviewees' overall online experience uncomfortable. Moreover, 8 interviewees believed that tracking was something bad that should not have place. 
%P09 – write about tampons story 
Moerover, P25 did not feel like there was any good law in place that would make users' data safe, which led them to believe that companies have low standards of data processing and make users more vulnerable. 

On the other hand 7 interviewees interpreted it as something good, for example using it as a tool for crime investigation (2) or suggesting ads with clothing (2). 

\begin{quote} it [online tracking] could mean positive tracking. I guess Google tracks many things that I do and I'm okay with that and I'm aware so I don't do too personal searches or anything. It's just for personalized advertising or search results that could be beneficial. (P31) \end{quote}

And for example, P13 believed that tracking is more visible than censorship:

\begin{quote}I guess I don't see the censorship even though I know it's there. Actually, I see it more when I travel to other countries because I go to access the same webpages, and I can't get to them. But I think on the other side of things, I see tracking more than I see censorship. So I see that everything on the internet is very personalized suddenly to me, and that's only within the past maybe like six years that it got really that bad.


(\dots) when I first started using the internet a long time ago, you just put in some search and you got some random stuff. That was not related to you. Sometimes it was way off your search criteria. You could bounce around the internet. Nothing seemed to be related to you. Now you go online, and I see things that even [inaudible] follows me for a week. So if I looked at a specific mug online, and I'm seeing that mug on every single webpage I click on, even if those webpages have completely nothing to do with each other. (\dots) I think it's super weird. (\dots)I do not like it. (\dots) I think the ad tracking has gotten way out of proportion. It's nice to be able to search for things, and they remember my old searches. So it's more related to me. That's nice. But then if it's something that I'm looking for that really has nothing to do with my personal interests, I end up with a lot of search criteria that are just unrelated to what I'm looking for.\end{quote} 



\textbf{Who Tracks Data}
We asked our interviewees who they think track online. 18 interviewees believed that big companies, such as Facebook (4) and Google (3) track people online. On the other hand 15 interviewees pointed government as an institution that tracks, naming NSA (3), FBI (2), CIA (1) and IRS (1). Furthermore, 8 interviewees stated that every website tracks its users, 6 that it is done by advertising agencies or data companies. 4 interviewees said that ISP can track its users, another 4 that it is done by hackers and 2 that school/university tracks your online activity. 5 interviewees said that tracking is not done by anyone in particular.

27 interviewees admitted that they believed that they are being tracked. 13 of which believed they are tracked by big companies, such as Google (3) and another 13 that even though they are tracked, they are not targeted. 


\begin{quote}I think the way that I, totally uninformed way you think of it is that there are probably machines, some AI, some program that looks for certain keywords and a certain combination or indicators, histories. I'm sure there's some algorithm that people have developed. I don't think someone is necessarily actively tracking me but I'm sure there is some software that I, along with many other people are being tracked by and if something I were to say was flagged on that, then I think there would be an individual, or at least more attention to me.(P32)\end{quote}


5 interviewees believed that they are not tracked, as they do not do anything suspicious.
%Not sure yet how to describe this quote

\textbf{Reasons for Tracking}    

We also asked participants about reasons that these entities may have to track. 15 responded that tracking is needed to gather data about users' preferences, also 15 said that to gather data about our online behavior and 14 that it is needed for advertising. 19 believed that tracking is for financial interest. For example, P25 mentioned that tracking is an essential part of A/B testing. 

P23 associated online tracking with location services, that track users' geographical position. They also added:

\begin{quote} 
Essentially, I feel sometimes like, I don't know if you ever heard of the novel 1984, but it's just like George Orwell, like beautifully describes just being watched and how were under a lot of supervision or if you read this philosopher Foucault, who talks about this Panopticon, that's were constantly being just supervised by a higher power and this power being like the people who controlling these, this data, cause essentially it is data, like where you are, it's stored where your cache. And also like the things you like and the sites you visit it's all start in your cashe, in your history and a site has the permission to use that cash in order to market things a certain way for you, for example if you. This is like super intense, but like, I don't know, I googled a pair of shoes right and then like two seconds, maybe like five minutes later I was just scrolling through my Instagram feed and I see those same shoes being on an ad in Instagram, that's like annoying. I mean I know you want me to buy your shit because it's capitalism and all, but I don't need to know that you're, I don't need to be like primed into like I want to buy that thing. That's essentially what I mean by tracking.\end{quote}

On the other hand, P07 described tracking as repercussion of Internet speed debate. The internet providers would track users to evaluate the number of people that visit each website. The bigger company with more viewers, would get faster internet to people, so Internet providers would control speed of the Internet for different companies. 



\subsubsection{Interviews}
Our interviews were semi-structured, so we asked all participants the same set of questions and asked additional follow-up questions whenever it was needed. We developed a question set based on our initial research questions, focusing on reasons why student use Virtual Private Networks and how they choose them. In order to get a better understanding of participants’ knowledge and background, we asked also about their general privacy and security awareness before getting to questions about VPN usage. 
Before conducting the interviews, participants were asked to fill out the consent form and consent to audio recording along with a short survey, where we collected data about their demographic information and their general online habits and behavior. 

\subsubsection{Analysis}
We first transcribed all recorded interviews and developed extensive codebook in order to apply it on transcriptions. The same was done to notes collected from interviews without audio recording. The codebook was first based on set of interview questions but it was also refined throughout the process of code application. We had x parent codes in total, for example xxx and xxx; and x child codes, like yyy and yyy

\subsection{Survey Methods}

\subsection{Limitations}