\subsection{Interview Methods}\label{Method}
In order to understand better how users interact with Virtual Private Networks and what are the factors that have an impact on which VPN provider they choose, we conducted qualitative interviews. Both, survey and lab study were based on findings from the interviews.
\subsubsection{Recruitment}
We recruited participants through emailing listservs of Princeton Survey Research Center, Twitter and we also posted an advertisement on Princeton Human-Computer Interaction Lab website. The recruitment message requested participants who had used Virtual Private Network before and were students enrolled to US University program. Our aim was to recruit 20 International students and 10 students from United States. We concluded that such diverse group would expand our knowledge and understanding on how and why participants use VPN. We offered 20 dollars amazon card as an incentive for participation in the interview.
\subsubsection{Participants}
In total, we conducted 32 interviews: 20 International and 12 from US. Four interviewees did not give consent to recording so detailed notes where taken during these interviews. Interviewees could choose between meeting on Princeton University campus and remotely, through Skype. We conducted 23 interviews via Skype and 9 were conducted on-site. 

\subsubsection{Demographics} Put the table

\subsubsection{Interviews}
Our interviews were semi-structured, so we asked all participants the same set of questions and asked additional follow-up questions whenever it was needed. We developed a question set based on our initial research questions, focusing on reasons why student use Virtual Private Networks and how they choose them. In order to get a better understanding of participants' knowledge and background, we asked also about their general privacy and security awareness before getting to questions about VPN usage. 
Before conducting the interviews, participants were asked to fill out the consent form and consent to audio recording along with a short survey, where we collected data about their demographic information and their general online habits and behavior. 

\subsubsection{Analysis}
We first transcribed all recorded interviews and developed extensive codebook in order to apply it on transcriptions. The same was done to notes collected from interviews without audio recording. The codebook was first based on set of interview questions but it was also refined throughout the process of code application. We had x parent codes in total, for example xxx and xxx; and x child codes, like yyy and yyy
\subsubsection{Knowledge about tracking and privacy}
\label{sec:methods-tracking}

\underline{Privacy}
Interviewees scored a median of 2 on General Caution scale and a median of 3 on Privacy Concern scale, which show that they were about below to average in terms of privacy attitudes. Nevertheless, interviewees had a median score of 4 on Technical Protection Scale. Such rating could indicate that our interviewees were less concerned  about the their privacy because they actively acted against its violation. 

On the SeBIS scale, interviewees scored a median of 5 on Device Securement, 3,5 on Password Generation, 3 on Proactive Awareness and a median of 3 on Updating. On the whole SeBIS scale, interviewees had a median score of 3, so an average score of Security Behavior Intentions with an above average rating of Device Securement.
\underline{Tracking}
Before the questions about VPNs, we asked our interviewees about their general knowledge about tracking. 
%\subsubsection{Censorship}
%At least 18 interviewees defined censorship on the Internet as removal of content on online platforms by higher authorities, such as big companies (6 interviewees) and government (3 interviewees). P10 made distinction between two different types of censorship, governmental one and in private sector:
%\begin{quote}You have censorship as a form of official government policy, and you see that in countries like China with The Great China Firewall and then I think you have, oddly enough, a private censorship. When you have sites that have become defacto public spaces, as Reddit or Facebook, although they are ostensibly private, they're owned by a private company, hence they do have the ability to censor however they'd like, it's sort of turned into a public space in the sense that the greater public meet the [inaudible] political opinion. However, those private spaces are often censored. (…) I think the official government censorship is terrible. (…)Well, I believe very strongly in the rights that, at least as Americans, we're afforded under the First Amendment, being able to freely say and think what we'd like, and freely associate with who we'd like to. Obviously, unless that speech crosses into violence, or rather encouraging violence, or I guess even stronger than encouraging. I can't think of the right word right now, but really compelling violence. I think unless it crosses over into that, you should be able to say and associate with whomever you'd like. Then as far as the private censorship, I mean, I think there's a case to be made for these sorts of defacto community sites historically treated a bit more how traditional media is treated whereby they are compelled to give airtime to both sides of an issue.\end{quote}


 %At least 17 defined it as limiting access to what people see or search and 8 noted that censorship depends on your geolocation. Moreover, 17 interviewees believed that censorship is worrisome, 4 of which thought that censorship is bad for democracy. At least 18 interviewees considered “free speech” as important. Three interviewees declared that in their opinion there should be no restrictions on the Internet but another 23 argued that there should be some regulations applied, which would filter harmful or dangerous content. 
 %\begin{quote}Sometimes I think it's necessary, sometimes there are things that are on the internet that shouldn't be on the internet for certain audiences, for example; children. I honestly think there should be a lot of censorship for children because there are some topics that just simply children are not ready for or shouldn't even be introduced to when they're young it's a matter of, in a way it's keeping the innocence because somethings on the internet are just very violent and yeah, again maybe censoring things for specific ages. If they're under the age of 18 because they don't have consent or anything, is necessary. I also think censorship is necessary in there's like eight, like for example I read an article recently where Facebook deleted 600 or X amount of accounts because of these Facebook groups were actually banding together and just like really, really hate crimes. They were planned hate crimes and they were advocating for more hate in other countries. (\dots) I'm all for free of speech, but when it comes to like imposing life threatening things then censorship might be necessary to kinda bridge the gap between like peoples safety and like freedom of speech. (P23)\end{quote}
%Moreover, seven of our participants believed that censorship should be left to the individual itself. 


%P10 defined censorship as an unequal treatment of different entities:

 %\begin{quote}As I understand it with American news media, if they give a certain amount of time to, say, the Republican candidate, then they need to give an equal amount of time to the Democratic candidate. I think there's something to be said for that, especially if you're trying to position yourself as the front page to the Internet as, say, Reddit does, for example. Yeah, or if you're trying to position yourself, I think, as a fair, impartial kind of platform like Twitter, you can't really go about censoring one side of an argument just because it's uncomfortable or just because you don't agree with it.\end{quote}


%P15 raised a subject of money related to censorship:  
%\begin{quote}I get it. I understand why the rules are strict and I understand that usually paying for the services that you're using, but I also think it's really hard to make people not choose the easy way and take, get stuff for free, instead of paying for it, especially because some services are pretty expensive. Netflix is \$13 a month, which is not something cheap if you think about how much your paying for the whole year and compared to how much your watching it. Same with Spotify and other services that everyone uses, but \dots yeah. It's definitely better to pay for music or films than to get a fine of \$2,000 for doing that and even get a criminal record and the actual sense for criminal record for something like that.As I said, I understand where this is coming from, but I think it's really hard to \dots keep track of how many people are actually paying, how many people are finding ways to get away with it.\end{quote}


%At least 18 interviewees believed that censorship has political background while 11 that its reasons are more societal, for example 4 interviewees from the latter group pointed purpose of censorship to be forming’s someone’s opinion.
%P27 talked about the bias that censorship produced between different groups of people:

% \begin{quote}And then the other instances of censorship that I've seen are like all of the black economic pages that I follow. A lot of times their posts get removed and I think one of my favorite social media personalities, she had a magazine shoot and she was naked on the cover. But she's a black woman and she's also doesn't fit body standards of beauty and so she's on the bigger side. She's heavy set. And they took down her photo but Kim Kardashian, her photos never get taken down but she's always half naked on the internet and stuff like the Tomi Lahren fan page.Tomi Lahren is a conservative spokesperson and her posts never get censored when they're hateful and things like that and are very political. So I think a lot of the censorship is biased.(\dots)I don't think it's fair. I think it's biased. That's exactly what I think. I think if I [inaudible] posts remove and other people not having their posts removed, it gets unfair and I also think it's representative. It's representative of the values of the United States.Even when you type in, if you type in black teen on google, I'm not sure if it's like this anymore, but the fact that you can change it kind of worries me because when you do a Google image search, you expect it to be random. Type in black teen, there's pictures of delinquents and mug shots and drugs and money and stuff like that. When you type in white teen, you get all these happy white teens in school and things.So I think that censorship kind of confirms but also perpetuates some of the racial stereotypes we have about people.\end{quote}

\textbf{What is Tracking}

Twelve interviewees defined \textit{Online tracking} as creating data about an individual, and gathering data about individual's location (5). More than half (17) of interviewees said that it happens through user's search history and general online activity. Another 2 that it can be done through web camera but also through apps (1) or looking at key logger passwords (1). Also, for 12 interviewees  \textit{tracking} meant that someone could see what they are doing online and 6 believed that it happens each time you go online, while 3 stated that you are not able to hide.

\begin{quote}To me that would mean collecting data about a user in any service or any capacity while they're online or doing anything that involves being online. That could be something like keeping a lot of the messages you sent, even if you delete them or that could be something as simple as just tracking what times you sign on to different things. I think that also applies to things like, with our [Proxy] cards, every time we buzz into a building that's documented somewhere and that's documented online, so that kind of goes into online tracking as well. Anytime we're using an online system, information we recorded. (P32)\end{quote}



\textbf{Comfort with Tracking}
For 4 interviewees tracking was frightening, and tracking made 2 interviewees' overall online experience uncomfortable. Moreover, 8 interviewees believed that tracking was something bad that should not have place. For example, P13 believed that tracking is more visible online than censorship.
%P09 – write about tampons story 
Moerover, P25 did not feel like there was any good law in place that would make users' data safe, which led them to believe that companies have low standards of data processing and make users more vulnerable. 

On the other hand 7 interviewees interpreted it as something good, for example using it as a tool for crime investigation (2) or suggesting ads with clothing (2). 

\begin{quote} it [online tracking] could mean positive tracking. I guess Google tracks many things that I do and I'm okay with that and I'm aware so I don't do too personal searches or anything. It's just for personalized advertising or search results that could be beneficial. (P31) \end{quote}


\textbf{Who Tracks Data}
We asked our interviewees who they think track online. 18 interviewees believed that big companies, such as Facebook (4) and Google (3) track people online. On the other hand 15 interviewees pointed government as an institution that tracks, naming NSA (3), FBI (2), CIA (1) and IRS (1). Furthermore, 8 interviewees stated that every website tracks its users, 6 that it is done by advertising agencies or data companies. Four interviewees said that ISP can track its users, another 4 that it is done by hackers and 2 that school/university tracks your online activity. Five interviewees said that tracking is not done by anyone in particular.

Twenty-seven interviewees admitted that they believed that they are being tracked. Thirteen of which believed they are tracked by big companies, such as Google (3) and another 13 that even though they are tracked, they are not targeted. 


\begin{quote}I think the way that I, totally uninformed way you think of it is that there are probably machines, some AI, some program that looks for certain keywords and a certain combination or indicators, histories. I'm sure there's some algorithm that people have developed. I don't think someone is necessarily actively tracking me but I'm sure there is some software that I, along with many other people are being tracked by and if something I were to say was flagged on that, then I think there would be an individual, or at least more attention to me.(P32)\end{quote}


Five interviewees believed that they are not tracked, as they do not do anything suspicious.


\textbf{Reasons for Tracking}    

We also asked participants about reasons that these entities may have to track. Fifteen responded that tracking is needed to gather data about users' preferences, also 15 said that to gather data about our online behavior and 14 that it is needed for advertising. Nineteen believed that tracking is for financial interest. For example, P25 mentioned that tracking is an essential part of A/B testing. 

P23 associated online tracking with location services, that track users' geographical position. They also added about their feeling being watched:

\begin{quote} 
Essentially, I feel sometimes like, I don't know if you ever heard of the novel 1984, but it's just like George Orwell, like beautifully describes just being watched and how were under a lot of supervision or if you read this philosopher Foucault, who talks about this Panopticon, that's were constantly being just supervised by a higher power and this power being like the people who controlling these, this data, cause essentially it is data, like where you are, it's stored where your cache. And also like the things you like and the sites you visit it's all start in your cashe, in your history and a site has the permission to use that cash in order to market things a certain way for you, for example if you. This is like super intense, but like, I don't know, I googled a pair of shoes right and then like two seconds, maybe like five minutes later I was just scrolling through my Instagram feed and I see those same shoes being on an ad in Instagram, that's like annoying. I mean I know you want me to buy your shit because it's capitalism and all, but I don't need to know that you're, I don't need to be like primed into like I want to buy that thing. That's essentially what I mean by tracking.\end{quote}

On the other hand, P07 described tracking as repercussion of Internet speed debate. The internet providers would track users to evaluate the number of people that visit each website. The bigger company with more viewers, would get faster Internet to users, so Internet providers would control speed of the Internet for different companies. 





\subsection{Survey Methods}
\subsubsection{Design}

We chose to design a survey for our study as surveys are effective for collecting a large number of responses to describe a diverse, concentrated student population with low cost-per-response and high accessibility [28]. Our audience was Princeton students who are 18 and over and had used VPN before. We recruited participants through the Princeton Survey Research Center, which has access to a directory of all undergraduate and graduate students. We randomly sampled 2,748 students in three waves, and filtered these students through our invitation email and through pre-screening survey questions. Our final sample of 350 valid and completed responses is large compared to Princeton University's population (4.3\%), and we can assume that it is a sufficiently large proportion (>5\%) of VPN-using-students at Princeton to draw meaningful conclusions with [28]. We utilized the Survey Research Center's capabilities over launching a public survey in order to minimize self-selection bias.

We built the survey in the Qualtrics software, as a license was provided by the Survey Research Center. The survey was previewed and tested by members of the Human-Computer Interaction lab group to refine and fix bugs from 2/6/19 to 2/25/19. The survey was launched to the first wave of respondents on 2/26/19, the second wave of respondents on 3/4/19, and the third wave of respondents on 3/12/19; respondents were provided with reminder emails 3 days after being invited. We closed the survey to further responses on 3/29/19.

Survey compensation was provided through a lottery drawing of valid, completed responses for one of two \$250 Amazon gift cards.

\subsubsection{Content}

We split the survey into 8 blocks: Pre-screening, Demographics, Privacy and Security Awareness, Privacy and Security Practices, VPN Perception, VPN Preferences, VPN Usage, and VPN Issues and Improvements. Note that these survey blocks were created solely for internal organizational purposes, and they cannot be viewed by survey respondents. As such, the block titles do not correspond 1-to-1 with their contents, or with our Results categories. The original survey can be found in Appendix B.
The pre-screening questions filtered out respondents who did not consent to the survey, were under 18, or had never used VPN. The Demographics block included a question on academic major, as we aimed to study online behaviors related to students in different academic fields.

The Privacy and Security Awareness block asked about respondents' perception and concern about online data collection, including the nature of data collected, who is collecting data, and why they are collecting data. The Privacy and Security Practices block asked about respondents' usage patterns of different tools and tactics to combat online risks, as well as how they sourced them.

The VPN Perception and VPN Usage blocks aimed to gain insights into students’ knowledge and usage patterns of different VPN types, including specific VPNs they had used. We also asked questions on respondents’ motivations in selecting and using VPNs, as well as how they sourced VPNs. The VPN Utility block centered around the user-provider relationship, and asked questions about perception of data collection by VPNs as well as vulnerability while using VPNs. The VPN Issues and Improvements block asked about issues that users faced while using VPNs, and also included open-ended questions about what respondents liked and disliked about their VPN services.

In designing the survey, we generally avoided open-ended questions to prevent user fatigue and reduce the complexity of data analysis; as a result, we asked only 3 open-ended questions. We also avoided double-barreled questions, negative questions, and biased wording [28]. We included two attention check questions that required a certain response, and we discarded responses that did not pass both (6/356).

\subsubsection{Participants}

TODO: table?

Our target audience included all Princeton students, both undergraduate and graduate. We sent email invitations to a random sample of 2,748 people in this audience, and included a note in the email to filter for VPN users who are over the age of 18. We collected 452 responses, of which 392 were generated by respondents who fit our criteria. Of these 392 responses, 356 were fully completed. Of the 356 complete responses, 350 passed our attention checks and are considered valid responses for the purposes of analysis.

Figure 2 shows detailed demographic data of the respondents. As expected, the majority of them were age 25 and under (79\%, 275/350). The vast majority of the respondents were American nationals (74\%, 258/350); the countries with the next-highest representation were China (6\%, 22/350) and Canada (4\%, 13/350). Most of the respondents were enrolled in Princeton's undergraduate program (63\%, 219/350), with 28\% (97/350) of them enrolled in a doctorate program. Of the undergraduate respondents, 34\% (74/219) were fourth-year students. The most popular majors among respondents were computer science (13\%, 46/350), economics (9\%, 33/350), public policy (7\%, 26/350), and molecular biology (7\%, 25/350). Notably, computer science majors were not overrepresented in our sample.

Almost all respondents (99\%, 348/350) believed that some data was collected about them when they used the Internet. The vast majority of these respondents (N=348) believed that companies (93\%, 323/348), websites (93\%, 322/348). their government (83\%, 290/348), and Internet Service Provider (81\%, 282/348) were collecting their data; in contrast, only 2\% (8/348) of them believed that friends and family were collecting data on them. Almost all of these respondents believed that their data was collected for advertising and other financial motives (99\%, 343/348); a smaller majority believed that their data was collected for political motives, such as influencing political leanings (72\%, 252/348). Almost all of these respondents believed that their online activities (97\%, 339/348), interests and preferences (96\%, 335/348), and location (96\%, 333/348) were collected. Smaller majorities believed that demographic information (85\%, 293/348) and device type (81\%, 283/348) were collected. Significant minorities of respondents believed that more sensitive data including private messages (41\%, 141/348), keystrokes (31\%, 109/348), and recordings (31\%, 109/348) were captured. On a five-point Likert scale, participants were on average "somewhat concerned" about this data collection (mean 2.94, median 3).

Survey respondents used a wide variety of tools and tactics outside of VPN. Among the most popular were ad blockers (80\%, 280/350), using two-factor authentication (75\%, 263/350), avoiding spam email (70\%, 246/350), and using private browsing mode (63\%, 222/350). Fewer respondents utilized high-effort tactics such as changing passwords frequently (15\%, 52/350), using password managers (17\%, 60/350), or avoiding social media accounts (27\%, 95/350). More obscure online tools were also less popular among respondents, with only 12\% (43/350) using tracker blockers such as Ghostery and 9\% (33/350) using Tor. Most respondents (57\%, 198/350) used at least some of these tools most of the time when they go online, on both laptops (98\%, 343/350) and phones (77\%, 271/350). However, the overall effort that respondents put into protecting themselves online was low. Out of a maximum score of 35, the median "protection effort score" of respondents was only 12 (see section 3.5 for how this statistic was derived). Participants largely heard about these tools and tactics online (73\%, 256/350) and from friends and family (71\%, 248/350), and most (64\%, 225/350) started using these tools and tactics 3 or more years ago.

\subsubsection{Data Analysis}

We used Qualtrics and R as software for data analysis. We first analyzed the response data using tools built-in with Qualtrics. We limited our analysis to the 350 valid and complete responses.
Next, we searched for statistically significant correlations by finding pairwise correlations for each variable. The raw response data was unsound for this purpose as it contained 605 variables, including separate variables for each checkbox in multiple-selection questions. Attempting to analyze a dataset this large through Qualtrics software resulted in erratic behavior, and would not produce meaningful results without first cleaning the data using methods not available in Qualtrics. We cleaned the response dataset using R, and reduced the number of variables by consolidating them into fewer categories when possible. For example, our question on what tools and tactics respondents used to protect themselves online had 20 variables -- one for each tool. We consolidated these variables into a single "protection effort score" by assigning an effort score to each tool, based on how difficult it was to execute, and summing these scores for each response. This allowed us to both decrease the number of observed variables to 124, and also increase the number of observations for each variable. Appendix B contains more information on how each variable was consolidated.

We chose to analyze pairwise correlations instead of using more in-depth methods of data analysis for two reasons. First, this study intends to produce a broad overview of our topic; as a result, the sheer number of observed variables made it impractical to deliver detailed relationships between each combination of variables. Second, given that our survey could not be exhaustive, there are doubtless many unobserved variables that could significantly affect the accuracy of any detailed analyses. Given the nature of this study, we decided that the goal of this part of our analysis would be to quantify the degrees to which our observed variables were correlated using R and p values.

\subsection{Limitations}

This study is intended to provide an overview of college students' usage and perceptions of VPNs; as discussed in section 3.5, we chose to limit our data analysis to correlations due to our constraints. Future studies could explore a subset of our variables in-depth, and potentially account for confounding factors to establish stronger relationships. In addition, although we speculate on relationships between variables, our correlations do not have definitive directions of causality; further inquiry is needed to confirm any assumptions on one variable influencing another. In addition, the methods we used to consolidate our variables are just one interpretation of how to categorize our choices. Other researchers may find different combinations that produce new results.

Our survey also had some inherent drawbacks. Recall bias is difficult to avoid in any survey [28]. In addition, while our sample size of 350 was sufficiently large, we analyzed intersections of variables that had different N values, some of which could be small enough to introduce sampling error [28]. Our survey was also not completely anonymous as it required participants who wished to enter our compensation drawing to submit an email. This could introduce error in the respondents' levels of honesty. Future researchers could build on our results using alternative methods.

Finally, we limited our survey pool to Princeton students due to ease of access, and used our respondents as a representative sample for college students in the US. However, this introduces sample bias as Princeton students could display specific traits. Future researchers could replicate our study with a more general college student population, or with other populations of interest.