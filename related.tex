\section{Related Work}\label{sec:related}

With privacy concerns and data breaches coming up more and more frequently,
how do people perceive their safety on the Internet and how do they protect
themselves? While the technical intricacies of the Internet may not be
critical to know in everyday usage, some level of understanding could be
beneficial to protect against threats. Analyzing this level of understanding
can help Internet software designers to create effective software that is more
easily understood by users, and policy-makers to create rules that help
facilitate this understanding. In the context of VPNs, understanding users'
perception of what VPNs provide in comparison to actual features can help VPN
providers create more effective messaging about their services.

Previous studies have analyzed users' knowledge of the Internet in general, as
well as privacy and security practices. A number of researchers have analyzed
users' mental models in their perceptions of the Internet [16, 17]. In terms
of data collection, Americans are concerned by the online tracking usage of
their data by outside entities, but lack an understanding of how their data is
used or transmitted [18, 19]. In particular, they expressed higher concern
towards third parties track and collecting their data [8]. In general,
however, users are confused as to how this online tracking works and how they
can protect themselves [9].

Researchers have also tried to find stronger connections to the usage of
privacy and security tools. One study suggested that a combination of
awareness of, motivation to use, and knowledge of how to use privacy and
security tools impacted their usage [20]. However, another study focused on
online privacy and security attitudes and behaviors found that while Internet
users with stronger technical backgrounds were more aware of privacy and
security threats, they did not engage in more secure practices than their less
knowledgeable peers [2]. The phenomenon of tech-savvy users neglecting to
utilize their knowledge to protect themselves could have interesting
implications for VPN-focused studies.

Related studies have attempted to reveal users' motivations behind using
online tracker- and ad-blocking tools [11, 12, 13]. Other researchers have
studied the usability of these extensions [14, 15]. One recent study has
attempted to comprehensively characterize the perceptions, usage, and
effectiveness of browser-based blocking extensions, revealing more information
about users' attitudes and behaviors with regard to their online privacy and
security [5].

We aim to gain related insights by studying users' perceptions and usage of
VPNs, beginning with college students. Some studies exist on the effectiveness
of VPNs, including research on data leakage and other flaws such as traffic
manipulation [1, 6]. However, there is a lack of literature on VPNs from the
end users' side, regarding their perceptions and usage of VPNs as well as how
they relate to their attitudes and behaviors towards privacy and security. In
addition, there have been no published usability studies on VPN services.
Further exploration is needed on the end user-side of VPNs to better inform
their design and messaging.
