\subsection{Findings}\label{sec:results}

\subsubsection{Users Know VPNs Mask Internet Activity In Some Manner}

The majority of interviewees and survey respondents defined a VPN as changing what others can see about your Internet activity. For instance, almost half (14/32) interviewees defined a VPN as an
Internet activity routed through third party machines and almost half (14/32) as a service for changing your IP
address, masking your identity (10/32) or reducing others ability to track
you (10/32).  \begin{quote}It's sort of a middle man. So instead of you actually
downloading the file from someplace where somebody might be looking at you
downloading it, they download it for you and then they send it to your
computer. So it figures that they downloaded it and not you.( P18)\end{quote}

Survey participants were also asked to report in an open-ended response, what they
thought a VPN was. We categorized these text responses by topic based on their
contents; some responses contained multiple topics. \mc{Andre, what did the rest of them think?} 42\% (147/350) of
responses mentioned that VPNs are tools that mask the user’s location. Only
4\% (14/350) of respondents indicated that they did not know what VPNs were.

The details of how the VPN worked however were less clear to interviewees and survey respondents. Some believed that VPNs allowed you to  access blocked content (13/32), allowing access into another network (7/32) and others described a VPN as secure,  private, or adding an extra level of safety (13/32). Similarly, most survey respondents indicated that they thought their VPNs guaranteed access to certain content
 (75\%, 263/350) and masking of their Internet Protocol addresses (53\%, 186/350).
Only around one third of survey respondents believed that their VPNs guaranteed privacy (36\%,
125/350), anonymity (30\%, 104/350), and safety from tracking (28\%, 99/350).

P25 descibed a VPN as:
\begin{quote} Its usefulness is pragmatism, it's like, "I need to see this YouTube video, but they don't let me see it in Brazil so I'm just going to do it
    in Belgium." I think that that's what VPNs are to me. \end{quote}
    
A few interviewees felt that using VPN is
bypassing rules and laws (5/32). A minority of participants defined VPN as a blackbox and described people who are
interested in VPN as technology-savvy (4/32). For instance, P27 shared their mental model of
people who use VPNs: \begin{quote}I feel like you have to be a super nerd to  understand what all that is and to be able to manipulate it in any
way\end{quote}

Most interviewees and survey respondents felt they were not experts in VPNs. A good proportion of interviewees (12/32) admitted that they did not have an extensive knowledge
about VPNs. In the survey, 76\% (266/350) reported having some knowledge of
VPNs, and 19\% (68/350) reported having no knowledge. In contrast, only 4\%
(16/350) considered themselves to have high knowledge or to be experts on
VPNs.




\subsubsection{Users Expect VPNs Allow Tracking But Also Expect Them To Provide Anonymity} 
Three quarters of interviewees (24/32) responded that in their opinion you can be tracked
while using VPN and some believed that there is always a way to do so
(8/32) but also because you can be tracked by VPN provider itself (9/32). 

%Check in dedoose if P01 is in this 9:
As P01 explained:

\begin{quote}So if it is SSL encryption, the VPN provider would still know
that you are communicating with a certain web service but the VPN provider
would not or probably not know the contents of the communication if it's SSL
encrypted. They would only know who you want to communicate with. And if it's
not encrypted, then they can see, they can be do packet sniffing or even more
malicious things like deep packet injection and deep packet inspection to
actually look at the contents of that communication and do potential malicious
things with that. \end{quote}

At least 20 interviewees did not
believe that VPN guaranteed them anonymity, although 8 interviewees admitted
that VPN offers something, that in their opinion, is the closest to anonymity.
On the other hand, 7 interviewees thought that VPN guaranteed them privacy and
6 access to sites they wanted to visit. Six interview respondents believed
that nothing is guaranteed.

As P04 explained: \begin{quote}If I use even a private VPN that I paid for, even
though the majority of the world does not see my IP address and everything, I
feel like the owners of the VPN provider will be able to still see it. And
then it's just a matter of having that security breached. So I don't think
there's ever really a sense of true privacy. Unless I make my own
VPN.\end{quote}



Also, 4 interviewees admitted that while using VPN, one could be tracked by
the government. For example P30 used VPN only in different countries to access
blocked content. They stopped in USA as they did not need VPN's access
properties anymore and did not see any privacy protecting reasons because all VPNs are \textit{"partially, controlled
or transparent to the government"}.

Moreover, two interviewees responded that one can still be tracked by
 advertising agencies. Although, 3 interviewees admitted that VPN makes
 tracking at least harder than normal. 

P21 explained that using VPN is not enough. In order to make tracking harder
for companies, they changed locations that they connected within VPN:
\begin{quote}Yes [I can be tracked while using VPN], especially if I'm using
the same IP address. That creates a problem because my internet footprint ...Or Chrome, for example, my web browser could definitely still track me
and connect that, see where I've been connecting from. Or Gmail could see
that. Gmail always tells you, "Oh, you've connected from this weird device, or
from this location that we don't recognize." So I think you can definitely
still be tracked.\end{quote}

\subsubsection{Users Know VPNs Collect and Share Data About Usage}

We found that 74\% (260/350) of survey respondents believed that their VPN
collects data. These respondents believed that VPNs collect
location (226/350) and online activity (192/350) data. Fewer of these survey
respondents believed that VPNs collected more sensitive, “nefarious” data,
such as private messages ( 44/350), recordings (39/350) or
keystrokes (15\%, 38/350). The majority of these respondents believed that
VPNs collected this data for commercial motives ( 178/350), or simply
because data collection is a “default consequence of using the Internet”
(178/350). Around half of these respondents (127/350) selected
both of those options.

In our interviews, we asked whether
interviewees believe that VPN providers keep their logs and other information stored.
23 interviewees answered “yes”, giving examples such as keeping information
for user statistics or to sell data. Some also referred to the
university’s VPN (7): 4 interviewees thought that university wants to have an
access to all information about students and 3 believed that VPN helps university to monitor
if someone’s cheating during exams. There were only 9/32 interviewees who did not
think that VPN providers could keep their logs.



\textbf{Who VPNs share data with} We also asked our interview participants
about their opinion on other VPN practices. When asked whether they thought
their VPN providers could be sharing their information, 17 responded “no” and
11 “yes” but 12 were uncertain about their response or did not know the
answer.  Moreover, from these who said that VPN providers do not share
information with other entities, 8 confessed that they hope their information
are not being shared, 5 admitted that while their VPN providers do not share
any information, others do. Moreover, 2 of these interviewees believed that
even though their VPN providers do not share data with others on regular
basis, they would with legal authorities. 

There was little consensus on who had access to the data collected by VPNs.
The largest proportion of survey respondents believed that companies (
113/350) and the government (88/350) had access to the data. A
smaller number believed that only the VPN had access (47/350). 14\% of survey respondents indicated that they did not know where
their data went (50/350). Of the 213 respondents who believed that their
VPN shared their data, they generally thought that their online activities
(150/350), location (157/350), interests (123/350) and
demographic information ( 122/350) were shared. Figures 9 and 10 contain
a full breakdown of response data.

\textbf{Users Do Not Feel VPNs Have Transparent Practices}
We asked our interviewees whether in
their opinion their VPN providers were transparent. 14 of our interviewees
replied that they did not feel like their VPN providers were transparent,
while 12 believed that they were. 

At least 15 interview participants answered that they did receive description
on how to use VPN from their providers and 14 admitted that they did not
receive any description.  

Moreover, 13 believed that it is important to get such description for
multiple reasons; interviewees believed that such description is necessary in
order to know how to make the best use of VPN, as well as to know what VPN
provides. They believed that it would make usage easier, would alert them to what should and should not be done while using a VPN, how a
VPN works with their devices, and how to connect to it properly. On the
other hand, at least 9 interviewees believed that using a VPN is very easy and
at least 8 did not consider it important for getting such description. 

\subsubsection{How people choose VPN} \label{sec:findings-choosing}
\textbf{Guidelines when choosing VPN} We were also interested in factors that
our participants take into account when choosing their VPN provider.

For the majority of our interviewees (19) the most
important was good reputation of the VPN provider and 3 interviewees added
that the fact that their friends had used it before had an impact on their
decision.  Moreover, 10 interviewees had different
security and privacy requirements such as making sure that the VPN had a secure network,
that the VPN provider did not store any of users‘ records, that the VPN provider did
not sell users‘ information and that the VPN protects users‘ data. For one
interviewee it was important that VPN did not require any personal information
when setting up the account and another one that there was an option of secure
payment. 

 Another factors considered by our interviewees were also
ease of use (8/32), speed (7/32) and ease of set up (5/32). Moreover, 5/32 participants
would look at the price before purchasing subscription and for 5/32 it was
important that VPN was for free.   

For example for P11, the main factors were word of mouth, experts’ opinion,
cost as well as customer service available: \begin{quote} I look
    on, Tech Radar and PC Monitor, those kinds of websites, to ascertain
    whether experts thinks it's the best. I get some reviews from friends, as
    well, see if their having a good experience with VPNs, and then I'll go on
    website if I'll, I think I narrowed it down to a couple of
    options. So, when I came to China I was deciding between Express and
    Astro, and I just looked on their websites, went through, server
    locations, cost, and their privacy policies, as well, and then into
    deciding on Express. and then, I also had Express available
customer service, which was very important as well.\end{quote}

When asked interviewees how they determined whether
their VPN provider was trustworthy, 13/32 interviewees said that it had
good reviews online. Another 10/32 would verify that through word of mouth and 7
knew it was trustworthy because of the providers of their VPN, e.g., the
university. 

The most important considerations for survey respondents (which was reflected in the interviews as well) when choosing VPNs
were cost (76\%, 266/350), ease of use (66\%, 266/350), and speed (60\%,
210/350); many respondents also regarded security (58\%, 202/350) and privacy
(48\%, 169/350) as important. Branding (20\%, 69/350) was somewhat less
important.

Survey respondents who regarded security as an important factor tended to avoid using
free VPNs. These respondents also tended to care about privacy when choosing
VPNs, and less about VPN brand or accessibility (R=.42). Respondents who cared
about privacy when choosing VPNs tended to care less about accessibility
factors, and also used VPN more often and in more places.

\mc{Andre is the paragraph below based on survey data or opinion?}
College students who regard security or privacy as important factors when
choosing a VPN tend to care about the other as well, and are more diligent
about using their VPN. In addition, these students tend to care about
tangible, usage-related features of their VPN over “softer” factors like
branding and accessibility.



\textbf{Users Learn About VPNs Primarily From Friends and Family} Half of our interviewees (16) found out about
VPNs and their VPN providers through word of mouth. At
least 9 interviewees learned about VPNs through their school or university and
7 through research. 4 interviewees admitted that they simply searched VPN
providers through Google search engine and used first thing in
search results. Moreover, 3 interviewees were told about VPNs in stores while
purchasing phone, 2 interview participants got to know about VPNs because they
were travelling to China and another 2 learned about them at their jobs.


P26 learned about different VPNs through advertisements on websites such as
The Pirate Bay: \begin{quote}So typically when I'm accessing those websites,
    like open-source libraries and especially The Pirate Bay, they always have
    different ads for different VPNs. So that's where I typically get a lot of
    them. I'll download them, try them out. Typically, I'll stick to the free
    ones.\end{quote}

The majority of respondents indicated first hearing about VPNs through friends
and family (61\%, 21/350) and/or online (53\%, 185/350). Smaller but
significant minorities reported hearing about VPNs through their institutions,
such as their school (38\%, 132/350) or employer (15\%, 52/350). The majority
of respondents started using VPN between 1 and 5 years ago (62\%, 218/350) and
in college or high school (80\%, 280/350).

Students who first heard about VPNs online or through personal connections
were more likely to use commercial VPNs; in contrast, students who heard about
VPNs through their institutions were more likely to use institutional VPNs
(R=.48). Students who heard about VPNs online generally did not feel safe
using institutional VPNs, and also tended to use VPNs for privacy and security
reasons; in contrast, students who heard about VPNs from their institutions
felt less safe when using free VPNs. Students who heard about VPNs online or
through personal connections tended to learn how VPNs worked from the same
sources (R=.38, R=.48). Students who heard about VPNs through personal
connections were more likely to consider VPN brand as an important factor in
choosing a VPN, and were generally less concerned about protecting security.

It appears that students who first hear about VPNs through online research
tended to be more concerned about privacy and security, and less trustful of
institutional VPNs. 

\textbf{Users Learn how VPNs Work By Searching Online}
We also asked survey participants to report where they learned how VPNs work.
The majority of respondents (55\%, 193/350) indicated that they learned how
VPNs work from online research, and 36\% (125/350) of respondents indicated
that they learned from friends and family. A smaller percentage of respondents
reported learning this information from their VPN providers (21\%, 72/350),
and very few reported learning from an expert’s testimony (5\%, 16/350). A
substantial number of participants (21\%, 72/350) did not know how VPNs work
at all.  Students who learned how VPNs work from their provider tended to be
more concerned about data collection in general, and tended to use their VPNs
more. Students who learned how VPNs work through online research tended to
protect themselves with more other online tools (R=.33). Students who
indicated that they VPNs to protect their privacy were more likely to have
learned how VPNs work through their providers or through online research.
Similarly, students who used VPN to protect their security were more likely to
have learned how VPNs work through their providers or through online research.
Students who learned how VPNs work online or through their providers also
tended to put more effort into verifying that their VPN was working.

Interestingly, students who learned how VPNs work through online research or
through their VPN provider tended to be more diligent and conscious about
their VPN usage. Students were also more likely to learn how VPNs work through
the same channels where they first heard about VPNs.







\subsubsection{Usage} \textbf{First experience} Our interviewees were
introduced to VPN in many different ways. At least 8 started to use VPN at
institutions like university, 4 were looking for ways to bypass geographic
firewalls, so they could watch something, 3 found out about this through their
parents and 2 were told to use it when they were going to China. Three
interviewees reported that they could not remember their first encounter with
VPNs.  At least 9 interviewees described their first experience with VPN as
not good, for various reasons. Some could not connect (2), for some it was too
slow (2) and some found it confusing (2).  On the other hand, at least 2
interviewees found it simple to use for the first time.

%How do you use it 7 interviewees reported that they used it only when they
%needed. Should we talk about it? It’s out of 14. And vs 4 always on.

\subsubsection{VPNs Are Not Used Consistently Over Time}

The vast majority of respondents used a VPN only rarely (46\%, 161/350) or
sometimes (41\%, 142/350). Few participants reported using VPN most of the
time or always (13\%, 47/350). Figure 11 displays a further breakdown of
locations where students reported using VPNs. Students who were more concerned
about data collection, more knowledgeable about VPNs, and put more effort into
protecting themselves online tended to use their VPN more frequently. Students
who were paid VPN users also tended to use their VPN more often (R=.36).



Of the surveyed respondents, only 42\% (148/350) reported that they were still
actively using VPN. These users tend to be more concerned about data
collection and put more effort into protecting themselves online. They are
more likely to have higher VPN knowledge, and to have started using VPN 3 or
more years ago. They tend to use their VPN to protect privacy and security,
and are less likely to use free VPNs.

The respondents who stopped using VPNs (58\%, 202/350) reported that they did
so because they were no longer location restricted (46\%, 92/202), did not
have anything to hide (37\%, 74/202), or simply did not use it enough (34\%,
68/202). Very few respondents reported a lack of security (1\%, 3/202) to be a
contributing factor in their decision to stop using VPNs.

Respondents were generally not diligent or consistent about VPN usage and
verification; most prior users had even abandoned VPNs entirely. We can
speculate that students who have maintained their usage of VPNs still use them
for privacy and security reasons, as opposed to purely a desire to access
content. Students who use paid VPNs also tend to use them more frequently and
more diligently. For students who stopped using VPNs, indifference to privacy
and security was a theme as they primarily used it as a tool to circumvent
censorship and other location-based restrictions. This trumped other factors,
such as cost or speed, which we initially speculated to be top-of-mind for
students.

\subsubsection{Users Use Free VPNs But Assume They Are Less Secure}

When asked whether it was important who their
VPN provider was, 14 interviewees reported that they would not pay attention
to this information and they simply did not care about this, while 11
interviewees reported that it was important, especially for these who used
university’s VPN (7), which was reassuring for them that this VPN was safe. On
the other hand, for 2 interviewees it was not important that university was
their provider, as one explained, university is only a client of a different
VPN provider, not a VPN provider itself. 



 22 interviewees reported that they would use free VPNs but with
many comments on it and restrictions,such as making sure that it was safe,
while 9 more strictly said that they would not use it. Among these from the
former group, 6 admitted that they did not need VPN, nor used it often, which
is why they did not mind using free VPN. 

As P24 shared, they were not a regular user so there was no need to pay.
Nevertheless, paid VPN could mean more secure VPN: \begin{quote} I think the
one that you have to pay for is more trustworthy. But, it could easily be the
other way around. Just because you have to pay for something doesn't mean that
it is more reliable, or even more efficient. But, I do think that the paid
ones generally people might think that they are more safe to use. And that
their information may be more secured, just because of that added price tag on
it.\end{quote}

P03 was also suspicious of free VPNs: \begin{quote}I would definitely try a
free VPN, but at the same time, if others cost money, and this one is free,
I'm like, so why aren't you? Why are you free? That'd definitely make me a
little, if most of the VPNs cost money and that's one free, then I'll be
suspicious. But then if it's like half of them cost money and half of them are
free. I'll like, oh, maybe the ones that cost money are for people who have
really, really, really sensitive information. And the ones that are free are
maybe for people who are just casual internet users who want that extra level
of security, but don't have a justification to need to pay money. But I'd
definitely be wiling to purchase a free one if I just saw what other VPN, like
I compare the reasons as to why some cost money and some didn't cost money.
(...) I guess maybe the ones that cost money offer a lot more protections. I
don't know if some VPNs are more secure. Maybe the ones that cost money are
very, very secure, whereas the ones that are free are maybe just basic level
of security, just a little bit better than what you get on the internet. Just
things like that.\end{quote}

As well as P14, who was worried that installing free VPN could be unsecure:
\begin{quote}I was worried that I possibly also downloaded some virus along
with the VPN, so, that's also what motivated me to just purchase ExpresVPN and
just deal with the cost because I wasn't really sure if what I had and so it
was from a trustworthy place whether that's a reasonable or not.\end{quote}

5 interviewees said that they would use it only if they were sure that
particular free VPN is secure. For example P12, would check reviews first:
\begin{quote}Why I want free VPN? 'Cause it's not very important to me, and I
don't have a lot of money, and I don't spend money on things that I don't
really need. Why I think I can trust it? I mean I would read up on people's
reviews of the privacy status of different VPNs to choose the one that I
install. I would only install it if I get a pretty good impression from the
reviews that I read.\end{quote}

We found that most respondents (65\%, 228/350) used VPNs through their school,
which was not surprising due to the accessibility of Princeton’s free VPN
service for students. A significant portion of respondents also used
commercial VPNs, including both free (49\%, 172/350) and paid (28\%, 97/350)
options. Few respondents set up their own personal VPNs (7\%, 24/350).

Paid commercial VPNs that a substantial number of respondents reported using
include ExpressVPN (13\%, 44/350), NordVPN (6\%, 20/350), and Private Internet
Access (6\%, 22/350). Free commercial VPNs that a substantial number of
respondents reported using include Hotspot Shield (13\%, 45/350), TunnelBear
(11\%, 37/350), Hola (9\%, 31/350), and Betternet (8\%, 29/350). For both
options, several respondents indicated that they used SonicWall and/or Connect
Tunnel which is the software used by the Princeton VPN, indicating that some
students are confused about the distinction between institutional and
commercial VPNs, as well as VPN providers and VPN software.

We asked respondents to indicate how safe they felt when using free, paid, and
institutional VPNs on a 5-point Likert scale. We mapped these choices to
integer values 0 through 4, and found that respondents felt safer when using
institutional (2.56/4.00) or paid (2.45/4.00) VPNs than when using free
commercial (1.84/4.00) VPNs. Students who started using VPN in high school or
earlier are more likely to use free VPNs (R=.46). Students who use paid VPNs
tend to have higher VPN knowledge and concern about data collection, and put
more effort into protecting themselves online. Paid VPN users tend to use
their VPNs more often (R=.36) and more carefully, and they also tend to use
VPNs for privacy and security reasons. Institutional and free commercial VPN
users are more likely to use their VPNs for content access reasons. Free VPN
users are more likely to think that their VPN providers collect their data,
while paid VPN users generally do not think so. Institutional VPN users tend
to be indifferent to their providers collecting data. Free VPN users were more
likely to have issues with ease of use, features, and content access.

Paid VPN users are generally more privacy-conscious, caring more about online
protection and putting in more effort to secure themselves. They use their
VPNs more frequently and carefully, and feel safe using their paid VPN over
other alternatives. In contrast, free commercial and institutional VPN users
seem to use their VPNs primarily for content access.


\subsubsection{Using institutional VPN} Out of these who used institutional
VPN, 5 reported that they would use it only for work, because they simply did
not feel private (3), they felt that university could track them (2) and
institutional network was vulnerable  (2). P13 gave an example about the
student who shut down exams by hacking into university's VPN through another
VPN. %ADD HERE THE QUOTE ABOUT STUDENTS HAVING ACCESS TO POSTDOC’S WORK.

On the other hand, 5 interviewees used institutional VPN for private
activities, like browsing, as well. For example, one of them believed that it
added security, and one would simply not switch it off. 


\textbf{Reasons for VPN usage} The majority of our interviewees (21) reported
that they use VPN to bypass geographic firewalls, 15 of which to watch movies
or TV shows online.  

\textbf{High school} For example P20 used it to get access to sites that were
blocked by his high school: 
%(also interesting word of mouth example and not allowing apps to see
%%location thinking that they can do it anyway): 
\begin{quote}I've used them for a few reasons,
    but privacy was never really one of them. It was just when my content was
    restricted when I was in boarding school, I went to boarding
    school for high school. Our WiFi was very tightly patrolled. So any number
    of things were blocked, like from adult content, to a lot of sports
    websites for instance were blocked, because they "encouraged gambling,"
    and I like to watch a lot of sports online illegally, because that was the
    only way I could watch them.\end{quote}

\textbf{Geographic blocking} On the other hand P24 used VPN not only to watch
TV show but also for accessing their app account that s/he subscribed to while
being in America and could not use it at home country: \begin{quote}I was in
New York this past summer, and so, I had a subscription for a couple months.
But then I came back to Vancouver, and wanted to end the subscription. But, I
couldn't, I found out that I couldn't access the app or login to my account
through the phone app, because I wasn't in the United States, basically. So, I
just thought like, hey maybe if I use my phone to change my IP address, so it
looks like I was accessing the app from the United States. And I downloaded, I
think it's called Express VPN. And that was able to help me work around the
location, geographical issue, and access the account so I could cancel the
subscription. So, that was a pretty useful tool for me.\end{quote}


P11 used VPN for news websites that were not blocked but had different or
limited content depending on IP address of the Internet user. As they were
from UK, they wanted to get into UK BBC website, while being located in the
US.

Also P25 shared that s/he used VPN in order to obtain different goods and for
piracy and for geo blcoking reasons: \begin{quote}I don't use it for privacy
much anymore, I use it as a way to separate my own identities mostly, I use it
when I need to be in another country theoretically. I think that that's the
most common way to use it for me, if I go and I try to see a video and it's
not available in my country I just scroll down from Brazil to Belgium and
usually it works. I think that that's what I mostly use VPNs for. Or if I'm
doing something that I feel I wouldn't want TNSA to see, TNSA is an abstract
thing in my head. If I didn't want the Panopticon to see it, I would turn on
the VPN but I don't think that ever happens.\end{quote}

Moreover, 7 interviewees would use it to access materials that normally they
would have to pay for, 5 admitted they used VPN for piracy and 3 for
downloading content.

P26 also used it for their relative to be able to watch a tv show:
\begin{quote}With the university, there
    are certain free textbook websites that they don't like students going on,
    but then I use my VPN to be able to access those sites. For example,
    Venezuela has blocked everything coming from their YouTube channels, and I
    have my mom reroute the US IP address to a Mexican IP address with a VPN,
    so then she could watch her Venezuelan TV shows.\end{quote}

\textbf{VPN is not used for Privacy} 13 interviewees said that privacy was not
their reason to use VPN. Fewer (7) participants used it to protect their
personal information, as for example P21: \begin{quote}I guess I don't like
the idea of Princeton or an ISP being able to see all of my traffic. I don't
think that I trust anyone with all of my traffic or consumer habits. I don't
want that stuff to be sold off. I don't want them to be able to build a
profile of me because it can be quite revealing, especially because you're
device\end{quote}


Also, 5 interviewees would use it while on public Wi-Fi and 4 while
travelling. 5 reported that they used it for security and 3 for anonymity.
Moreover 3 interviewees used it because they liked the idea that there was a
"free" space on the Internet and P25 believed that using VPN is a statement
that security is important:

\begin{quote}it's why Private Network Access I think? (...) It's why it got so
popular. They tried to subpoena the guys to release information about some of
the people who used the VPN, and then they actually didn't have it on their
servers. So people knew that they didn't keep records, so everybody started
using that one. Then using VPN is something that gives me information about
how the game is played, like whether the people who run VPNs are vulnerable to
these legal mechanisms that might be used by grouped institutions or
legitimate institutions. It just helps make it clear in my head how the game
works.\end{quote}


19 interviewees would use it for work, out of which 16 used it in their
university or school. P21 used it, so their boss could not be able to see if
they were online at work.



\textbf{VPN vs. No VPN} 10 interviewees reported that VPN did not change their
Internet usage and online experience, 8 did not recognize many differences
between using and not using VPN and 7 did not see any difference while
browsing with or without VPN. On the other hand, 10 interviewees saw one
important difference: VPN allowed them to access content they could not access
before. Moreover, 8 believed that VPN makes their online experience more
secure but 2 believed that VPN is actually more vulnerable and 3 admitted that
they left their guard down when using VPN. Also, 2 interviewees found that
Internet connection is slower while using VPN.

Survey respondents reported that content-related reasons were most important
in driving them to use VPN. 66\% (230/350) of respondents reported access to
institutional materials, such as those from the university library, as an
important factor; 48\% (168/350) reported circumvention of Internet censorship
as a reason. Privacy (36\%, 127/350) and security (30\%, 105/350) were also
significant factors, with 27\% (93/350) of respondents selecting both and 40\%
(139/350) selecting at least one of the two. Of these 139 respondents, the
majority were concerned about protecting themselves from companies (73\%),
websites (70\%), the government (66\%), and hackers (65\%).

Students majoring in Engineering were more likely to use VPN to protect
privacy and security. Students who regard privacy and security as important
motivations for using VPNs tended to be more concerned about online data
collection, particularly by institutions and commercial entities. These
students tended to think that their actions, including their online activities
and messages, were being collected, and took additional efforts to secure
their online privacy and security with other tools (R=.34, R=.33). Privacy and
security motivations for using VPNs were likely to be linked (R=.71). Students
using institutional or free VPNs were more likely to use VPNs for content
access, whereas users of paid VPNs tended to use them for privacy and security
reasons. Students using VPNs for content access were less likely to list
privacy and security as motivations for using VPNs. Students who used VPNs for
privacy and security reasons also tended to use them more frequently,and were
more likely to have learned how VPNs work from their provider or online
research. 

We observe that students who use VPN for privacy or security reasons tend to
use it for the other as well. They are generally more conscious of their
online activities and concerned about data collection, particularly by
institutions such as the government. These students generally have higher VPN
knowledge, trust paid VPNs more, and were more diligent about VPN usage; they
also tended to learn how their VPN works from their providers or through
online research.


\subsubsection{Problems, Liked Qualities, and Disliked Qualities} We asked our
interview participants what kind of problems they encountered while using VPN.
\textbf{Accessing content} 7 interviewees said that VPN would disconnect, 6
that it was slow, another 6 that they could not access the content they wanted
because some websites knew they were using VPN. 

%Check if P08 in dedoose in this 6, ale Usage by others

For example P08 was annoyed by the fact that they need to switch off the VPN
while using Netflix, otherwise platform would not allow to access the content

\begin{quote}So, the only use cases I've seen mentioned in Germany was to, for
example, when Netflix wasn't available in Germany, people would try to use-
(\dots)It is [available] now, but in the beginning, things like Netflix were
only available in the US, so people would try to use it and then they needed a
DNS unblocker or a VPN to use Netflix. I don't think that works anymore. One
of the reasons that I'm not using VPNs as much anymore is that a lot of these
services just block the IP ranges of these data centers, so whenever I want to
watch something on Netflix, I have to turn off the VPN, which is annoying.
(\dots) I don't like it. I understand why they are doing, basically being
pressured by \dots I think this whole idea that they buy content for specific
regions and then content that's region-blocked is stupid, but that's the world
we live in and this how their contracts and everything are set up, so they
have to do something to prevent people from regions to access that content. I
mean, for me, it's just annoying, because then I have to switch off the VPN.
\end{quote}


4 interviewees mentioned that installing VPN was problematic, because it was a
complicated process (1). 16 reported that they had no problems at all. 

\textbf{Checking if VPN is working} We also wanted to know how our
interviewees check whether their VPN is actually working while they are using
it. 10 interviewees replied that they get information on their screen that VPN
is connected, 9 that since they use VPN for websites that otherwise are
blocked, having an access means that they are connected. 6 interviewees
actually checked their IP address to make sure VPN is working, 3 looked at VPN
icon and 3 said that they never check. 
We also surveyed respondents on what methods they used to verify that their
VPN was working. “Passive” methods of verification were most frequently
chosen, including viewing alerts from the VPN (41\%, 144/350), checking that
normally restricted content was accessible (48\%, 127/350), and viewing the
VPN’s tray or taskbar icon (45\%, 159/350).

Survey participants were asked to report what issues they experienced with
their VPNs. 41\% (144/350) reported issues with stability, and 35\% (121/350)
reported issues with content access. Figure 14 shows a detailed breakdown of
the issues students reported facing. Notably, 8 of the 10 “Other” responses
mentioned speed as an issue, a figure that would likely be much higher if
speed were presented as an option.

The types of issues students faced was not correlated with their level of
knowledge with VPN, or with their effort in verifying that their VPN was
working. Notably, students who experienced issues with ease of use including
stability, complicated installation processes, and difficulty in understanding
were more likely to have stopped using VPN.

We asked respondents to report, in short-answer form, what they liked and
disliked about their VPNs. We categorized these text responses by topic based
on their contents; some responses contained multiple topics. The ability to
access restricted content was by far the most commonly “liked” quality of
students’ VPNs (63\%, 221/350). Other qualities, including security, privacy,
interface, and transparency received far fewer mentions. Respondents indicated
that they prefer VPNs based on cost, speed, ease of use, privacy, and
security, yet they do not perceive the VPNs that they use as having those
positive qualities.

There was little consensus on what respondents did not like about their VPNs.
After speed and stability, respondents most frequently disliked the interfaces
(20\%, 69/350) of their VPNs. Although respondents indicated that privacy and
security are important factors for VPN users, they are not mentioned
substantially as either liked or disliked qualities of respondents’ VPNs. This
possibly suggests that students do not care as much about privacy and security
in their VPNs as they indicated earlier. Only 9\% of respondents reported
having nothing they disliked about their VPNs.



\textbf{Possible improvements} When asked what kind of improvements our
interviewees would like to see, half of them reported more transparency from
the VPN provider side (16) and 14 proposed more education effort for users to
know how VPN works (11), reasons why they should use VPN (3) and about online
security overall (2), since there are many people without sufficient
technological background to know it (7). 

\begin{quote}So it would be nice if the VPN providers themselves had more of
    an analysis of what kinds of tracking is generally being done to you if
    you are say in the US or if you're say in Europe.It would be nice if the
    VPNs themselves made a better case for using them. Like if you just go to
    Private Internet Access's website, it's kind of preaching to the choir,
    you know? It says, "Browse anonymously. Keep your IP address cloaked.
    Defend yourself from data monitoring," whatever. But it'd be nice to have
    like very concrete examples of, you know, "In the US, you could get in
    trouble for doing this, that, or the other on the Internet, and it becomes
    impossible to get in trouble for that if you use this." (P02)\end{quote}


Also, 5 interviewees wanted VPN providers to communicate what they do with
their logs. Furthermore, 8 interviewees would like VPN to be more accessible
and 5 more user-friendly. 

P13 encountered problems while trying to set up VPN with their friend:
\begin{quote}There were instructions online on how to download it. But they
don't have a very good system. \end{quote}

Moreover, P13 explained how he would like to change security and access
permission around institutional VPN: \begin{quote}Okay. I just would have more
    security around it. It just doesn't feel secure to me. It doesn't ... it's
    too easy to access, too easy for me to get into pretty much anything I
    wanted here, which is good for my work for bad if I wasn't doing work.
    Let's say I had a vindictive student who then could go in and delete all
    our files or put some kind of virus in the computer in the lab. There's no
    protections against that. They have full access just as much as we do. \end{quote}



