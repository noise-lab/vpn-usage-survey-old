\subsection{Results}\label{sec:results}

\subsubsection{Background}

For 11 of our participants, accessing block content was the priority when
using VPN. For 9, speed was a key component as well. Six declared that price
is an important factor when choosing VPN; including 2 that did not want to use
free one, being afraid that it could be more vulnerable option, and 4 that
would always choose a cheaper option. Four participants stated that it was
important for them that VPN is secure and/or they are anonymous while using
VPN. For one of these participants it was important where VPN company is
located so it does not have any ties with US government.  Moreover, 4
participants did not want to have any advertisements shown while using VPN. 

\subsubsection{Ratings}

\begin{table}[h!]
\centering
\begin{tabular}{|c c c c|} 
 \hline
P\#/Rating & \#1 & \#2 & \#3 \\
\hline
P1 & 4 & 4 & 4\\
P2 & 3 & 3 & 1\\
P3 & 3 & 3 & 1\\
P4 & 4 & 4 & 3\\
P5 & 4 & 4 & 5\\
P6 & 3 & 3 & 1,5\\
P7 & 4 & 4 & 2,5\\
P8 & 3 & 2 & 1\\
P9 & 2 & 4 & N/A\\
P10 & 5 & 4,5 & 2,5\\
P11 & 3,5 & 3,5 & 3,5\\
P12 & 2,5 & 1 & 1 \\
 \hline
\end{tabular}
\caption{Ratings of the VPN after each condition. Rating \#1 = after control condition, Rating \#2 = after intervention with the tool, Rating \#3 = after giving all the additional information about tracking, N/A = no rating score.}
\label{table:2}
\end{table}
As stated in the method section, we asked our participants to assess their
willingness to use the VPN on the scale from 1 to 5 (1 being the less likely
and 5 very likely) three times: 1.	after completing a set of tasks using the
VPN, 2.	after completing another set of task using the VPN and our tool, as
well as 3.	after explaining how trackers get data about their online activity
while using VPN and giving additional information about  how our tool and VPN
work.  In the first measurement, participants scored a median of 3.25, second
3.75 and third 2.5. Nine participants did not change their scoring between the
first and second measurements, with two that did not change their scoring at
all, between all 3 measurements. One participant was more willing to use the
VPN after the second measurement. Also, one participant was more willing to
use the VPN we presented after 3rd measurement.  In the next paragraphs we are
trying to look deeper in their motives for such scoring.  
    


\subsubsection{Setting up the VPN}

While looking around the website, 8 participants admitted that they did not
trust all VPN provider's promises; more specifically, they did not believe
that VPN provides unlimited bandwidth and/or that it ensures anonymity.
Moreover, 7 participants were confused, since they did not understand content
on the website (5) and were not sure how VPN works (4). At least one person
felt uncertain about VPN being legal. Five participants were interested in
privacy aspect that VPN provides, 3 were curious about pricing and another 3
were looking for information about the speed. At least one person wanted to
know what kind of locations user can connect to. One participant was
particularly interested in information on VPN's website, that VPN can be used
to save money on hotel and rental car costs.  Also, one participant said that
they would appreciate a comparison between different VPNs, so users could make
informed decision when choosing the provider. One of our participants admitted
that they had used this particular VPN provider in the past but stopped using
it after couple of months, because they did not like how VPN's icon looked
like. 

While setting up the VPN, participants were mostly interested in different
locations they could choose from (9). At least 3 participants did not like the
fact that free version of the VPN had a small number of locations to choose
from (2) and no US location, in case someone wanted to watch Netflix from
abroad (1).


\subsubsection{First rating - Control condition}

After setting up the VPN, we asked our participants to complete a set of tasks
related to online browsing, so they could assess the VPN and their willingness
to use it in the future. Participants scored a median of 3.25 on the scale
from 1 to 5 on how likely they would be using this particular VPN. After
control condition, 5 participants believed that they did not need VPN at all.
Also, 5 participants believed that using VPN made their online experience too
slow. Again, 2 participants complained that it did not have enough locations
to choose from. Overall, participants felt quite neutral towards the VPN.


\subsubsection{Second rating - VPN Audit condition}

Next, we asked participants to complete similar set of tasks but we also asked
them to use the tool that we had developed. This time, the participants'
median score was  3.75, so slightly higher than the first measurement. Eight
participants' ratings were the same as the first one, as they did not see any
difference in using the VPN. Three participants rated the VPN lower than the
first time: two changed their mind because of the information they saw on the
extension and the fact that data was being sent to trackers and one because
they felt that accessing websites was slower than during the first set of
tasks.  On the other hand, P9 rated VPN higher than after the Control
condition, because it seemed easy to use and did not slow down the connection.
Furthermore, two participants, P9 and P7, who rated VPN the same way for both
measurements, believed that trackers are not able to see their real IP
address, only the VPN server’s. Thus they believed they were not tracked and
the tool showed them only the trackers that are embedded on websites they
visited. 

\subsubsection{Third rating - Additional information} 

After completing both sets of tasks, participants were asked to read articles
about how tracking works while using VPN. Then, they were asked again to rate
their willingness to use this VPN in the future. Participants scored a median
of 2.5, so lower than in previous measurements. Six participants' scores were
lower than first two measurements, because of the information they read and
clarification they received. Two participants, P1 and P11, remained with the
same scoring as they were not concerned about their privacy. As P1 explained,
the data would be sent to trackers regardless of them using the VPN or not.
For P11 it was more important whether VPN is easy to use, cheap and allows
them to access blocked content. On the other hand, P5 gave even better rating
after getting all information about trackers. As P5 explained:

\begin{quote}A VPN, even if it allows information to these two trackers, it
still takes care of your other information. It doesn't allow other information
that are not sent to this tracker, and it encrypts that data.\end{quote} Five
of our participants stated that even though they know VPN sends the data to
trackers, at least it allows them to access the content they want.   


\subsubsection{Follow-up questions} 

Seven of our participants were not sure how VPN Audit works or what
information it was presenting. For example, 2 participants, P11 and P2,
thought that VPN Audit is the tracker that collects data. Eight participants
explained that it shows hits sent to trackers. 

When asked if the tool changed their perspective on how VPNs work, 7
participants agreed as they had believed that there should be no records of
their data (4). On the other hand, 4 participants did not feel like it changed
their view on how VPN works, since they were not concern about privacy of
their data.

Notably, 7 participants felt deceived when they learned about the fact that
VPNs send information to trackers but they do not mention it explicitly on
their websites. Six of our participants would be more willing to use the VPN
if the company was transparent about the fact, that using free version of VPN
results in sending data to advertisers. Five participants would not use such
VPN, but they would appreciate such information and trust the VPN company
more. 


