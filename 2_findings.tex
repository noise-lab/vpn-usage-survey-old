\subsection{Results}\label{sec:results}

\begin{table}[h!]
\centering
\begin{tabular}{|c c c c|} 
 \hline
P\#/Rating & \#1 & \#2 & \#3 \\
\hline
P1 & 4 & 4 & 4\\
P2 & 3 & 3 & 1\\
P3 & 3 & 3 & 1\\
P4 & 4 & 4 & 3\\
P5 & 4 & 4 & 5\\
P6 & 3 & 3 & 1,5\\
P7 & 4 & 4 & 2,5\\
P8 & 3 & 2 & 1\\
P9 & 2 & 4 & N/A\\
P10 & 5 & 4,5 & 2,5\\
P11 & 3,5 & 3,5 & 3,5\\
P12 & 2,5 & 1 & 1 \\
 \hline
\end{tabular}
\caption{Ratings of the VPN after each condition. Rating \#1 = after control condition, Rating \#2 = after intervention with the tool, Rating \#3 = after giving all the additional information about tracking, N/A = no rating score.}
\label{table:2}
\end{table}

\paragraph{Ratings}

As stated in \ref{sec:experiment_method}, we asked our participants to assess their
willingness to Hotspot Shield on the scale from 1 to 5 (1 being the less likely
and 5 very likely) three times: 1) after completing an initial set of tasks using the
VPN, 2) after completing another set of tasks using Hotspot Shield and \tool, and
3) after explaining how certain trackers get data about the participants' online activity while using the VPN, and giving additional information about how our tool and the VPN work. The participants gave a median rating of 3.25, 3.75, and 2.5 for each measurement, respectively.

Nine participants did not change their ratings between the first and second measurements, and two participants did not change their ratings at
all between the three measurements. Interestingly, one participant was more willing to Hotspot Shield after we showed them \tool. \mc{say why} Another participant was more willing to
use Hotspot Shield after we explained the VPN's privacy practices and showed them articles about what data certain trackers can collect through the VPN. In the following paragraphs, we explore the participants' reasoning the ratings they gave.

\paragraph{\tool mostly did not change views on privacy}

\mc{some of the rating results are repeated here and in the first paragraph - should remove repetition}
After setting up Hotspot Shield, we asked our participants to complete a set of tasks without being shown \tool. Participants gave a median rating of 3.25 on a scale
from 1 to 5 on how likely they would use Hotspot Shield. However, five participants believed that they did not need the VPN at all for the tasks we gave them, i.e. looking up information about movies, sexual health, and/or mental health. \AH{A quote from one of these 5 participants should go here.}

Interestingly, after completing a second set of tasks with \tool, the participants' median rating was 3.75, i.e. higher than the first measurement. Eight participants gave the same rating as the first measurement. P9 rated Hotspot Shield higher than the first measurement because the VPN seemed easy to use and did not slow down the connection. Nonetheless, three participants rated the VPN lower than the
first measurement. Two participants changed their mind because of the information they saw on the extension and the fact that data was being sent to trackers, and one because
they felt that accessing websites was slower than during the first set of
tasks.

When asked if \tool changed their perspective of how VPNs work, seven
participants agreed because they had believed VPNs do not keep records of their data. \AH{Agnieszka, can you confirm these numbers?} However, four participants did not feel like \tool changed their opinion of Hotspot Shield because they were not concerned about privacy. \AH{A quote from of of these four participants should go here}.

\paragraph{Articles changed views on privacy}

After completing the first two sets of tasks, participants were finally asked to read articles
about the privacy practices of Hotspot Shield. We also explained what information \tool was showing, and we explained how Hotspot Shield allowed certain trackers to see the participants' browsing data. Then, they were asked to rate
their willingness to use Hotspot Shield in the future. Participants gave a median rating
of 2.5, i.e. lower than the first two measurements. Six participants' ratings were
lower because of the information they received from the articles.

Notably, seven participants felt deceived when they learned about the fact that
VPNs send information to trackers but they do not mention it explicitly on
their websites. Six of our participants would be more willing to use Hotspot Shield
if the company was transparent about the fact that using free version of the VPN
results in browsing data being sent to trackers. \AH{A quote from one of the six participants should go here.} Five participants would still not use Hotspot Shield if they explicitly disclosed such behavior, but they would trust the company more.

On the other hand, two participants--P1 and P11--gave the same ratings because they were not concerned about their privacy. As P1 explained, their browsing history would be sent to trackers regardless if they used Hotspot Shield or not.
For P11, it was more important whether VPNs are easy to use, cheap, and allow
them to access blocked content. \AH{A quote for this participant should go here.}

% On the other hand, P5 gave a better rating after reading the articles. As P5 explained:

% \begin{quote}A VPN, even if it allows information to these two trackers, it
% still takes care of your other information. It doesn't allow other information
% that are not sent to this tracker, and it encrypts that data.\end{quote} Five
% of our participants stated that even though they know VPN sends the data to
% trackers, at least it allows them to access the content they want.

% \paragraph{Speed and location are important}
% 
% While setting Hotspot Shield in Google Chrome, 9 participants stated that they were mostly interested in learning about the locations for the VPN that they could choose from. At least 3 participants did not like the fact that the free version of the VPN had a small number of locations to choose from. Furthermore, 2 participants did not like that the free version offered no U.S. location, and 1 participant indicated that they want a U.S. location to watch Netflix from abroad. Three participants were curious about pricing, and another 3 participants were looking for information about the speed of Hotspot Shield.

% After completing the first set of tasks, 5 participants believed that using Hotspot Shield made their online experience too slow. Again, 2 participants complained that the VPN did not have enough locations to choose from. Overall, the participants felt quite neutral towards the VPN.
% Even after completing another set of tasks with \tool, eight participants' rating remained the same. \AH{A quote would be good here.}

\paragraph{Participants did not understand how VPNs work}

While looking around the website for Hotspot Shield, eight participants admitted that they did not trust all of the VPN's promises. In particular, they did not believe
that Hotspot Shield provides unlimited bandwidth and/or that it ensures anonymity.
Moreover, seven participants were confused, since they did not understand the content
on the website (5) and were not sure how the VPN works (4). At least one person
felt uncertain about Hotspot Shield being legal. One participant also said that
they would appreciate a comparison between different VPNs so that they could make an
informed decision.

After completing both sets of tasks, P9 and P7--who rated Hotspot Shield the same way for the first two measurements--believed that trackers are not able to see their real IP
addresses. Thus, they believed they were not being tracked while using Hotspot Shield, and that the tool just listed trackers that are embedded on websites they visit. \AH{A quote would be great here.}

Seven of our participants were also not sure how \tool works or what
information it was presenting. For example, two participants, P11 and P2,
thought that \tool is tracker that collects data. Eight participants
explained that it shows "hits" sent to trackers, but they did not elaborate on what this means. \AH{Agnieszka, can you confirm these numbers?}
