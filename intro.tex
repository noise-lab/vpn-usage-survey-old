\section{Introduction}

Virtual Private Networks (VPNs) are one of the many tools that Internet users can utilize to
protect against online privacy and security risks. This is increasingly important today as data breaches and data privacy scandals occur frequently \cite{brandom_4, grothaus_21, meyer_22, reilly_23, perlroth_24}. VPNs work by creating a
secure, private connection ("tunnel") through the provider's server through
which the user can safely access a destination server \cite{vaughan-nichols_25}.   In 2016, the
worldwide market for VPNs was 15.64 billion USD, and is forecasted to reach
double that by 2021 \cite{vpnmarket_3}. Though VPNs are
increasing in popularity, their usage is far from mainstream; only 16\% of
Internet users in North America reported using VPN at some point \cite{mander_27}. In
addition, most Internet users lack a thorough understanding of online data
collection and the risks that it poses to privacy and security \cite{shirazi_9}. In the US,
education curriculum has not evolved to accommodate rapidly changing
technology \cite{levin_29}. As Internet technology becomes integral to our daily lives, data privacy risks
will continue to be highlighted. Discourse about these risks has increased in
recent years; . Given this, we expect a high degree of disconnect between
user perceptions of VPNs and the services actually provided by VPNs. It is
unknown how users actually perceive VPNs, and what their explicit purposes for
using them are. In this paper, we focus on college students and study the
following research questions:\\
RQ1: What are people’s mental models of VPNs?\\
RQ2: How and why do people use VPNs?\\
RQ3: Do people’s attitudes about VPNs change when you show them VPN privacy protection mechanisms and data leakages?\\

To answer our research questions, we employed a mixed-methods approach. We conducted both interviews and a survey, in order to learn more about VPN users' expectations and needs related to VPNs. We also conducted a focus group study, where we investigated people's reactions on information about how their data is being processed, while using particular VPN.

We found that students generally had weak VPN knowledge, and did not fully
understand how VPNs worked. They sporadically used VPN primarily for content
access and regarded privacy and security reasons as secondary, and they
considered cost, ease of use, and speed as the most important elements of a
VPN. Most college students use the VPN provided by their school, though a
substantial number use commercial VPNs. Students are dissatisfied with the
speed, stability, and interfaces of their VPNs.

Our results have implications for educators, policymakers, and VPN providers
and designers. Both educators and VPN providers should emphasize the
importance of online privacy and security. VPN providers should further
educate its users on VPN technology, and also run pricing analyses.
Policymakers should require VPN providers to be transparent about their own
privacy violations. VPN designers should improve the installation, login, and
reconnecting processes of VPNs as well as create designs that better
communicate what VPNs are actively doing.

