%
% The first command in your LaTeX source must be the \documentclass command.
\documentclass[acmlarge,screen]{acmart}

%
% defining the \BibTeX command - from Oren Patashnik's original BibTeX documentation.
\def\BibTeX{{\rm B\kern-.05em{\sc i\kern-.025em b}\kern-.08emT\kern-.1667em\lower.7ex\hbox{E}\kern-.125emX}}
    
% Rights management information. 
% This information is sent to you when you complete the rights form.
% These commands have SAMPLE values in them; it is your responsibility as an author to replace
% the commands and values with those provided to you when you complete the rights form.
%
% These commands are for a PROCEEDINGS abstract or paper.
\copyrightyear{2018}
\acmYear{2018}
\setcopyright{acmlicensed}
\acmConference[Woodstock '18]{Woodstock '18: ACM Symposium on Neural Gaze Detection}{June 03--05, 2018}{Woodstock, NY}
\acmBooktitle{Woodstock '18: ACM Symposium on Neural Gaze Detection, June 03--05, 2018, Woodstock, NY}
\acmPrice{15.00}
\acmDOI{10.1145/1122445.1122456}
\acmISBN{978-1-4503-9999-9/18/06}

%
% These commands are for a JOURNAL article.
%\setcopyright{acmcopyright}
%\acmJournal{TOG}
%\acmYear{2018}\acmVolume{37}\acmNumber{4}\acmArticle{111}\acmMonth{8}
%\acmDOI{10.1145/1122445.1122456}

%
% Submission ID. 
% Use this when submitting an article to a sponsored event. You'll receive a unique submission ID from the organizers
% of the event, and this ID should be used as the parameter to this command.
%\acmSubmissionID{123-A56-BU3}

%
% The majority of ACM publications use numbered citations and references. If you are preparing content for an event
% sponsored by ACM SIGGRAPH, you must use the "author year" style of citations and references. Uncommenting
% the next command will enable that style.
%\citestyle{acmauthoryear}

%
% end of the preamble, start of the body of the document source.
\begin{document}

%
% The "title" command has an optional parameter, allowing the author to define a "short title" to be used in page headers.
\title{The Name of the Title is Hope}

%
% The "author" command and its associated commands are used to define the authors and their affiliations.
% Of note is the shared affiliation of the first two authors, and the "authornote" and "authornotemark" commands
% used to denote shared contribution to the research.
\author{Ben Trovato}
\authornote{Both authors contributed equally to this research.}
\email{trovato@corporation.com}
\orcid{1234-5678-9012}
\author{G.K.M. Tobin}
\authornotemark[1]
\email{webmaster@marysville-ohio.com}
\affiliation{%
  \institution{Institute for Clarity in Documentation}
  \streetaddress{P.O. Box 1212}
  \city{Dublin}
  \state{Ohio}
  \postcode{43017-6221}
}

\author{Lars Th{\o}rv{\"a}ld}
\affiliation{%
  \institution{The Th{\o}rv{\"a}ld Group}
  \streetaddress{1 Th{\o}rv{\"a}ld Circle}
  \city{Hekla}
  \country{Iceland}}
\email{larst@affiliation.org}

\author{Valerie B\'eranger}
\affiliation{%
  \institution{Inria Paris-Rocquencourt}
  \city{Rocquencourt}
  \country{France}
}

\author{Aparna Patel}
\affiliation{%
 \institution{Rajiv Gandhi University}
 \streetaddress{Rono-Hills}
 \city{Doimukh}
 \state{Arunachal Pradesh}
 \country{India}}
 
\author{Huifen Chan}
\affiliation{%
  \institution{Tsinghua University}
  \streetaddress{30 Shuangqing Rd}
  \city{Haidian Qu}
  \state{Beijing Shi}
  \country{China}}

\author{Charles Palmer}
\affiliation{%
  \institution{Palmer Research Laboratories}
  \streetaddress{8600 Datapoint Drive}
  \city{San Antonio}
  \state{Texas}
  \postcode{78229}}
\email{cpalmer@prl.com}

\author{John Smith}
\affiliation{\institution{The Th{\o}rv{\"a}ld Group}}
\email{jsmith@affiliation.org}

\author{Julius P. Kumquat}
\affiliation{\institution{The Kumquat Consortium}}
\email{jpkumquat@consortium.net}

%
% By default, the full list of authors will be used in the page headers. Often, this list is too long, and will overlap
% other information printed in the page headers. This command allows the author to define a more concise list
% of authors' names for this purpose.
\renewcommand{\shortauthors}{Trovato and Tobin, et al.}

%
% The abstract is a short summary of the work to be presented in the article.
\begin{abstract}
A clear and well-documented \LaTeX\ document is presented as an article formatted for publication by ACM in 
a conference proceedings or journal publication. Based on the ``acmart'' document class, this article presents
and explains many of the common variations, as well as many of the formatting elements
an author may use in the preparation of the documentation of their work.
\end{abstract}

%
% The code below is generated by the tool at http://dl.acm.org/ccs.cfm.
% Please copy and paste the code instead of the example below.
%
\begin{CCSXML}
<ccs2012>
 <concept>
  <concept_id>10010520.10010553.10010562</concept_id>
  <concept_desc>Computer systems organization~Embedded systems</concept_desc>
  <concept_significance>500</concept_significance>
 </concept>
 <concept>
  <concept_id>10010520.10010575.10010755</concept_id>
  <concept_desc>Computer systems organization~Redundancy</concept_desc>
  <concept_significance>300</concept_significance>
 </concept>
 <concept>
  <concept_id>10010520.10010553.10010554</concept_id>
  <concept_desc>Computer systems organization~Robotics</concept_desc>
  <concept_significance>100</concept_significance>
 </concept>
 <concept>
  <concept_id>10003033.10003083.10003095</concept_id>
  <concept_desc>Networks~Network reliability</concept_desc>
  <concept_significance>100</concept_significance>
 </concept>
</ccs2012>
\end{CCSXML}

\ccsdesc[500]{Computer systems organization~Embedded systems}
\ccsdesc[300]{Computer systems organization~Redundancy}
\ccsdesc{Computer systems organization~Robotics}
\ccsdesc[100]{Networks~Network reliability}

%
% Keywords. The author(s) should pick words that accurately describe the work being
% presented. Separate the keywords with commas.
\keywords{datasets, neural networks, gaze detection, text tagging}

%
% A "teaser" image appears between the author and affiliation information and the body 
% of the document, and typically spans the page. 
%%\begin{teaserfigure}
%%  \includegraphics[width=\textwidth]{sampleteaser}
%%  \caption{Seattle Mariners at Spring Training, 2010.}
%%  \Description{Enjoying the baseball game from the third-base seats. Ichiro Suzuki preparing to bat.}
%%  \label{fig:teaser}
%%\end{teaserfigure}

%
% This command processes the author and affiliation and title information and builds
% the first part of the formatted document.
\maketitle

% Sections
\section{Introduction}

As Internet technology becomes integral to our daily lives, data privacy risks will continue to be highlighted. Discourse about these risks has increased in recent years; data breaches occur frequently, and media coverage of data privacy scandals has become mainstream [4, 21, 22, 23, 24]. VPNs, or Virtual Private Networks, are one of the many tools that Internet users can utilize to protect against online privacy and security risks. VPNs work by creating a secure, private connection ("tunnel") through the provider's server through which the user can safely access a destination server [25]. Though VPNs are increasing in popularity, their usage is far from mainstream; only 16\% of Internet users in North America reported using VPN at some point [27]. In addition, most Internet users lack a thorough understanding of online data collection and the risks that it poses to privacy and security [9]. In the US, education curriculum has not evolved to accommodate rapidly changing technology [29]. Given this, we expect a high degree of disconnect between user perceptions of VPNs and the services actually provided by VPNs. It is unknown how users actually perceive VPNs, and what their explicit purposes for using them are. In this paper, we focus on college students and study the following research questions:

TODO

To answer our research questions, we employed a mixed-methods approach. TODO

We found that students generally had weak VPN knowledge, and did not fully understand how VPNs worked. They sporadically used VPN primarily for content access and regarded privacy and security reasons as secondary, and they considered cost, ease of use, and speed as the most important elements of a VPN. Most college students use the VPN provided by their school, though a substantial number use commercial VPNs. Students are dissatisfied with the speed, stability, and interfaces of their VPNs.

Our results have implications for educators, policymakers, and VPN providers and designers. Both educators and VPN providers should emphasize the importance of online privacy and security. VPN providers should further educate its users on VPN technology, and also run pricing analyses. Policymakers should require VPN providers to be transparent about their own privacy violations. VPN designers should improve the installation, login, and reconnecting processes of VPNs as well as create designs that better communicate what VPNs are actively doing.


\section{Related Work}\label{Related Work}

\section{Interview Method}\label{Method}

\subsection{Recruitment}

\subsection{Participants}

\subsection{Interviews}

\subsection{Analysis}

\section{Limitations}
\label{sec:limitations}


\subsection{Our Study Focuses on External Actors, Not In-Home Privacy Threats}

\section{Results}\label{Findings}
The main themes that emerged from our interviews are as follows: 

\subsection{Knowledge about tracking and privacy}
\label{sec:findings-conv}
%\subsubsection{Censorship}
%At least 18 interviewees defined censorship on the Internet as removal of content on online platforms by higher authorities, such as big companies (6 interviewees) and government (3 interviewees). P10 made distinction between two different types of censorship, governmental one and in private sector:
%\begin{quote}You have censorship as a form of official government policy, and you see that in countries like China with The Great China Firewall and then I think you have, oddly enough, a private censorship. When you have sites that have become defacto public spaces, as Reddit or Facebook, although they are ostensibly private, they're owned by a private company, hence they do have the ability to censor however they'd like, it's sort of turned into a public space in the sense that the greater public meet the [inaudible] political opinion. However, those private spaces are often censored. (…) I think the official government censorship is terrible. (…)Well, I believe very strongly in the rights that, at least as Americans, we're afforded under the First Amendment, being able to freely say and think what we'd like, and freely associate with who we'd like to. Obviously, unless that speech crosses into violence, or rather encouraging violence, or I guess even stronger than encouraging. I can't think of the right word right now, but really compelling violence. I think unless it crosses over into that, you should be able to say and associate with whomever you'd like. Then as far as the private censorship, I mean, I think there's a case to be made for these sorts of defacto community sites historically treated a bit more how traditional media is treated whereby they are compelled to give airtime to both sides of an issue.\end{quote}


 %At least 17 defined it as limiting access to what people see or search and 8 noted that censorship depends on your geolocation. Moreover, 17 interviewees believed that censorship is worrisome, 4 of which thought that censorship is bad for democracy. At least 18 interviewees considered “free speech” as important. Three interviewees declared that in their opinion there should be no restrictions on the Internet but another 23 argued that there should be some regulations applied, which would filter harmful or dangerous content. 
 %\begin{quote}Sometimes I think it's necessary, sometimes there are things that are on the internet that shouldn't be on the internet for certain audiences, for example; children. I honestly think there should be a lot of censorship for children because there are some topics that just simply children are not ready for or shouldn't even be introduced to when they're young it's a matter of, in a way it's keeping the innocence because somethings on the internet are just very violent and yeah, again maybe censoring things for specific ages. If they're under the age of 18 because they don't have consent or anything, is necessary. I also think censorship is necessary in there's like eight, like for example I read an article recently where Facebook deleted 600 or X amount of accounts because of these Facebook groups were actually banding together and just like really, really hate crimes. They were planned hate crimes and they were advocating for more hate in other countries. (\dots) I'm all for free of speech, but when it comes to like imposing life threatening things then censorship might be necessary to kinda bridge the gap between like peoples safety and like freedom of speech. (P23)\end{quote}
%Moreover, seven of our participants believed that censorship should be left to the individual itself. 


%P10 defined censorship as an unequal treatment of different entities:

 %\begin{quote}As I understand it with American news media, if they give a certain amount of time to, say, the Republican candidate, then they need to give an equal amount of time to the Democratic candidate. I think there's something to be said for that, especially if you're trying to position yourself as the front page to the Internet as, say, Reddit does, for example. Yeah, or if you're trying to position yourself, I think, as a fair, impartial kind of platform like Twitter, you can't really go about censoring one side of an argument just because it's uncomfortable or just because you don't agree with it.\end{quote}


%P15 raised a subject of money related to censorship:  
%\begin{quote}I get it. I understand why the rules are strict and I understand that usually paying for the services that you're using, but I also think it's really hard to make people not choose the easy way and take, get stuff for free, instead of paying for it, especially because some services are pretty expensive. Netflix is \$13 a month, which is not something cheap if you think about how much your paying for the whole year and compared to how much your watching it. Same with Spotify and other services that everyone uses, but \dots yeah. It's definitely better to pay for music or films than to get a fine of \$2,000 for doing that and even get a criminal record and the actual sense for criminal record for something like that.As I said, I understand where this is coming from, but I think it's really hard to \dots keep track of how many people are actually paying, how many people are finding ways to get away with it.\end{quote}


%At least 18 interviewees believed that censorship has political background while 11 that its reasons are more societal, for example 4 interviewees from the latter group pointed purpose of censorship to be forming’s someone’s opinion.
%P27 talked about the bias that censorship produced between different groups of people:

% \begin{quote}And then the other instances of censorship that I've seen are like all of the black economic pages that I follow. A lot of times their posts get removed and I think one of my favorite social media personalities, she had a magazine shoot and she was naked on the cover. But she's a black woman and she's also doesn't fit body standards of beauty and so she's on the bigger side. She's heavy set. And they took down her photo but Kim Kardashian, her photos never get taken down but she's always half naked on the internet and stuff like the Tomi Lahren fan page.Tomi Lahren is a conservative spokesperson and her posts never get censored when they're hateful and things like that and are very political. So I think a lot of the censorship is biased.(\dots)I don't think it's fair. I think it's biased. That's exactly what I think. I think if I [inaudible] posts remove and other people not having their posts removed, it gets unfair and I also think it's representative. It's representative of the values of the United States.Even when you type in, if you type in black teen on google, I'm not sure if it's like this anymore, but the fact that you can change it kind of worries me because when you do a Google image search, you expect it to be random. Type in black teen, there's pictures of delinquents and mug shots and drugs and money and stuff like that. When you type in white teen, you get all these happy white teens in school and things.So I think that censorship kind of confirms but also perpetuates some of the racial stereotypes we have about people.\end{quote}

\subsubsection{What is Tracking}

Twelve interviewees defined \textit{Online tracking} as creating data about an individual, and gathering data about individual's location (5). More than half (17) of interviewees said that it happens through user's search history and general online activity. Another 2 that it can be done through web camera but also through apps (1) or looking at key logger passwords (1). Also, for 12 interviewees  \textit{tracking} meant that someone could see what they are doing online and 6 believed that it happens each time you go online, while 3 stated that you are not able to hide.

\begin{quote}To me that would mean collecting data about a user in any service or any capacity while they're online or doing anything that involves being online. That could be something like keeping a lot of the messages you sent, even if you delete them or that could be something as simple as just tracking what times you sign on to different things. I think that also applies to things like, with our [Proxy] cards, every time we buzz into a building that's documented somewhere and that's documented online, so that kind of goes into online tracking as well. Anytime we're using an online system, information we recorded. (\dots) I think the way that I, totally uninformed way you think of it is that there are probably machines, some AI, some program that looks for certain keywords and a certain combination or indicators, histories. I'm sure there's some algorithm that people have developed. I don't think someone is necessarily actively tracking me but I'm sure there is some software that I, along with many other people are being tracked by and if something I were to say was flagged on that, then I think there would be an individual, or at least more attention to me. (P32)\end{quote}





\subsubsection{Comfort with Tracking}
For 4 interviewees tracking was frightening, and tracking made 2 interviewees' overall online experience uncomfortable. Moreover, 8 interviewees believed that tracking was something bad that should not have place. 
%P09 – write about tampons story 
Moerover, P25 did not feel like there was any good law in place that would make users' data safe, which led them to believe that companies have low standards of data processing and make users more vulnerable. 

On the other hand 7 interviewees interpreted it as something good, for example using it as a tool for crime investigation (2) or suggesting ads with clothing (2). 

\begin{quote} it [online tracking] could mean positive tracking. I guess Google tracks many things that I do and I'm okay with that and I'm aware so I don't do too personal searches or anything. It's just for personalized advertising or search results that could be beneficial. (P31) \end{quote}

And for example, P13 believed that tracking is more visible than censorship:

\begin{quote}I guess I don't see the censorship even though I know it's there. Actually, I see it more when I travel to other countries because I go to access the same webpages, and I can't get to them. But I think on the other side of things, I see tracking more than I see censorship. So I see that everything on the internet is very personalized suddenly to me, and that's only within the past maybe like six years that it got really that bad.


(\dots) when I first started using the internet a long time ago, you just put in some search and you got some random stuff. That was not related to you. Sometimes it was way off your search criteria. You could bounce around the internet. Nothing seemed to be related to you. Now you go online, and I see things that even [inaudible] follows me for a week. So if I looked at a specific mug online, and I'm seeing that mug on every single webpage I click on, even if those webpages have completely nothing to do with each other. (\dots) I think it's super weird. (\dots)I do not like it. (\dots) I think the ad tracking has gotten way out of proportion. It's nice to be able to search for things, and they remember my old searches. So it's more related to me. That's nice. But then if it's something that I'm looking for that really has nothing to do with my personal interests, I end up with a lot of search criteria that are just unrelated to what I'm looking for.\end{quote} 



\subsubsection{Who Tracks Data}
We asked our interviewees who they think track online. 18 interviewees believed that big companies, such as Facebook (4) and Google (3) track people online. On the other hand 15 interviewees pointed government as an institution that tracks, naming NSA (3), FBI (2), CIA (1) and IRS (1). Furthermore, 8 interviewees stated that every website tracks its users, 6 that it is done by advertising agencies or data companies. 4 interviewees said that ISP can track its users, another 4 that it is done by hackers and 2 that school/university tracks your online activity. 5 interviewees said that tracking is not done by anyone in particular.

27 interviewees admitted that they believed that they are being tracked. 13 of which believed they are tracked by big companies, such as Google (3) and another 13 that even though they are tracked, they are not targeted. 


\begin{quote}I think the way that I, totally uninformed way you think of it is that there are probably machines, some AI, some program that looks for certain keywords and a certain combination or indicators, histories. I'm sure there's some algorithm that people have developed. I don't think someone is necessarily actively tracking me but I'm sure there is some software that I, along with many other people are being tracked by and if something I were to say was flagged on that, then I think there would be an individual, or at least more attention to me.(P32)\end{quote}


5 interviewees believed that they are not tracked, as they do not do anything suspicious.
%Not sure yet how to describe this quote

\subsubsection{Reasons for Tracking}    

We also asked participants about reasons that these entities may have to track. 15 responded that tracking is needed to gather data about users' preferences, also 15 said that to gather data about our online behavior and 14 that it is needed for advertising. 19 believed that tracking is for financial interest. For example, P25 mentioned that tracking is an essential part of A/B testing. 

P23 associated online tracking with location services, that track users' geographical position. They also added:

\begin{quote} 
Essentially, I feel sometimes like, I don't know if you ever heard of the novel 1984, but it's just like George Orwell, like beautifully describes just being watched and how were under a lot of supervision or if you read this philosopher Foucault, who talks about this Panopticon, that's were constantly being just supervised by a higher power and this power being like the people who controlling these, this data, cause essentially it is data, like where you are, it's stored where your cache. And also like the things you like and the sites you visit it's all start in your cashe, in your history and a site has the permission to use that cash in order to market things a certain way for you, for example if you. This is like super intense, but like, I don't know, I googled a pair of shoes right and then like two seconds, maybe like five minutes later I was just scrolling through my Instagram feed and I see those same shoes being on an ad in Instagram, that's like annoying. I mean I know you want me to buy your shit because it's capitalism and all, but I don't need to know that you're, I don't need to be like primed into like I want to buy that thing. That's essentially what I mean by tracking.\end{quote}

On the other hand, P07 described tracking as repercussion of Internet speed debate. The internet providers would track users to evaluate the number of people that visit each website. The bigger company with more viewers, would get faster internet to people, so Internet providers would control speed of the Internet for different companies. 


\subsection{Knowledge about VPNs / Mental model}

\subsubsection{What is VPN and how it works}

When asked what is VPN in their opinion, 14 interviewees defined VPN as an Internet activity routed through third party machines and 14 as changing IP address. 
\begin{quote}It's sort of a middle man. So instead of you actually downloading the file from someplace where somebody might be looking at you downloading it, they download it for you and then they send it to your computer. So it figures that they downloaded it and not you.( P18)\end{quote}


But for example, P20 was not sure how IP address was changed:
\begin{quote}I think that depending on where you geographically use the internet, you have an IP address, and a VPN will be able to know the IP address of a different place and make it look like you're using the internet from that place instead. That's just speaking on, like when I had the VPN for watching Netflix, I could have it at literally any country in the world would say, "Oh, I'm watching from Rwanda." It would have known the IP address used that, so ten \dots Yeah, so that's how I would understand it.
As for how their making it look like you're using that IP address rather than your actual IP address, I actually don't know.\end{quote}


According to 7 interviewees you are getting into another network while using VPN and 6 interviewees describe this network as secure and 4 as private. 
%(Only) 
Three described VPN as an extra level of safety. Furthermore, 13 interviewees defined VPN usage as accessing blocked things and 5 that using VPN is bypassing rules and laws. 10 interviewees believed that VPN is masking their identity and another 10 believed that it is reducing others ability to track you.  6 of our interviewees described VPN as a high-tech thing and 4 that it is not really known. 12 admitted that they did not have an extensive knowledge about VPNs.

At least 4 participants defined VPN as a blackbox and described people who are interested in VPN as technology-savvy. As P27 shared their mental model of people who use VPN: 
\begin{quote}I don't know if this is helpful but when I think of VPN I feel like you, I mean I do computer coding but I feel like you have to be a super nerd to kind of, for lack of a better term, to kind of understand what all that is and to be able to manipulate it in any way\end{quote}


As well as P18:
\begin{quote}The usage of this VPN sort of hinges on two things, the desire to obtain copyrighted material for free and also the knowledge of the existence of VPN's. Those are two I think pretty big bottlenecks that sort of limit this sort of information to sort of tech nerds, if that makes sense.\end{quote}


And P25 descibed VPN as a blackbox but also as a statement:
\begin{quote}Everything in computer science is a black box. I don't know if the VPN is working, I don't know is somewhere along the line \dots (\dots)
I think that it is a button that is important for very serious reasons. It's part of citizenship in a sense, trying to do something that is right. Its usefulness is pragmatism, it's like, "I need to see this YouTube video, but they don't let me see it in Brazil so I'm just going to do it in Belgium." I think that that's what VPNs are to me. What I was saying about black box is I don't actually know if VPNs work, at the end of the day there could be some step in their server chain, their routers that happens to be in a jurisdiction, happens to be something not kosher in a sense and then the whole thing collapses. 'Cause it's just sort of an endgame of all the links in the chain. I don't know if there is some other link in the chain that is already going wrong and compromising the whole thing, and using VPNs is useless.


I think that there is a very fair chance that I'm achieving nothing as far as security goes with VPNs, but it's also a statement in a sense and I guess that that is important too.\end{quote}

We also asked our participants whether they know if and where their real IP is visible. Half (16) of the interviewees stated that they did not know, while 14 believed that their VPN provider would be able to see it. Also, 6 respondents had a sense that such information are anonymous and 2 respondents said that their Internet Service Provider (ISP) is still able to see it. Moreover, 2 respondents did not find it important.

\subsubsection{VPN Expectations of Privacy}
At least 20 interviewees did not believe that VPN guaranteed them anonymity, although 8 interviewees admitted that VPN offers something, that in their opinion, is the closest to anonymity. On the other hand, 7 interviewees thought that VPN guaranteed them privacy and 6 access to sites they wanted to visit. 6 respondents believed that nothing is guaranteed.

As P04 explained:
\begin{quote}I think it never really guarantees me it. Because even though \dots if I use even a private VPN that I paid for, even though the majority of the world does not see my IP address and everything, I feel like the owners of the VPN provider will be able to still see it. And then it's just a matter of having that security breached. So I don't think there's ever really a sense of true privacy. Unless I make my own VPN.\end{quote}


Also, P13 did not think the VPN is secure, even though it was institutional one:
\begin{quote}Probably nothing. I think it's pretty open so whatever security wherever you're [inaudible] into. Because I'm going to the university network, I know there security is there. But depending on where I've been, those securities have been better or worse. So at Rutgers ... I did my Ph.D. at Rutgers, and they had a huge hack shut down the whole system. I mean, that was just a student that did that. I don't think their securities are very good at all. Here, I haven't heard about any problems, but I don't know. Seems like the security is a little better here. But it's very easy for me to VPN and grab my IP off any of the computers in the lab and just get it onto a computer. I don't think it's that secure. There's no extra steps I have to take to do that.\end{quote}


At least 24 interviewees responded that in their opinion you can be tracked while using VPN as well, as they believed that there is always a way to do so (8) but also because you can be tracked by VPN provider itself (9). 
%Check in dedoose if P01 is in this 9:
As P01 explained:

\begin{quote}So if it is SSL encryption, the VPN provider would still know that you are communicating with a certain web service but the VPN provider would not or probably not know the contents of the communication if it's SSL encrypted. They would only know who you want to communicate with. And if it's not encrypted, then they can see, they can be do packet sniffing or even more malicious things like deep packet injection and deep packet inspection to actually look at the contents of that communication and do potential malicious things with that. \end{quote}

Also, 4 interviewees admitted that while using VPN, one could be tracked by the government. For example P30 used VPN only in different countries to access blocked content. They stopped in USA as they did not need VPN's access properties anymore and did not see any privacy protecting reasons:

\begin{quote}I don't think there is any marginal benefit to using a VPN to evade tracking. Seems like VPN's are all like, at least partially, controlled or transparent to the government.\end{quote}

 Moreover, two interviewees responded that one can still be tracked by advertising agencies. Although, 3 interviewees admitted that VPN makes tracking at least harder than normal. 

P21 explained that using VPN is not enough. In order to make tracking harder for companies, they changed locations that they connected within VPN:
\begin{quote}Yes [I can be tracked while using VPN], especially if I'm using the same IP address. That creates a problem because my internet footprint ... or like, Chrome, for example, my web browser could definitely still track me and connect that, see where I've been connecting from. Or Gmail could see that. Gmail always tells you, "Oh, you've connected from this weird device, or from this location that we don't recognize." So I think you can definitely still be tracked, so it's important not to be complacent and stick with the same IP address and obscure your web footprint in other ways in order to make it ... I think there's some stuff that's out there. There's some amount that I've already been tracked, and I can't take that back. But I can change my footprint, and try to make my devices that I'm using and the way that I connect to the internet as vanilla or ... I don't even know the right way to say it, but basically in a way that doesn't jump out and can't be profiled.\end{quote}


\subsubsection{Configure VPN - Expectations}
At least 25 interviewees admitted that they had not prepared themselves and their devices before installing their VPN.

\subsubsection{VPN practices: Keeping logs, Sharing information, Transparency - Expectations} 
\textbf{How VPNs keep data}
We asked our interviewees whether they believe that VPN providers keep their logs and other information stored. 23 interviewees answered “yes”, giving examples such as keeping information for user statistics (5) or to sell data (5). Some also referred to university’s VPN (7): 4 interviewees thought that university wants to have an access to all information and 3 believed that VPN helps university to monitor if someone’s cheating during exams etc. There were 9 interviewees who did not think that VPN providers could keep their logs, from which at least 4 were not sure but hoped they did not and 3 stated that this is not what their service is for.

\textbf{Who VPNs share data with}
We also asked our interview participants about their opinion on other VPN practices. When asked whether they thought their VPN providers could be sharing their information, 17 responded “no” and 11 “yes” but 12 were uncertain about their response or did not know the answer.  Moreover, from these who said that VPN providers do not share information with other entities, 8 confessed that they hope their information are not being shared, 5 admitted that while their VPN providers do not share any information, others do. Moreover, 2 of these interviewees believed that even though their VPN providers do not share data with others on regular basis, they would with legal authorities. 

\textbf{What VPNs tell users}
Furthermore we asked our interviewees whether in their opinion their VPN providers were transparent. 14 of our interviewees replied that they did not feel like their VPN providers were transparent, while 12 believed that they were. 

At least 15 interview participants answered that they did receive description on how to use VPN from their providers and 14 admitted that they did not receive any description.  

Moreover, 13 believed that it is important to get such description for multiple reasons; interviewees believed that such description is necessary in order to know how to make the best use of VPN (4), as well as to know what VPN provides (4). They believed that it would make usage easier (2), they would like to know what should and should not be done while using VPN (2) and how VPN works with their devices (2) and how to connect to it properly (2). On the other hand, at least 9 interviewees believed that using VPN is very easy and at least 8 did not consider important getting such description. 


\subsection{How people choose VPN} 
\label{sec:findings-choosing}
\subsubsection{Guidelines when choosing VPN}
We were also interested in factors that our participants take into account when choosing their VPN provider. 

\textbf{Reputation} For the majority of our interviewees (19) the most important was good reputation of the VPN provider and 3 interviewees added that the fact that their friends had used it before had an impact on their decision. 

\textbf{Security less important} Moreover, 10 interviewees had different security and privacy requirements: 5 made sure that VPN had secure network, 3 that VPN provider did not store any of users‘ records, 2 that VPN provider did not sell users‘ information and also 2 that VPN protect users‘ data. For one interviewee it was important that VPN did not require any personal information when setting up the account and another one that there was an option of secure payment. 

\textbf{Ease of use} Another factors considered by our interviewees were also ease of use (8), speed (7) and ease of set up (5). Moreover, 5 participants would look at the price before purchasing subscription and for 5 it was important that VPN was for free.   

For example for P11, the main factors were word of mouth, experts’ opinion, cost as well as customer service available:
\begin{quote}I tend to look at reviews. I like to think that I'd know whether I could decide a VPN was good enough or not on my own, but I tend to trust other people, so I look on, like, Tech Radar and PC Monitor, those kinds of websites, to ascertain whether experts thinks it's the best. I get some reviews from friends, as well, see if their having a good experience with VPNs, and then I'll go on website if I'll, like \dots I think I narrowed it down to a couple of options. So, when I came to China I was deciding between Express and Astro, and I just looked on their websites, went through, like, server locations, cost, and their privacy policies, as well, and then into deciding on Express.(\dots)
and then, I also had Express available customer service, which was very important as well.\end{quote}

\textbf{Cheap, Fast, Secure?} We asked our participants to rate three components: secure, fast and cheap, from the most important to least important when choosing VPN. \textit{Security} scored 77 and was the most important for our participants. Second was \textit{Fast} scoring 54 and the least important was \textit{Cheap}, with 41 points.


\textbf{Locations} We asked our interviewees whether the number of servers locations that VPN provides and to which they can connect to, is important. For 12 interviewees it was important, for example 3 reported that there are different data regulations in different countries and 2 stated that decentralization was a good thing. Another 2 interviewees simply believed that more locations give them confidence that they would stay connected: if one connection would not work, there would be another one. Two other interviewees believed that bigger number of servers’ locations is useful when travelling.  On the other hand, 11 interviewees did not think that number of servers’ location is important, as long as there would be couple of them (5).  

\subsubsection{Source of knowledge}
Half of our interviewees (16) found out about Virtual Private Network and their VPN providers through word of mouth. At least 9 interviewees learned about VPNs through their school or university and 7 through research. 4 interviewees admitted that they simply searched VPN providers through Google search engine and 2 of them used first thing in search results. Moreover, 3 interviewees were told about VPNs in stores while purchasing phone, 2 interview participants got to know about VPNs because they were travelling to China and another 2 learned about them at their jobs.

P26 learned about different VPNs through advertisements on websites such as The Pirate Bay:
\begin{quote}So typically when I'm accessing those websites, like open-source libraries and especially The Pirate Bay, they always have different ads for different VPNs. So that's where I typically get a lot of them. I'll download them, try them out. Typically, I'll stick to the free ones. So I'll test it out and see like ... the one I use now, Windscribe, it has a 10GB limit, and so that's more than enough for what I need. I've had others that an ad comes up every five minutes that you're on there. That bothered me. There was one, it would slow it down substantially, so I didn't use that either. I think the one I found right now, Windscribe, is a pretty good one.
(\dots)I'm not going to see those ads on Google or YouTube. It's on those open-source libraries that you can, that they'll advertise them because you should probably use one of those before you're downloading anything.\end{quote}

\subsubsection{Trust in VPN}
When asked interviewees how they determined whether their VPN provider was trustworthy, 13 of out interviewees said that it had good reviews online. Another 10 would verify that through word of mouth and 7 knew it was trustworthy because of the founders of their VPN, including university (4). 

\textbf{Price}
22 interviewees reported that they would use free VPNs but with many comments on it and restrictions,such as making sure that it was safe, while 9 more strictly said that they would not use it. Among these from the former group, 6 admitted that they did not need VPN, nor used it often, which is why they did not mind using free VPN. 

As P24 shared, they were not a regular user so there was no need to pay. Nevertheless, paid VPN could mean more secure VPN:
\begin{quote} I think the one that you have to pay for is more trustworthy. But, it could easily be the other way around. Just because you have to pay for something doesn't mean that it is more reliable, or even more efficient. But, I do think that the paid ones generally people might think that they are more safe to use. And that their information may be more secured, just because of that added price tag on it.\end{quote}

P03 was also suspicious of free VPNs:
\begin{quote}I would definitely try a free VPN, but at the same time, if others cost money, and this one is free, I'm like, so why aren't you? Why are you free? That'd definitely make me a little, if most of the VPNs cost money and that's one free, then I'll be suspicious. But then if it's like half of them cost money and half of them are free. I'll like, oh, maybe the ones that cost money are for people who have really, really, really sensitive information. And the ones that are free are maybe for people who are just casual internet users who want that extra level of security, but don't have a justification to need to pay money. But I'd definitely be wiling to purchase a free one if I just saw what other VPN, like I compare the reasons as to why some cost money and some didn't cost money. (...) I guess maybe the ones that cost money offer a lot more protections. I don't know if some VPNs are more secure. Maybe the ones that cost money are very, very secure, whereas the ones that are free are maybe just basic level of security, just a little bit better than what you get on the internet. Just things like that.\end{quote}

As well as P14, who was worried that installing free VPN could be unsecure:
\begin{quote}I was worried that I possibly also downloaded some virus along with the VPN, so, that's also what motivated me to just purchase ExpresVPN and just deal with the cost because I wasn't really sure if what I had and so it was from a trustworthy place whether that's a reasonable or not.\end{quote}

5 interviewees said that they would use it only if they were sure that particular free VPN is secure. For example P12, would check reviews first:
\begin{quote}Why I want free VPN? 'Cause it's not very important to me, and I don't have a lot of money, and I don't spend money on things that I don't really need. Why I think I can trust it? I mean I would read up on people's reviews of the privacy status of different VPNs to choose the one that I install. I would only install it if I get a pretty good impression from the reviews that I read.\end{quote}

\textbf{Different VPN providers} When asked whether it was important who their VPN provider was, 14 interviewees reported that they would not pay attention to this information and they simply did not care about this, while 11 interviewees reported that it was important, especially for these who used university’s VPN (7), which was reassuring for them that this VPN was safe. On the other hand, for 2 interviewees it was not important that university was their provider, as one explained, university is only a client of a different VPN provider, not a VPN provider itself. 


\subsection{Goals}

\subsubsection{Perfect VPN}
\textbf{Performance} We asked our interviewees what, in their opinion, "perfect" VPN would provide. 9 interviewees listed that they would like their VPN to operate at the same speed and latency as  “normal” Internet. For at least 8 interviewees reliability was a key component of a perfect VPN and for 7 it was security. 

\textbf{Security} Also 6 interviewees mentioned that perfect VPN would not access any personal information, 6 that it would not hand over or sell data to other companies. 

\textbf{Data practices} At least 5 interviewees believed that VPN should be transparent and also 5 that it should serve only users' interest. But as P25 observed, this could not make usage better, since users do not get information about their data being stored anyway.


\textbf{Access} Moreover, 8 interviewees believed that number of locations would be important, 7 added that perfect VPN would keep offering location spoofing and 4 that ideal VPN would provide an access to absolutely everything.

P18 described their perfect VPN:
\begin{quote}I would look to some sort of a VPN that operates in another country, free from the jurisdiction of whatever country I'm residing in, because I know that in some countries it's harder for the United States to work with a VPN or whatever companies if the VPN is hosted in a country that's not really friendly towards the United States, that would be interested in having people from the government of the US working in their country. I think that would be a big plus.


I think another big plus would be that it's not relatively well known, ironically. That goes against sort of what I said before, that it has good reviews. At the same time, I don't want many other people knowing about the existence of that VPN because I don't want it coming up on the radar of say the US government scouring these VPN's looking to get information on people. It's also not profitable. It's not profitable just to go for people who are downloading some movies or something like that, to go halfway across the world just to get a few IP's that are going to lead nowhere, relatively speaking. It's not profitable.\end{quote}


\subsubsection{Possible improvements}
When asked what kind of improvements our interviewees would like to see, half of them reported more transparency from the VPN provider side (16) and 14 proposed more education effort for users to know how VPN works (11), reasons why they should use VPN (3) and about online security overall (2), since there are many people without sufficient technological background to know it (7). 

\begin{quote}So it would be nice if the VPN providers themselves had more of an analysis of what kinds of tracking is generally being done to you if you are say in the US or if you're say in Europe.It would be nice if the VPNs themselves made a better case for using them. Like if you just go to Private Internet Access's website, it's kind of preaching to the choir, you know? It says, "Browse anonymously. Keep your IP address cloaked. Defend yourself from data monitoring," whatever. But it'd be nice to have like very concrete examples of, you know, "In the US, you could get in trouble for doing this, that, or the other on the Internet, and it becomes impossible to get in trouble for that if you use this."

I guess they can't exactly endorse illegal activity like that, but they could use other examples, like say Amazon is tracking your buying history. Well, I guess you still have to be logged into Amazon, but even if you just search on Amazon, they can track your searches or whatever. So you know, things like that. (P02)\end{quote}


Also, 5 interviewees wanted VPN providers to communicate what they do with their logs. Furthermore, 8 interviewees would like VPN to be more accessible and 5 more user-friendly. 

P13 encountered problems while trying to set up VPN with their friend:
\begin{quote}There were instructions online on how to download it. But they don't have a very good system. I was actually trying to help my friend over the phone the other day with her Apple computer, and we couldn't get it to work. (\dots)They were missing some key steps. (\dots)That were really important, like where to find certain things if you don't use your computer very often and because I'm not an Apple user, I had trouble being like well, I know in Windows, you go here and here. I don't know how to find that in your computer. So we had to Google the in between steps. (\dots)I think a lot of tech documents assume that most people are expert users of their computer. (\dots) It should have text. It should have pictures so people know what it looks like when they get there and less computer jargon the better.\end{quote}

Moreover, P13 explained how he would like to change security and access permission around institutional VPN:
\begin{quote}Okay. I just would have more security around it. It just doesn't feel secure to me. It doesn't ... it's too easy to access, too easy for me to get into pretty much anything I wanted here, which is good for my work for bad if I wasn't doing work. Let's say I had a vindictive student who then could go in and delete all our files or put some kind of virus in the computer in the lab. There's no protections against that. They have full access just as much as we do. (…) It's backed up theoretically, but there's still limitation to that backup. The backup is every 12 hours it backs up. Let's imagine you just recorded 12 hours of data and then you piss off a student and then he goes in and deletes it all. There's nothing stopping them from doing that. 
(\dots)You can put a no delete on your files for certain people, permissions for certain people. But then every time somebody joins the lab and has access to that system, you would have to do that and it just requires a lot more work. Some people have their files completely blocked that you can't access them, but that makes it hard to collaborate and with collaboration, there has to be some sense of trust.\end{quote}



\subsection{Usage}
\subsubsection{First experience}
Our interviewees were introduced to VPN in many different ways. At least 8 started to use VPN at institutions like university, 4 were looking for ways to bypass geographic firewalls, so they could watch something, 3 found out about this through their parents and 2 were told to use it when they were going to China. Three interviewees reported that they could not remember their first encounter with VPNs.  
At least 9 interviewees described their first experience with VPN as not good, for various reasons. Some could not connect (2), for some it was too slow (2) and some found it confusing (2).  On the other hand, at least 2 interviewees found it simple to use for the first time.

%How do you use it
%7 interviewees reported that they used it only when they needed. Should we talk about it? It’s out of 14. And vs 4 always on.


\subsubsection{Using institutional VPN} 
Out of these who used institutional VPN, 5 reported that they would use it only for work, because they simply did not feel private (3), they felt that university
could track them (2) and institutional network was vulnerable  (2). P13 gave an example about the student who shut down exams by hacking into university's VPN through another VPN. %ADD HERE THE QUOTE ABOUT STUDENTS HAVING ACCESS TO POSTDOC’S WORK.

On the other hand, 5 interviewees used institutional VPN for private activities, like browsing, as well. For example, one of them believed that it added security, and one would simply not switch it off. 

\subsubsection{Privacy policy, have you read?}
We also wanted to know whether our interviewees read their VPN providers’ privacy policies. 24 interviewees replied that they did not read it and 11 of them added that they never read any privacy policies. Also, for 11 interviewees it was too long and not readable. Moreover, 11 of the interviewees who did not read it, believed that it was not important to read it, for various reasons; 5 believed that it did not matter whether they would read it or not, because either way you have to trust your VPN provider (3) or because they would collect their data no matter what they would write in their privacy policy (2). 2 interviewees believed that they did not have any valuable information to be worried about and one believed that reading it would only make them more confused. Among all interviewees, 6 believed that providing and reading privacy policy was important. On the other hand 5 interviewees read their VPN provider privacy policy but 3 of them only looked through it, one did not find anything specific and one found it standard. 

For example, P11 read it after some of his friends had bad experience with a different VPN:
\begin{quote}when I was renewing my subscription, and I had, like, a look on the internet to see what the privacy policy is, and the terms and conditions and things like that, just to make sure that I was definitely getting the best value \dots It was mainly the best value for money, and just to make sure that everything was as above-board as it could be, in terms of security. (\dots)I had a good look, because I'd heard from a couple of friends who were using \dots I think it was Onavo, and that they were kind of worried that it was owned by Facebook, and it was kind of influencing what they were getting adverts for, I think. (\dots) So, I didn't want to have that lingering over me when I was using a VPN. (\dots) It [ExpressVPN] seemed like it was the best kind of \dots It was the best kind of policies, and I did some, like, online research, people who know a bit more about this thing than me.\end{quote}

\subsubsection{Reasons for VPN usage}
The majority of our interviewees (21) reported that they use VPN to bypass geographic firewalls, 15 of which to watch movies or TV shows online.  

\textbf{High school} For example P20 used it to get access to sites that were blocked by his high school: 
%(also interesting word of mouth example and not allowing apps to see %location thinking that they can do it anyway): 
\begin{quote}So I don't use many VPN's anymore. I used them when I was in India, and I used them \dots Okay, yeah. I've used them for a few reasons, but privacy was never really one of them. It was just when my content was restricted (\dots) when I was in boarding school, I went to boarding school for high school. Our WiFi was very tightly patrolled. So any number of things were blocked, like from adult content, to a lot of sports websites for instance were blocked, because they "encouraged gambling," and I like to watch a lot of sports online illegally, because that was the only way I could watch them. So to get around that especially, but also just to not have the school looking at everything you were doing. (\dots) 

And then I came in my sophomore year and heard about the VPN. I mean, that was great because it allowed us to use the internet after 12:00 to message our girlfriends or whatever. Just stuff we couldn't do before. So that was big. But then I think someone must have tipped off the IT department again with the incident I discussed with a student illegally downloading movies. I think that they sort of realized how many people were circumventing the internet, and because my school had liability for any number of things, they didn't want someone's parents complaining, and then they have to deal with that.


So then after my sophomore year, that summer we came back, and all the Chrome add on VPN's that we had been using, basically just weren't effective anymore. I really have no idea how they did that.\end{quote}


\textbf{Geographic blocking} On the other hand P24 used VPN not only to watch TV show but also for accessing their app account that s/he subscribed to while being in America and could not use it at home country:
\begin{quote}I was in New York this past summer, and so, I had a subscription for a couple months. But then I came back to Vancouver, and wanted to end the subscription. But, I couldn't, I found out that I couldn't access the app or login to my account through the phone app, because I wasn't in the United States, basically. So, I just thought like, hey maybe if I use my phone to change my IP address, so it looks like I was accessing the app from the United States. And I downloaded, I think it's called Express VPN. And that was able to help me work around the location, geographical issue, and access the account so I could cancel the subscription. So, that was a pretty useful tool for me.\end{quote}


P11 used VPN for news websites that were not blocked but had different or limited content depending on IP address of the Internet user. As they were from UK, they wanted to get into UK BBC website, while being located in the US.

Also P25 shared that s/he used VPN in order to obtain different goods and for piracy and for geo blcoking reasons:
\begin{quote}I don't use it for privacy much anymore, I use it as a way to separate my own identities mostly, I use it when I need to be in another country theoretically. I think that that's the most common way to use it for me, if I go and I try to see a video and it's not available in my country I just scroll down from Brazil to Belgium and usually it works. I think that that's what I mostly use VPNs for. Or if I'm doing something that I feel I wouldn't want TNSA to see, TNSA is an abstract thing in my head. If I didn't want the Panopticon to see it, I would turn on the VPN but I don't think that ever happens.\end{quote}

Moreover, 7 interviewees would use it to access materials that normally they would have to pay for, 5 admitted they used VPN for piracy and 3 for downloading content.

P26 also used it for their relative to be able to watch a tv show:
\begin{quote}It masks my IP address from my internet service provider, and through that I can access content that maybe they wouldn't want me to access that's blocked from their networks, or maybe access things that they wouldn't be happy with me accessing. Like with the university, there are certain free textbook websites that they don't like students going on, but then I use my VPN to be able to access those sites. For example, Venezuela has blocked everything coming from their YouTube channels, and I have my mom reroute the US IP address to a Mexican IP address with a VPN, so then she could watch her Venezuelan TV shows. (\dots) It's things like that that I mainly use it for.
(\dots)Or like watching Netflix from another country. Things like that. (\dots)Sometimes there are shows that are only accessible from different countries. Like, for example, if you were to watch the \dots if you were to access Netflix from, let's say, Brazil, you'd have a lot more Portuguese language shows than you would on the American version of Netflix. So for things like that it's very helpful. And also sometimes the movies that are released on American Netflix aren't the same movies that are released on, let's say, Spanish Netflix, and so you can watch those. Even American-made shows are sometimes licensed differently across different countries. That's when it's really good to use an international VPN like that.\end{quote}

\textbf{VPN is not used for Privacy} 13 interviewees said that privacy was not their reason to use VPN. Fewer (7) participants used it to protect their personal information, as for example P21:
\begin{quote}I guess I don't like the idea of Princeton or an ISP being able to see all of my traffic. I don't think that I trust anyone with all of my traffic or consumer habits. I don't want that stuff to be sold off. I don't want them to be able to build a profile of me because it can be quite revealing, especially because you're device\end{quote}


Also, 5 interviewees would use it while on public Wi-Fi and 4 while travelling. 5 reported that they used it for security and 3 for anonymity.  Moreover 3 interviewees used it because they liked the idea that there was a "free" space on the Internet and P25 believed that using VPN is a statement that security is important:

\begin{quote}it's why Private Network Access I think? (...) It's why it got so popular. They tried to subpoena the guys to release information about some of the people who used the VPN, and then they actually didn't have it on their servers. So people knew that they didn't keep records, so everybody started using that one. Then using VPN is something that gives me information about how the game is played, like whether the people who run VPNs are vulnerable to these legal mechanisms that might be used by grouped institutions or legitimate institutions. It just helps make it clear in my head how the game works.\end{quote}


19 interviewees would use it for work, out of which 16 used it in their university or school. P21 used it, so their boss could not be able to see if they were online at work.


\textbf{Why others use VPN}We also asked what other people's reasons to use VPN are. 5 believed that others use VPN for privacy. 11 interviewees replied that others use it for watching something online and 3 that for piracy.

As P18 described what VPN is:
\begin{quote}
Just a way for people to get around government agencies looking at their IP address so that they can download whatever they want from wherever they want within reason because these VPN's are also subject to their own vulnerabilities. These IP addresses, some VPN's have policies where they will provide information from IP addresses if they're requisitioned from a government entity. That's pretty much it with regards to VPN's, just used to download information. Tore isn't really used for that because it's slower, because of the technical architecture of Tore. VPN's are just used mostly for that.\end{quote}

and why people would use it:
\begin{quote}Oh, so VPN's with regards to downloading things illegally. People use VPN's to maybe get copyrighted material but don't want their IP address to be tracked by law enforcement agencies or recorded by their internet service provider. Just a plain old dummy would probably go somewhere, to a file hosting site like the pirate bay or mega upload or something like that, that have lists of torrents. (\dots) I also know that VPN's are used in engaging illicit commerce on websites that \dots A famous example is the silk road which was taken down back in 2013. \end{quote}

\textbf{Usage in different countries}
%Think about home country/other countries 
P15, one of our interviewees from Romania, believed that people from US are more afraid of using VPN than in their home country:
\begin{quote}I feel like people are more likely to buy their stuff rather than try to get it for free, while at home no one cares, so they're just gonna go for the free thing because it's easier and you're never gonna see someone getting accused for, of a, and getting a criminal record for downloading a game. It's only if you did some sort of, I don't know, WikiLeaks type of thing. But, yeah.\end{quote}

\textbf{VPN vs. No VPN}
10 interviewees reported that VPN did not change their Internet usage and online experience, 8 did not recognize many differences between using and not using VPN and 7 did not see any difference while browsing with or without VPN. On the other hand, 10 interviewees saw one important difference: VPN allowed them to access content they could not access before. Moreover, 8 believed that VPN makes their online experience more secure but 2 believed that VPN is actually more vulnerable and 3 admitted that they left their guard down when using VPN. Also, 2 interviewees found that Internet connection is slower while using VPN.

\subsubsection{Problems}
We asked our interview participants what kind of problems they encountered while using VPN. 
\textbf{Accessing content} 7 interviewees said that VPN would disconnect, 6 that it was slow, another 6 that they could not access the content they wanted because some websites knew they were using VPN. 

%Check if P08 in dedoose in this 6, ale Usage by others

For example P08 was annoyed by the fact that they need to switch off the VPN while using Netflix, otherwise platform would not allow to access the content

\begin{quote}So, the only use cases I've seen mentioned in Germany was to, for example, when Netflix wasn't available in Germany, people would try to use-
(\dots)It is [available] now, but in the beginning, things like Netflix were only available in the US, so people would try to use it and then they needed a DNS unblocker or a VPN to use Netflix. I don't think that works anymore. One of the reasons that I'm not using VPNs as much anymore is that a lot of these services just block the IP ranges of these data centers, so whenever I want to watch something on Netflix, I have to turn off the VPN, which is annoying. (\dots) I don't like it. I understand why they are doing, basically being pressured by \dots I think this whole idea that they buy content for specific regions and then content that's region-blocked is stupid, but that's the world we live in and this how their contracts and everything are set up, so they have to do something to prevent people from regions to access that content. I mean, for me, it's just annoying, because then I have to switch off the VPN. \end{quote}


4 interviewees mentioned that installing VPN was problematic, because it was a complicated process (1). 16 reported that they had no problems at all. 

\textbf{Checking if VPN is working}
We also wanted to know how our interviewees check whether their VPN is actually working while they are using it. 10 interviewees replied that they get information on their screen that VPN is connected, 9 that since they use VPN for websites that otherwise are blocked, having an access means that they are connected. 6 interviewees actually checked their IP address to make sure VPN is working, 3 looked at VPN icon and 3 said that they never check. 




\section{Discussion}\label{discussion} 

Our study highlights potential areas of improvement for VPN messaging, design,
and policy. We organize our presentation of these areas by the same topics as
the Results section. We further denote the implications our findings have for
stakeholders in the VPN usage process, including educators, VPN providers, VPN
designers, and policymakers. We also present potential future studies and
address limitations.

\subsection{Mental Models, Perceptions, and Expectations of VPNs}
\subsubsection{Educators}

Educators should introduce privacy and security to students earlier as an
important concern. As previously discussed, Internet usage is exploding, and
online technologies are becoming more and more integrated with daily life.
While companies have rapidly developed tools to accommodate and exploit this
phenomenon, the American education system has largely not caught up with
online technologies [29]. A higher rate of Internet literacy in today’s
digital age would be beneficial for a multitude of reasons, including popular
concern for online privacy and security risks. We observed that there was a
general feeling of indifference towards data collection, both in terms of
perception and tool usage. Improved curriculum on Internet technology,
including not only how user data is collected and used online through case
studies but also what people can do to protect against it, could spur the next
generation of Internet users to be more privacy and security conscious.

We observed that students who use other privacy and security tools earlier are
more likely to use VPN earlier. However, we also found that students who were
introduced to VPN at a younger age were more likely to use potentially
dangerous free VPNs, likely because they view high cost as a larger barrier
than privacy and security concerns [7]. Students who are more concerned about
data collection are also more likely to habitually use VPN for privacy and
security reasons. Improved education on the risks of online data collection at
a younger age could make privacy and security higher priorities in students’
browsing habits, including their usage of protective online tools.

\subsection{VPN Usage} \subsubsection{VPN Providers}

VPN providers should educate their users more effectively on what VPNs are and
what they do. We observed that students who learned about how their VPNs
worked through online research or directly from their providers tend to be
more diligent about VPN usage and more conscious about VPN technology than
those who learned from other sources. As students cannot be forced to do their
own research, this knowledge gap should be addressed by VPN providers who can
do more to inform users of VPN functionality and purposes. Anecdotally,
searching for definitions of VPNs online yields many different results that
could prove to be confusing even for tech-savvy users; some explanations are
very technical, and others are simplified to the point of being misleading.
Having authoritative and clear answers from their VPN providers could help
users gain better knowledge and understanding of VPNs. Students largely see
VPNs as censorship circumventors and location spoofers, and less so as a tool
for privacy and security. However, privacy and security are important factors
that students consider when choosing a VPN; furthermore, students who use VPNs
for privacy and security reasons are more likely to keep using VPN. Thus,
commercial VPN providers are incentivized to educate users on these factors
and highlight them as features, as this could improve user acquisition and
retention. This would have the added benefit of increasing users’ perception
of VPN transparency.

\subsubsection{Policymakers}

As the threat of large-scale data privacy violations is now omnipresent,
technology companies’ privacy policies have come under scrutiny by governing
bodies. With existing US laws on individual privacy rendered largely outdated
by the information age, policymakers are incentivized to create updated
regulations that will protect consumer privacy starting with consumer data
[31]. In 2018, the EU Parliament passed the “General Data Protection
Regulation,” which bolstered individual privacy rights by introducing new
rules and regulations on companies requiring them to request consent,
anonymize collected data, and notify consumers about data breaches [30]. The
US government has yet to come up with similar legislation on the federal level
[31].

While policy to help combat the misuse of data by corporations would be
welcomed, policymakers could additionally make efforts to encourage individual
usage of privacy- and security-enhancing online tools. One way to achieve this
is to build Internet literacy into secondary school curriculum, as discussed
in section 5.1.1. However, we have highlighted that students who adopt VPNs
earlier tend to use free VPNs, which can be dangerous, invasive of privacy,
and malware-infested [7]. Requiring VPNs and other online tool providers to
disclose their data collection and sharing practices would deter educated VPN
seekers from choosing dangerous options. In tandem with improving educational
standards, policymakers have an opportunity to increase the prevalence of
safer VPN options for individual consumers.

\subsection{VPN Issues and Improvements} \subsubsection{VPN Providers}

We observed that cost is a key reason why many students choose to stop using
paid VPNs. Paid VPN providers should fix the messaging of their products, as
discussed in section 5.2.1, to properly inform users of the benefits of their
products, then do pricing analyses to optimize profit.

\subsubsection{VPN Designers}

Many students indicated that accessibility, ease of setup, and ease of use
were important considerations when choosing VPNs. However, we also observed
that many students had issues with the interfaces of their VPNs, showing that
the current offerings are not satisfying user expectations in this area. While
other popular issues such as speed, stability, and cost can be difficult to
resolve quickly at scale, designers can improve user satisfaction and
retention by improving VPN user interfaces. Key complaints that were repeated
throughout respondents’ complaints involved installation and setup processes,
login processes, and behavior when auto-reconnecting. Furthermore, a number of
respondents do not understand what their VPNs are doing. In addition, students
who verify that their VPN is working generally do so through passive methods
such as viewing alerts from the VPN or viewing the VPN’s tray or taskbar icon.
These situations could be improved with design elements that better
communicate the actions of the VPN, which could additionally improve user
perception of VPN transparency.


\section{Conclusion}\label{sec:conclusion}

Data privacy is an increasingly important concern for Internet users. However,
most Internet users today do not fully understand the Internet [18, 19], and
the US education system does not provide sufficient background on online
technologies [29]. Virtual Private Networks (VPNs) are important tools that
can help safeguard against data-intrusive practices. Although VPNs are growing
in usage and prevalence [3], we lack an understanding of how users perceive
VPNs and why they use them. Our study focuses on college students and asks the
following research questions:

TODO

We employed a mixed-methods approach involving interviews, a survey, and a lab
study.

We found that students have weak understandings of VPNs, both in terms of what
they are and how they work. They use VPN infrequently and primarily for
content access, regarding privacy and security as secondary concerns. Students
consider cost, ease of use, and speed as the most important elements of a VPN.
Most college students use school-provided VPNs, though a substantial number
use commercial VPNs. Students are generally dissatisfied with the speed,
stability, and interfaces of their VPNs.

Our findings have implications for educators, policymakers, and VPN providers
and designers. Educators and VPN providers should become better resources for
users to learn about online privacy and security. VPN providers should also
educate its users more effectively on VPN technology, and also run pricing
analyses. In tandem with other policies in the online privacy space,
policymakers should explore transparency regulations for VPN providers to
disclose their own privacy violations. VPN designers should improve the user
experience of the installation, login, and reconnecting processes of VPNs;
they should also develop designs that better communicate what VPNs are
actively doing.

Our study is a first foray into the perception and usage of VPNs, and
establishes a broad overview of the space through the lens of college
students. Our work provides insights that can inform practical improvements in
policy and design. However, we also highlighted many areas for further
exploration. Future researchers can take deeper dives into the areas that we
have introduced, or corroborate our design recommendations through
comprehensive usability studies. In the long term, Internet users stand to
benefit from a stronger, better-communicated, and better-received VPN
ecosystem.


\end{document}
